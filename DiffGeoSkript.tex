\documentclass[a4paper,oneside,11pt,DIV=12,parskip=half]{scrartcl}
\usepackage[ngerman]{babel}
\usepackage[utf8]{inputenc}
\usepackage[T1]{fontenc}
\usepackage{microtype}
\usepackage{lmodern}
\usepackage{amsmath}
\usepackage{amssymb}
\usepackage{showkeys}
\usepackage{amsthm}
\usepackage{booktabs}
\usepackage{graphicx}
\usepackage{listings}
\usepackage{enumerate}
\usepackage{hyperref}

\title{DiffGeo}
\author{ Luka Ili\'{c}, Johannnes Mader, Jakob Deutsch, Fabian Schuh}





\newcommand{\R}{\mathbb R}


\newenvironment{definition}{\textbf{Definiton.} ~~~~}{}
\newenvironment{note}{\textbf{Bemerkung.} ~~~~}{}
\newenvironment{proposition}{\textbf{Proposition.} ~~~~}{}
\newenvironment{lemma}{\textbf{Lemma.} ~~~~}{}
\newenvironment{theorem}[1][]{\textbf{Satz.} #1~~~~}{}
\newenvironment{example}{\textbf{Beispiel.} ~~~~}{}
\newenvironment{lemma, definition}{\textbf{Lemma und Definition.} ~~~~}{}
\newenvironment{note, example}{\textbf{Bemerkung und Beispiel.} ~~~~}{}

\begin{document}
	
	\maketitle
	
	\pagebreak
	
	\tableofcontents
	
	\pagebreak
	
\section{Kurven}
\subsection{Parametrisierung und Bogenlänge}

Wiederholung: Ein Euklidischer Raum $\mathcal{E}$ ist:
\begin{enumerate}
	\item Ein affiner Raum $(\mathcal{E},V,\tau)$ 
	\item über einem Euklid. Vektorraum $(V,<,>)$ .
\end{enumerate}

	Dabei: $\tau: V\times \mathcal{E} \rightarrow \mathcal{E}; (v,X) \mapsto \tau_v(X)=:X+v$ genügt
	\begin{enumerate}
		\item $\tau_0 = id_{\mathcal{E}}$ und $\forall v,w \in V ~ \tau_v \circ \tau_w = \tau_{v+w}$
		\item $\forall X,Y \in \mathcal{E} \exists! v \in V ~ \tau_v(x) = Y$ ((d.h. $\tau  \textit{ ist einfach transitiv}$)).
	\end{enumerate}
	
	
\begin{definition}
	Eine \textit{(parametrisierte-) Kurve} ist eine Abbildung \[X: I \rightarrow \mathcal{E}\] auf einem offenen Intervall $I \subseteq \R$, die regulär ist (d.h. $\forall t \in I ~ X'(t) \not = 0$).
	Wir nennen $X$ auch Parametrisierung der Kurve $\mathcal{C} = X(I)$.
\end{definition}

\begin{note}
	Alle Abbildungen in dieser VO sind beliebig oft differenzierbar (d.h. $C^{\infty}$).
\end{note}

\begin{example}
	Eine \textit{(Kreis-) Helix} mit Radius $r>0$ und Ganghöhe $h$ ist die Kurve
	\[X: \R \rightarrow \mathcal{E}^3; t \mapsto X(t) := O + e_1r\cos(t) + e_2rsin(t) + e_3ht. \]
\end{example}

\begin{definition}
	\textit{Umparametrisierung} einer param. Kurve $X: I \rightarrow \mathcal{E}$ ist eine param. Kurve
	\[\overline{X}: \overline{I} \rightarrow \mathcal{E}; s \mapsto \overline{X}(s)=X(t(s)),\]
	wobei $t: \overline{I} \rightarrow I$ eine surjektive, reguläre Abbildung ist.

\end{definition}

Motivation: Für eine Kurve $t \mapsto X(t)$ 
\begin{enumerate}
	\item X'(t) ist \textit{Geschwindigkeit(-svektor)} (''veloicity''),
	\item |X'(t)| ist (skalare) Geschwindigkeit (''speed'').
\end{enumerate}

Rekonstruktion durch Integration:
\[X(t)= X(o) + \int_{o}^{t}X'(t)dt\]

und die Länge des Weges von $X(0)$ nach $X(t)$:
\[s(t) = \int_{o}^{t}|X'(t)|dt\]

\begin{definition}
	
	Die \textit{Bogenlänge} einer Kurve $X: I \rightarrow \mathcal{E}$ von $X(0)$ für $o \in I$, ist
	\[s(t) := \int_{o}^{t}|X'(t)|dt\] (wobei $\int_{o}^{s}|X'(t)|dt$ auch als $\int_{o}^{t} ds$ geschrieben wird)
	
\end{definition}

\begin{note}
	Dies ist tatsächlich die Länge des Kurvenbogens zwischen $X(o)$ und $X(t)$, wie man z.B. durch polygonale Approximation beweist (s. Ana2 VO)
	Also: Die Bogenlänge zwischen zwei Punkten ist \textit{invariant} ("ändert sich nicht")
	unter Umparametrisierung.
\end{note}

\begin{lemma, definition}
	Jede Kurve $t \mapsto X(t)$ kann man nach Bogenlänge (um-) parametrisieren, d.h. so, dass sie konstante Geschwindigkeit $1$ ($|X'(t)|\equiv 1$) hat.
	Dies ist die \textit{Bogenlängenparametrisierung} und üblicherweise notiert $s \mapsto X(s)$ diesen Zusammenhang.
\end{lemma, definition}
\begin{proof}
	Wähle $o \in I$ und bemerke \[s'(t) = |X'(t)| > 0.\]
	Also ist $t \mapsto s(t)$ streng monoton wachsend, kann also invertiert werden, um $t= t(s)$ zu erhalten: Damit erhält man für 
	\[\overline{X}:=X\circ t\]
	\[|\overline{X}'(s)|= |X'(t(s))|*|t'(s)| = \frac{s'(t)}{s'(t)}= 1,\]
	d.h. $\overline{X}$ ist nach Bogenlänge parametrisiert. (nämlich durch Division mit der Inversen.)
\end{proof}

\begin{note}
	Eine Bogenlängenparametrisierung ist eindeutig bis auf Wahl von $o$ und Orientierung.
\end{note}

\begin{example}
	Eine Helix \[t \mapsto X(t) = O + e_1r\cos(t)+e_2r\sin(t)+ e_3ht\]
	hat Bogenlänge \[s(t) = \int_{O}^{t} \sqrt{r^2 + h^2}dt = \sqrt{r^2+h^2}*t\]
	und somit Bogenlängenparametrisierung \[s \mapsto \overline{X}(s)= O + e_1r\cos\frac{s}{\sqrt{r^2+h^2}}+e_2r\sin\frac{s}{\sqrt{r^2+h^2}}+ e_3\frac{hs}{\sqrt{r^2+h^2}}.\]
\end{example}

\begin{note, example}
	
	Üblicherweise ist es nicht möglich eine Bogenlängenparam. in elem. Funktionen anzugeben: Eine Ellipse\[t \mapsto O + e_1a\cos(t)+e_2b\sin(t) (a>b>0)\]
	hat Bogenlänge
	\[s(t) = \int_{O}^{t} \sqrt{b^2 + ( a^2-b^2)\sin(t)}dt,\]
	dies ist ein elliptisches Integral, also nicht mit elem. Funktionen invertierbar.

\end{note, example}



\end{document}