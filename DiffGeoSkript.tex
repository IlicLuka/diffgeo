\documentclass[a4paper,oneside,11pt,DIV=12,parskip=half]{scrartcl}
\usepackage[ngerman]{babel}
\usepackage[utf8]{inputenc}
\usepackage[T1]{fontenc}
\usepackage{microtype}
\usepackage{lmodern}
\usepackage{amsmath}
\usepackage{amssymb}
\usepackage{showkeys}
\usepackage{amsthm}
\usepackage{booktabs}
\usepackage{graphicx}
\usepackage{listings}
\usepackage{enumerate}
\usepackage[usenames, dvipsnames]{color}
\usepackage{hyperref}

\title{DiffGeo}
\author{ Luka Ili\'{c}, Johannnes Mader, Jakob Deutsch, Fabian Schuh}





\newcommand{\R}{\mathbb R}
\newcommand{\N}{\mathbb N}
\newcommand{\E}{\mathcal E}
\newcommand{\s}{\mathcal S}

\newcommand{\pmat}[1]{\begin{pmatrix}
		#1
\end{pmatrix}}
\newcommand{\abs}[1]{\left| #1\right|}

\newcommand{\skal}[1]{\left \langle #1 \right\rangle}
\newcommand{\D}[1]{\nabla_{\!\!{\partial\over \partial #1}}}

\theoremstyle{plain}
\newtheorem{theorem}{Satz}[section]
\newtheorem{lemma}[theorem]{Lemma.} 
\newtheorem{proposition}[theorem]{Proposition.}  
\newtheorem{corollary}[theorem]{Korollar.}

\theoremstyle{definition}
\newtheorem{definition}[theorem]{Definition.}
\newtheorem{remark, definition}[theorem]{Bemerkung und Definition.}
\newtheorem{lemma, definition}[theorem]{Lemma und Definition.}  
\newtheorem{theorem, definition}[theorem]{Satz und Definition.}  

\theoremstyle{remark}
\newtheorem*{remark}{\textbf{Bemerkung}}
\newtheorem*{example}{\textbf{Beispiel}}
\newtheorem*{remark, example}{\textbf{Bemerkung und Beispiel}} 


\begin{document}

\maketitle

\pagebreak

\tableofcontents

\pagebreak

%\section*{Vorwort}
%\addcontentsline{toc}{section}{Vorwort}
Das folgende Skriptum ist begleitend zur Vorlesung Differentialgeometrie gehalten von Univ.Prof. Hertrich - Jeromin und wird von einigen Studenten (oben angeführt) während der Vorlesung geschrieben und danach auf Fehler kontrolliert und bearbeitet. Natürlich schleichen sich nach Möglichkeit Fehler ein, die übersehen werden, dies ist gerne bei den schreibenden Personen anzumerken.
Das Skriptum enthält großteils das Tafelbild der Stunden und keinenfalls die Garantie in irgendeiner Weise vollständig zu sein. (Wir geben unser Bestes.)

\textbf{Viel Vergnügen mit DiffGeo!}

\begin{remark}
	\textbf{Literaturempfehlung} (zusätzlich): \begin{enumerate}
		\item Strubecker: Differentialgeometrie I - III; Sammlung Göschen
		\item Spivak: A comprehensive Introduction to Diffenrential Geometry I - V; Publisher Perish
		\item O'Neil: Semi.Riemannian Geometrie; Acad. Press
		\item Hicks: Notes on Differential Geometry (Es gibt (möglicherweise nicht legale) Versionen im Internet.)\\
		(ersteres ist kompakter, zweiteres eher komplementär gedacht, drittes für Physik-Interessierte, letzteres vergleicht die Methoden der Differentialgeometrie)
	\end{enumerate}
\end{remark}
%\pagebreak
%
%\section{Kurven}
%\subsection{Parametrisierung und Bogenlänge}

Wiederholung: Ein Euklidischer Raum $\mathcal{E}$ ist:
\begin{enumerate}
	\item Ein affiner Raum $(\mathcal{E},V,\tau)$ 
	\item über einem Euklid. Vektorraum $(V,<,>)$ .
\end{enumerate}

	Dabei: $\tau: V\times \mathcal{E} \rightarrow \mathcal{E}, \quad  (v,X) \mapsto \tau_v(X)=:X+v$ genügt
	\begin{enumerate}
		\item $\tau_0 = id_{\mathcal{E}}$ und $\forall v,w \in V ~ \tau_v \circ \tau_w = \tau_{v+w}$
		\item $\forall X,Y \in \mathcal{E} \exists! v \in V ~ \tau_v(x) = Y$ ((d.h. $\tau  \emph{ ist einfach transitiv}$)).
	\end{enumerate}
	
	
\begin{definition}
	Eine \emph{(parametrisierte-) Kurve} ist eine Abbildung \[X: I \rightarrow \mathcal{E}\] auf einem offenen Intervall $I \subseteq \R$, die regulär ist (d.h. $\forall t \in I ~ X'(t) \not = 0$).
	Wir nennen $X$ auch Parametrisierung der Kurve $\mathcal{C} = X(I)$.
\end{definition}

\begin{remark}
	Alle Abbildungen in dieser VO sind beliebig oft differenzierbar (d.h. $C^{\infty}$).
\end{remark}

\begin{example}
	Eine \emph{(Kreis-) Helix} mit Radius $r>0$ und Ganghöhe $h$ ist die Kurve
	\[X: \R \rightarrow \mathcal{E}^3, \quad t \mapsto X(t) := O + e_1r\cos(t) + e_2rsin(t) + e_3ht. \]
\end{example}

\begin{definition}
	\emph{Umparametrisierung} einer param. Kurve $X: I \rightarrow \mathcal{E}$ ist eine param. Kurve
	\[\widetilde{X}: \widetilde{I} \rightarrow \mathcal{E}, \quad s \mapsto \widetilde{X}(s)=X(t(s)),\]
	wobei $t: \widetilde{I} \rightarrow I$ eine surjektive, reguläre Abbildung ist.

\end{definition}

Motivation: Für eine Kurve $t \mapsto X(t)$ 
\begin{enumerate}
	\item X'(t) ist \emph{Geschwindigkeit(-svektor)} (''velocity''),
	\item |X'(t)| ist (skalare) Geschwindigkeit (''speed'').
\end{enumerate}

Rekonstruktion durch Integration:
\[X(t)= X(o) + \int_{o}^{t}X'(t)dt\]

und die Länge des Weges von $X(0)$ nach $X(t)$:
\[s(t) = \int_{o}^{t}|X'(t)|dt\]

\begin{definition}
	
	Die \emph{Bogenlänge} einer Kurve $X: I \rightarrow \mathcal{E}$ ab $X(o)$ für $o \in I$, ist
	\[s(t) := \int_{o}^{t}|X'(t)|dt\] (wobei $\int_{o}^{s}|X'(t)|dt$ auch als $\int_{o}^{t} ds$ geschrieben wird)
	
\end{definition}

\begin{remark}
	Dies ist tatsächlich die Länge des Kurvenbogens zwischen $X(o)$ und $X(t)$, wie man z.B. durch polygonale Approximation beweist (s. Ana2 VO)
	Also: Die Bogenlänge zwischen zwei Punkten ist \emph{invariant} ("ändert sich nicht")
	unter Umparametrisierung.
\end{remark}

\begin{lemma, definition}\label{umpar}
	Jede Kurve $t \mapsto X(t)$ kann man nach Bogenlänge (um-) parametrisieren, d.h. so, dass sie konstante Geschwindigkeit $1$ ($|X'(t)|\equiv 1$) hat.
	Dies ist die \emph{Bogenlängenparametrisierung} und üblicherweise notiert $s \mapsto X(s)$ diesen Zusammenhang.
\end{lemma, definition}
\begin{proof}
	Wähle $o \in I$ und bemerke \[s'(t) = |X'(t)| > 0.\]
	Also ist $t \mapsto s(t)$ streng monoton wachsend, kann also invertiert werden, um $t= t(s)$ zu erhalten: Damit erhält man für 
	\[\widetilde{X}:=X\circ t\]
	\[|\widetilde{X}'(s)|= |X'(t(s))| \cdot |t'(s)| = \frac{s'(t)}{s'(t)}= 1,\]
	d.h. $\widetilde{X}$ ist nach Bogenlänge parametrisiert. (nämlich durch Division mit der Inversen.)
\end{proof}

\begin{remark}
	Eine Bogenlängenparametrisierung ist eindeutig bis auf Wahl von $o$ und Orientierung.
\end{remark}

\begin{example}
	Eine Helix \[t \mapsto X(t) = O + e_1r\cos(t)+e_2r\sin(t)+ e_3ht\]
	hat Bogenlänge \[s(t) = \int_{0}^{t} \sqrt{r^2 + h^2}dt = \sqrt{r^2+h^2} \cdot t\]
	und somit Bogenlängenparametrisierung \[s \mapsto \widetilde{X}(s)= O + e_1r\cos\frac{s}{\sqrt{r^2+h^2}}+e_2r\sin\frac{s}{\sqrt{r^2+h^2}}+ e_3\frac{hs}{\sqrt{r^2+h^2}}.\]
\end{example}

\begin{remark, example}
	
	Üblicherweise ist es nicht möglich eine Bogenlängenparam. in elem. Funktionen anzugeben: Eine Ellipse\[t \mapsto O + e_1a\cos(t)+e_2b\sin(t) ~ (a>b>0)\]
	hat Bogenlänge
	\[s(t) = \int_{0}^{t} \sqrt{b^2 + ( a^2-b^2)\sin(t)}dt,\]
	dies ist ein elliptisches Integral, also nicht mit elem. Funktionen invertierbar.

\end{remark, example}

\subsection{Streifen und Rahmen}

\begin{definition}
	
	Sei $X: \R \supseteq I \rightarrow \E$ eine parametrisierte Kurve.
	Die \emph{Tangente} an einem Punkt $X(t)$, wird durch den Punkt und seinen \emph{Tangentialvektor} $X'(t)$ beschrieben. $\mathcal{T}(t)=X(t) + [X'(t)]$ notiert diese Gerade.
	Die Ebene $\mathcal{N}(t)=X(t)+ \{X'(t)\}^\perp $ heißt \emph{Normalebene}.
	
	Alternativ können wir sagen: Wir erhalten Tangente, bzw. Normalebene, durch legen des \emph{Tangentialraumes} $[X'(t)]$ bzw. \emph{Normalraumes} $\{X'(t)\}^\perp$ durch den Punkt $X(t)$.
	
\end{definition}

\begin{definition}
	Das \emph{Tangential-} und \emph{Normalbündel} einer Kurve $X:I \rightarrow \E^3$ werden durch die folgenden Abbildungen definiert:
	\[I \ni t \mapsto T_tX := [X'(t)]\subseteq V \text{ bzw.}\]
	\[I \ni t \mapsto N_tX:= \{ X'(t) \}^\perp. \]
	
	eine Abbildung $Y: I \rightarrow V$ heißt
	\begin{enumerate}
		\item \emph{Tangentialfeld} entlang $X$, falls \[ \forall t \in I: Y(t) \in T_tX \]
		\item \emph{Normalenfeld} entlang $X$, falls \[  \forall t \in I: Y(t) \in N_tX \]
	\end{enumerate}
\end{definition}

\begin{remark, definition}

Jede Kurve hat ein \textbf{(und nur ein!)} harmonisches \emph{Einheitstangentenfeld} (ETF)
\[ T: I \rightarrow V, \quad t \mapsto \frac{X'(t)}{|X'(t)|} \]

Aber -- \textbf{es gibt haufenweise Normalenfelder.}

\end{remark, definition}

\begin{definition}
	Ein \emph{Streifen} (''ribbon'') ist ein Paar $(X,N)$, wobei $$X:I \rightarrow \E$$ eine Kurve und $$ N: I \rightarrow V $$ ein \emph{Einheitsnormalenfeld} (ENF) ist, d.h.,
	\[N\perp T \text{ und } |N|=1. \]
\end{definition}

\begin{remark, definition}
	(Im dreidimensionalen Raum können wir folgendes sagen:)
	Ein Streifen ist also eine Kurve mit einer ''vertikalen Richtung''.
	Weiters erhält man eine ''seitwärts Richtung'' durch die \emph{Binormale} $$B:=T\times N : I\rightarrow V.$$ (Hier ist $T \times N$ das ''bekannte'' Kreuzprodunkt)
	
\end{remark, definition}

\begin{lemma, definition}
	
	Der \emph{(angepasste) Rahmen} eines Streifens $(X,N): I\rightarrow\E^3\times S^2$ ist eine Abbildung \[ F=(T,N,B): I\rightarrow SO(V) \] seine \emph{Strukturgleichungen} sind von der Form \[ F' = F \phi \text{ mit } \phi =|X'| \begin{pmatrix}
	0 & - \kappa_n & \kappa_g \\
	\kappa_n & 0 & -\tau\\
	-\kappa_g & \tau & 0
	\end{pmatrix}, \] wobei\begin{enumerate}
		\item $\kappa_n$ die \emph{Normalkrümmung}
		\item $\kappa_g$ die \emph{geodätische Krümmung}, und
		\item $\tau$ die \emph{Torsion} des Streifens $(X,N)$ \\ bezeichnen.
		
	\end{enumerate}

\begin{proof}
	Da $F: I \rightarrow SO(V)$, gilt \[F^tF \equiv id\] und daher \[ 0 = (F^tF)' = F'^tF + F^tF' = (F\phi)^tF + F^tF\phi = \phi^tF^tF + F^tF\phi = \phi^t+ \phi  ~,\] d.h., $\phi:I \rightarrow o(V)$ ist schiefsymmetrisch. Insbesondere: Es gibt Funktionen $\kappa_n, \kappa_g, \tau$, so dass $\phi$ von der behaupteten Form ist.
\end{proof}

\textbf{Wiederholung:}
$$O(V) = \{ A \in End(V) ~ | ~ A^tA\equiv id \}$$
$$SO(V) = \{ A \in O(V) ~ | ~\det(A) = 1 \}$$
$$o(V) = \{ B \in End(V) ~ | ~ B^t +B \equiv 0 \}$$

\begin{remark}
	Krümmung und Torsion eines Streifens sind \emph{geometrische Invarianten} des Streifens, d.h., sie sind unabhängig von Position und (in gewisser Weise) Parametrisierung des Streifens.
	
	\begin{enumerate}
		\item ist $(\widetilde{X}, \widetilde{N}) = (\widetilde{O} + A(X-O), AN)$ mit $O,\widetilde{O} \in \E$ und $A \in SO(V)$ eine Euklidsche Bewegung des Streifens $(X,N)$, so sind $\widetilde{T}= AT$ und $\widetilde{B}= AT\times AN = A(T\times N) = AB$, also $\widetilde{F}=AF$ und damit $\widetilde{\phi} = \widetilde{F}^t\widetilde{F}'= F^tA^tAF' = \phi$. (Da $A \in SO(V))$
		
		\item ist $s \mapsto (\widetilde{X}, \widetilde{N})(s) = (X,N)(t(s))$ eine \emph{orientierungstreue Umparametrisierung}, d.h., $t' >0$, von $t \mapsto (X,N)(t)$, so gilt
		\[\widetilde{\phi}(s) = \widetilde{F}^t(s)\widetilde{F}'(s) = F^t(t(s))F'(t(s))\cdot t'(s) =\phi(t(s)) \cdot t'(s) \] und \[ |\widetilde{X}'(s)| = |X'(t(s))|\cdot|t'(s)| =  |X'(t(s))|\cdot t'(s) \] und damit $\widetilde{\kappa_n}(s)= \kappa_n(t(s))$ usw.
	\end{enumerate}
\end{remark}

\begin{lemma}
	Für einen Streifen $(X,N)$ gilt
	\begin{align*}
		\kappa_n 	= -\frac {\skal{N',T}}{\abs{X'}} = \frac {\skal{N,T'}}{\abs{X'}}, &&
		\kappa_g 	= -\frac {\skal{B,T'}}{\abs{X'}} = \frac {\skal{B',T}}{\abs{X'}}, &&
		\tau  		= -\frac {\skal{N,B'}}{\abs{X'}} = \frac {\skal{N',B}}{\abs{X'}} .
	\end{align*}
\end{lemma}
\begin{proof}
	Dies folgt sofort aus der Definition von $\kappa_n,\kappa_g$ und $\tau$ und aus der Orthonormalität von $T,N$ und $B$.
\end{proof}

\begin{remark, definition}
	Ist ein Streifen $(\widetilde{X},\widetilde{N})$ gegeben durch eine \emph{Normalrotation} eines Streifens $(X,N)$, d.h., $\widetilde{X}, \widetilde{N} = (X,N \cos \varphi + B \sin \varphi)$ mit $\varphi: I \rightarrow \R$, so gilt
	\begin{equation*}
	\begin{pmatrix} 
		\widetilde{\kappa_n}\\
		\widetilde{\kappa_g}
	\end{pmatrix}
	=
	 \begin{pmatrix} 
	 \cos \varphi & - \sin \varphi \\
	 \sin \varphi & \cos \varphi
	 \end{pmatrix}
	 \begin{pmatrix}
	 \kappa_n\\
	 \kappa_g
	 \end{pmatrix}
	\end{equation*} und
	$$\widetilde{\tau} = \tau + \frac{\varphi'}{|X'|}  .$$
	
	
\end{remark, definition}
	
\end{lemma, definition}

\begin{example}
	
	\begin{enumerate}
		 \item \textbf{Helix}: Betrachte den Streifen $(X,N)$ mit $$t \mapsto X(t) = o + e_1r\cos t + e_2r \sin t + e_3ht$$ und 
		$t \mapsto N(t) = -(e_1\cos t + e_2 \sin t)$.
		Für $$T(t) = (-e_1r\sin t + e_2r \cos t + e_3h) \frac{1}{\sqrt{r^2 + h^2}}$$ und
		$$B(t) = (e_1h\sin t - e_2h \cos t + e_3r) \frac{1}{\sqrt{r^2 + h^2}}$$ bekommt man
		$F = (T,N,B): \R \rightarrow SO(V)$ und damit \[ T' = N \cdot \frac{r}{\sqrt{r^2 + h^2}} \]		\[ N' = T \cdot \frac{-r}{\sqrt{r^2 + h^2}} + B \cdot \frac{h}{\sqrt{r^2 + h^2}} \]	
		\[B'= \frac{-h}{\sqrt{r^2 +h^2}}\]
		also (mit $|X'|=\sqrt{r^2 + h^2}$),
		\begin{align*}
		\kappa_n  = \frac{r}{r^2 + h^2}, &&
		\kappa_g  = 0, &&
		\tau = \frac{h}{r^2+ h^2}. 
		\end{align*} 
		
		\item \textbf{sphärische Kurve}: Sei $s\mapsto X(s) \in \E^3$ eine bogenlängenparametrisierte Kurve, d.h. mit Mittelpunkt $O \in \E^3$ und Radius $r>0$, der Sphäre gilt:
		\[ |X-O|^2 \equiv r^2 \text{ und } |X'|^2 \equiv 1  .\]
		Bemerke:
		$\langle X',X-O \rangle = \frac 12 (|X-O|^2)' = 0  .$
		Also liefert $N :=(X-O)\frac 1r $ ein ENF. Damit berechnen wir
		\begin{align*}
			\kappa_n &= - \langle N', T \rangle \equiv \frac 1r\\
			\kappa_g &= - \langle B, T' \rangle = - \frac 1r \langle X' \times (X-O), X'' \rangle = \frac{det(X-O,X',X'')}{r}\\
			\tau &= \langle N',B\rangle = \frac{1}{r^2} \langle X',X'\times (X-O) \rangle \equiv 0  .
			\end{align*}
		 
	\end{enumerate}

\end{example}

\begin{remark}
	$\kappa_g \equiv 0$ im ersten Bsp. und $\tau \equiv 0$ im zweiten Bsp.
\end{remark}

\begin{theorem}[Fundamentalsatz für Streifen]
	Seien $$\kappa_n, \kappa_g, \tau: I \rightarrow \R, \quad s \mapsto \kappa_n(s), \kappa_g(s), \tau(s)$$ gegeben. Dann gibt es eine bogenlängenparametrisierte Kurve $$X: I \rightarrow \E$$ und ein ENF $$N: I \rightarrow V  ,$$ so dass $\kappa_n, \kappa_g, \tau$ Normal- bzw. geodätische Krümmung und Torsion des Streifens $(X,N)$ sind.\\ Dieser Streifen ist bis auf Euklid. Bewegung eindeutig.
\end{theorem}

\begin{proof}
	
	Wähle $o \in I$ und $F_o \in SO(V)$ beliebig und fest.
	Nach Satz von Picard-Lindelöf hat das AWP 
		\[ F' = F\phi, ~ F(o)= F_o \] mit 
	\[ \phi =\begin{pmatrix}
		0 & -\kappa_n & \kappa_g \\
		\kappa_n & 0 & -\tau \\
		-\kappa_g & \tau & 0
	\end{pmatrix}\]
	eine eindeutige Lösung $F = (T,N,B) : I \rightarrow End(V)$.
	Nun zeigen wir, dass $F$ ein Rahmen ist:
	\begin{enumerate}
		\item $(FF^t)' = F( \phi  + \phi^t) F^t \equiv 0 $ also $FF^t \equiv id,$ und $F:I \rightarrow O(V)$
		\item $\det: O(V) \rightarrow \{\pm1\}$ ist stetig, also $\det F: I \rightarrow \{ \pm 1 \}$ konstant und somit $$\det F =\det F_o = 1  ,$$ also $F: I \rightarrow SO(V).$
	\end{enumerate}
	
	Insbesondere $|T| \equiv 1$ und man erhält eine bogenlängenparametrisierte Kurve
		\[X: I \rightarrow \E^3, t \mapsto O + \int_{o}^{t} T(s) ds  . \]
		
	$(X,N)$ mit $F=(T,N,B)$ liefert einen Streifen, Krümmung und Torsion wie behauptet.\\
	Eindeutigkeit bis auf Euklid. Bewegung folgt aus der Eindeutigkeit in Picard-Lindelöf und jener der Integration.
\end{proof}

\subsection{Normalzusammenhang \& Paralleltransport}

\begin{definition}

Für ein Normalenfeld kann man die Ableitung $N' = {\color{red}N' - \langle N',T  \rangle T} +  {\color{ForestGreen}\langle N',T  \rangle T} $ in {\color{red}Normal-} und {\color{ForestGreen}Tangentialanteil} zerlegen.

\end{definition}

\begin{definition}
	Ein Normalenfeld $N: I \rightarrow V $ entlang $X: I \rightarrow \E$ heißt \emph{parallel}, falls $\nabla^\perp N := N'-  \langle N',T\rangle T = 0,$ wobei $\nabla^\perp $ den \emph{Normalzusammenhang} entlang $X$ bezeichnet.
\end{definition}

\begin{remark}
	Hier wird \underline{nicht} $| N | = 1$ angenommen.
\end{remark}

\begin{lemma}
	Der Normalzshg. ist \emph{metrisch}, d.h., $$ \langle N_1,N_2 \rangle ' = \langle \nabla^\perp N_1,N_2 \rangle + \langle N_1, \nabla^\perp N_2 \rangle; $$
	
	parallele Normalenfelder haben konstante Länge und schließen konstante Winkel ein.
\end{lemma}

\begin{proof}
		
		Für Normalenfelder $N_1, N_2 : I \rightarrow V$ entlang $X: I\rightarrow \E$ gilt:
		\[ \langle \nabla^\perp N_1,N_2 \rangle + \langle N_1,\nabla^\perp N_2 \rangle = \langle N_1'- \langle N_1,T \rangle T,N_2 \rangle + \langle N_1,N_2' - \langle N_2',T \rangle T \rangle\]
		\[ =\langle N_1',N_2 \rangle + \langle N_1,N_2' \rangle = \langle N_1,N_2 \rangle ' .\]
		Insbesondere sind $N_1,N_2$ parallel, so ist $\langle N_1,N_2 \rangle ' = 0$
		
		Damit: 
		
		\begin{enumerate}
			\item  ist $N$ parallel, so gilt $$ (|N|^2)' = 2 \langle N,\nabla^\perp N \rangle = 0$$
			
			\item  sind $N_1, N_2$ parallel, so ist der Winkel $\alpha$ zwischen $N_1,N_2$
			$$ \alpha = \arccos \frac{\langle N_1,N_2 \rangle}{|N_1||N_2|}= const. $$
		\end{enumerate}
		
\end{proof}

\begin{example}
	Für einen Kreis $t \mapsto X(t) = O + (e_1 \cos t + e_2 \sin t)r$ ist $t\mapsto N(t) := e_1 \cos t + e_2 \sin t$ ein paralleles Normalenfeld.
\end{example}


\begin{remark}
	Ist $(X,N)$ ein Krümmungsstreifen, $\tau \equiv 0,$ so ist $N$ parallel. Aus $ N' = (-\kappa_n T + \tau B)\abs{X'} $ folgt
	\[  \nabla^\perp N  = (-\kappa_n T + \tau B)\abs{X'} +\kappa_nT = B\tau\abs{X'}= 0. \]
	Andererseits: Ist $N: I \rightarrow V$ parallel entlang $X: I \rightarrow \E$ , so ist $ (X,\frac{N}{|N|}) $ Krümmungsstreifen (falls $N \not = 0$).
\end{remark}

\begin{remark}
	Ist $N$ parallel längs $X$, so auch $B = T\times N.$
\end{remark}

\begin{remark}
	Ist $(X,N)$ durch eine Normalrotation von $(X, \widetilde{N})$ gegeben, d.h. \[ (X,N) = (X,\widetilde{N} \cos \varphi + \widetilde{B}\sin \varphi) \] mit $\varphi: I \rightarrow \R$, so gilt 
	\[ \tau = \widetilde{\tau} + \frac{\varphi '}{|X'|}; \]
	folglich: Man erhält einen Krümmungsstreifen bzw. ENF $N: I \rightarrow V$ einer Kurve $X: I \rightarrow \E$ durch \[ N = \widetilde{N} \cos \varphi + \widetilde{B} \sin \varphi \text{ mit } \varphi(t) = \varphi_o - \int_{o}^{t} \tau(t) ds. \]
	
	Wobei $\varphi_o$ eine Integrationskonstante ist und eine konstante Normaldrehung liefert und $ds$ für $|X'| dt$ -- das Bogenlängenelement -- steht.
	
	Da konstante Skalierungen eines parallelen Normalenfeldes parallel ist, folgt:
	
	
\end{remark}

\begin{lemma}
	Sei $X: I \rightarrow \E$ eine Kurve, $o \in I$ und $N_o \in N_oX$. Dann existiert ein eindeutiges paralleles Normalenfeld $N: I \rightarrow V$ mit $N(o) = N_o$
\end{lemma}

\begin{proof}
	der Beweis folgt aus der Bemerkung darüber. Allerdings nur für Dimension 3. Der Beweis gilt auch sonst, dann braucht man allerdings Picard-Lindelöf
\end{proof}

\begin{example}
	Für das ''radikale'' ENF $\widetilde{N} = - (e_1 \cos t + e_2 \sin t)$ der Helix
	$$ X = O + e_1 r \cos t + e_2 r \sin t + e_3 h t$$
	ist $\widetilde{\tau} = \frac{h}{r^2 + h^2}$. Also liefert \[ N(t) := \widetilde{N}(t) \cos (\frac{ht}{\sqrt{r^2 + h^2}}) + \widetilde{B}(t) \sin (-\frac{ht}{\sqrt{r^2 + h^2}}) \] ein paralleles Normalenfeld.

\end{example}

\begin{lemma, definition}
	Parallele Normalenfelder entlang $X: I \rightarrow \E$ definieren eine lineare Isometrie von $N_oX$ nach $N_tX$. Diese Isometrie heißt \emph{Paralleltransport} entlang $X$.
\end{lemma, definition}

\begin{remark}
	Dies erklärt den Begriff ''Zusammenhang'' für $\nabla^\perp$:$\nabla^\perp$ liefert einen Zusammenhang zwischen Normalräumen einer Kurve.
\end{remark}

\begin{proof}
	Wähle $N_o \in N_oX$; nach Lemma vorher gibt es ein eindeutiges(!) paralleles NF $N : I \rightarrow V$ entlang $X$ mit $N(o) = N_o$; also definiere durch 
	\[\pi_t: N_oX \rightarrow N_t X,~ N_o \mapsto N(t)  \] eine wohldefinierte Abbildung. Da die Gleichung $\nabla^\perp N = 0$ linear ist, sind konstante(!) Linearkombinationen von Lösungen wieder Lösungen -- also ist $\pi_t$ linear.
\end{proof}

\subsection{Frenet Kurven}

Wir diskutieren $\kappa_g \equiv 0$ (vorheriger Abschnitt $\tau \equiv 0$).

Bemerke: ist $(\widetilde{X},\widetilde{N}) =(X,N \cos \varphi + B \sin \varphi )$ Normalrotation eines Streifens $(X,N)$, so gilt 
\begin{equation*}
\begin{pmatrix} 
\widetilde{\kappa_n}\\
\widetilde{\kappa_g}
\end{pmatrix}
=
\begin{pmatrix} 
\cos \varphi & - \sin \varphi \\
\sin \varphi & \cos \varphi
\end{pmatrix}
\begin{pmatrix}
\kappa_n\\
\kappa_g
\end{pmatrix}
\end{equation*}

insbesondere $\widetilde{\kappa}_n = - \kappa_g $ und  $\widetilde{\kappa}_g = \kappa_n$
für $(\widetilde{X}, \widetilde{N}) = (X,B)$. % N und B werden vertauscht und damit vertauscht sich \kappa_g und \kappa_n

\begin{definition}
	$X: I \rightarrow \E^3$ heißt \emph{Frenet Kurve}, falls \[ \forall t \in I: (X' \times X'')(t) \not = 0. \]
\end{definition}

\begin{remark}
	In diesem Kapitel wird stets der 3-dimensionale Raum angenommen.
\end{remark}

\begin{remark}
	Die Frenet-Bedingung ist invariant unter Umparametrisierung.
\end{remark}

\begin{lemma, definition}
	Ist $X: I \rightarrow \E^3$ Frenet, so gilt \[ \forall t \in I : T'(t) \not = 0 \] und $\frac{T'}{|T'|} =: N$ definiert ein ENF: Dies ist die \emph{Hauptnormale} von $X$.
\end{lemma, definition}

\begin{proof}
	Mit der Frenet-Bedingung:
	\[0 \not = X' \times X''= X' \times (T|X'|)' = X' \times T|X'|' + X'\times T'|X'| \Rightarrow T' \not = 0 \]
	
	Weiters: \[ 0 = (1)' = (|T|^2)' = 2\langle T,T' \rangle, \]
	also definiert $N = \frac{T'}{|T'|}$ ein ENF.
\end{proof}

\begin{lemma, definition}
	Ist $X: I \rightarrow \E^3$ Frenet mit Hauptnormale $N: I \rightarrow V$, so sind die Strukturgleichungen des \emph{Frenet Rahmens} $F = (T,N,B)$ der Kurve die \emph{Frenet-Serret Gleichungen} $F' = F\phi$ mit $\phi = |X'|\begin{pmatrix}
	0 & - \kappa & 0\\
	\kappa & 0 & - \tau \\
	0 & \tau & 0
	\end{pmatrix}$
	 mit der \emph{Krümmung} $\kappa > 0$ und \emph{Torsion} $\tau$ der Frenet Kurve $X$.
\end{lemma, definition}

\begin{remark}
	Für eine Frenet Kurve (mit Hauptnormale) gilt also $\kappa_n = \kappa > 0$ und $\kappa_g = 0$.
\end{remark}

\begin{proof}
	Für einen Frenet Rahmen gilt 
	\[ \kappa_g = - \frac{\langle T', B \rangle}{|X'|} = - \frac{\langle N, B \rangle |T'|}{|X'|} = 0, \]
	
	\[ \kappa_n = \frac{\langle T', N \rangle}{|X'|} = \frac{\langle N, N \rangle |T'|}{|X'|} >0. \]
\end{proof}

\begin{example}
	Eine Helix
	
	\[ X = O + e_1 r \cos t + e_2 r \sin t + e_3 h \]
	
	hat Hauptnormalenfeld (s. Kapitel 1.2) \[ t \mapsto N(t) = - (e_1 \cos t + e_2 \sin t) \] und Krümmung und Torsion 
	\[ \kappa \equiv \frac{r}{r^2 +h^2} \text{ bzw. } \tau \equiv \frac{h}{r^2+h^2} . \]
\end{example}

\begin{remark}
	Krümmung und Torsion einer Frenet Kurve sind \[ \kappa = \frac{|X' \times X''|}{|X'|^3} \text{ bzw. }  \tau = \frac{\det(X',X'',X''')}{|X' \times X''|^2} \]
	
	Insbesondere: Krümmung und Torsion hängen nur von der Kurve ab (daher: ''Krümmung'' und ''Torsion der Kurve'').
\end{remark}

\begin{theorem}[Fundamentalsatz für Frenet Kurven]
	Für zwei Funktionen \[ s \mapsto \kappa(s), \tau(s) \text{ mit } \forall s \in I: \kappa(s) > 0 \] gibt es eine Bogenlängenparametrisierte Frenet Kurve $X: I \rightarrow \E ^3$ mit Krümmung $\kappa$ und Torsion $\tau$. 
	Weiters: $X$ ist eindeutig bis auf Euklid. Bewegung.
\end{theorem}

\begin{proof}
	Nach dem Fundamentalsatz für Streifen existiert bogenlängen-parametrisierte Kurve $X: I \rightarrow \E^3$ und ENF $N: I \rightarrow V$ so dass der Streifen $ (X,N)$ Krümmung und Torsion \[  \kappa_n = \kappa, \kappa_g = 0, \tau = \tau,  \] d.h.\[F' = F\phi \text{ mit } \phi = \begin{pmatrix}
	0 & - \kappa & 0\\
\kappa & 0 & - \tau \\
0 & \tau & 0
	\end{pmatrix} \] hat.
	
	Insbesondere $T' = N\kappa \not = 0$, daher:
	\begin{enumerate}
	\item $X$ ist Frenet, da
	\[ X' \times X'' = T \times T' = T \times N\kappa \not = 0 \]
	und \item  $N$ ist Hauptnormalenfeld, da \[ N = T'\frac{1}{\kappa} = \frac{T'}{|T'|}. \]
	
	\end{enumerate}
\end{proof}

\begin{remark}
	Einen einfacheren Fundamentalsatz gibt es für ebene Kurven. (Aufgabe: Formulieren und -- ohne Picard-Lindelöf-- beweisen!)
\end{remark}

\begin{example}
	Seien $\kappa > 0, \tau \in \R$ Zahlen. Nach dem Fundamentalsatz existiert (eind. bis auf Eukl. Bewegung) eine bogenlängenparametrisierte Frenet Kurve mit Krümmung $\kappa$ und Torsion $\tau$. Andererseits: \[ \R \ni s \mapsto X(s) = O + e_1 r \cos \frac{s}{\sqrt{r^2+h^2}} + e_2 r \sin \frac{s}{\sqrt{r^2+h^2}} + e_3 \frac{hs}{\sqrt{r^2 + h^2}}\] ist bogenlängenparam. Frenet Kurve mit Krümmung $\kappa$ und Torsion $\tau$ für \[ r= \frac{\kappa}{\kappa^2 + \tau^2} \text{ und } h = \frac{ \tau}{\kappa^2 + \tau^2}. \]
	
	Damit haben wir bewiesen:
\end{example}

\begin{theorem}[Klassifikation der Helices]
	Eine Frenet Kurve ist genau dann Helix, wenn sie konstante Krümmung und Torsion hat.
\end{theorem}

\begin{example}
	
	Falls $  (u,v) \mapsto X(u,v) $ Werte in einer (festen) Ebene annimmt,
	\[ \pi = \{ X \in \E^3 ~|~ \skal{X-0,n} = d \} \] d.h., $ \skal{dX,n}=0 $, so ist $ N \equiv \pm n $, demnach also $\S \equiv 0$ und jeder Punkt der Fläche ist Flachpunkt. 
	
\end{example}


%\pagebreak
%
%\section{Flächen}
%
\subsection{Parametrisierung \& Metrik}
\begin{definition}
	

Eine Abbildung \[ X:\R^2 \supseteq M \rightarrow \E \] heißt \emph{parametrisierte Fläche}, falls $M$ offen und zusammenhängend ist und $X$ regulär ist, d.h., $  \forall(u,v) \in M~:~d_{(u,v)}X: \R^2 \rightarrow V $ ist injektiv.
Wir sagen auch: $  X  $ ist eine \emph{Parametrisierung} der \emph{Fläche} $ X(M) \subseteq \E $.
Wobei $ d_{(u,v)} $ definiert ist über:
	\[d_{(u,v)}X(\pmat{x \\y})= X_u(u,v)\cdot x + X_v(u,v) \cdot y. \]

\end{definition}

\begin{remark}
	
 Äquivalent zur letzten Forderung ist die Forderung, dass die Jacobi-Matrix maximalen Rang hat. Diese braucht aber eine Festgelegte Basis, was oft zu Schwierigkeiten bei Berechnungen führt und wird daher von Prof. Jeromin nicht empfohlen.

\end{remark}

\begin{remark}
	Einmal mehr sind alle geforderten Abbildungen so oft differenzierbar, wie wir das wünschen.
\end{remark}

\begin{remark}
	
	$ d_{(u,v)} : \R^2 \rightarrow V$ ist die Ableitung am Punkt $ (u,v) \in M $, \[X(u+x,v+y) \approx X(u,v) + d_{(u,v)}  X(\begin{pmatrix}
	x\\
	y
	\end{pmatrix}) = X(u,v) + X_u(u,v)\cdot x + X_v(u,v)\cdot y,  \]
	wir können also identifizieren: \[ d_{(u,v)}X \cong (X_u,X_v)(u,v), \]
	bzw., nach Wahl einer Basis von V, mit der Jacobi-Matrix am Punkt $ (u,v) $.
	
\end{remark}

\begin{example}
	


Ein \emph{Helicoid} $ X: \R^2 \rightarrow \E^3 $ ist die \emph{(Regel-)Fläche}  
\[ \R^2 \ni (r,v) \mapsto O + e_1r \cos(v) + e_2 r \sin(v) + e_3 v \in \E^3. \]

Wir zeigen, dass $(X_r,X_v)(r,v)$ linear unabhängig für alle $ (r,v) \in \R^2 $ sind:

\[ X_r(r,v) = e_1\cos(v) + e_2 \sin(v) \neq 0 \]
\[ X_v(r,v) = -e_1r \sin(v) + e_2 r \cos(v) + e_3 \neq 0 \]

und da $ X_v(r,v) $ von $e_3 \neq 0$  abhängt sind die beiden linear unabhängig.

\end{example}

\begin{example}
	
	Eine übliche Parametrisierung von $ \mathbb{S}^2 \subseteq \E^3 $ (mit Mittelpunkt $O \in \E^3$) ist  
	\[ (u,v) \mapsto O + e_1\cos(u)\cos(v) + e_2\cos(u)\sin(v)+e_3\sin(u) \]
	liefert keine parametrisierte Fläche, da die Sphäre an den Polen nicht regulär ist.
	Dies ist also nur eine Parametrisierung auf $ M = (-\frac{\pi}{2},\frac{\pi}{2}) \times \R $
	
	Insbesondere kann man sogar zeigen, dass es keine (reguläre) Parametrisierung der (ganzen) Sphäre gibt (''Hairy Ball Theorem'' bzw. ''Satz vom Igel'').
	
	$ \mathbb{S}^2 $ ist also \textbf{keine} Fläche im Sinne der Definition. Dieses ''Problem'' wird später gelöst.
\end{example}

\begin{lemma, definition}
	
	Die \emph{induzierte Metrik} oder \emph{erste Fundamentalform} einer parametrisierten Fläche $ X: M \rightarrow \E$ ist definiert durch
	\[ \mathrm{I} := \langle dX,dX \rangle. \]
	Für jeden Punkt $ (u,v) \in M $ liefert 
	\[ \R^2 \times \R^2 \ni \left( \pmat{x_1\\y_1}, \pmat{x_2\\y_2} \right)  \mapsto \mathrm{I} \big|_{(u,v)}(\begin{pmatrix}
	x_1 \\
	y_1
	\end{pmatrix},\begin{pmatrix}
	x_2 \\
	y_2
	\end{pmatrix}) := \left\langle d_{(u,v)}X(\begin{pmatrix}
	x_1 \\
	y_1
	\end{pmatrix}),  d_{(u,v)}X(\begin{pmatrix}
	x_2 \\
	y_2
	\end{pmatrix})  \right\rangle \]
	eine positiv definite, symmetrische Bilinearform.
\end{lemma, definition}

\begin{proof}
	%Das darf ich wohl machen.. \#getrektluka
	Zu zeigen ist, dass $I\big| _{(u,v)}$ für jeden Punkt $(u,v)\in M$ eine positiv definite symmetrische Bilinearform ist. 
	
	Weil $ I\big| _{(u,v)} $ eine Komposition aus linearen Funktionen und einer Bilinearform ist, ist auch $I\big| _{(u,v)}$ linear. Die Symmetrie ist ebenfalls leicht ersichtlich. Fehlt noch die positive Definitheit.
	
	Sei $\pmat{x\\y} \neq 0$ beliebig, so gilt
		\[ I\big| _{(u,v)}(\pmat{x\\y},\pmat{x\\y})
			= \skal{d_{(u,v)}X(\pmat{x\\y}),d_{(u,v)}X(\pmat{x\\y})}> 0. \]
	Die letzte Ungleichung gilt, weil $d_{(u,v)}X$ injektiv und linear ist, daher bildet nur $0$ auf $0$ ab. Also ist $d_{(u,v)}X(\pmat{x\\y})\neq 0$. Der Rest folgt, weil $\skal{.,.}$ positiv definit ist.
\end{proof}

\begin{remark}
	
	$I$ wird oft mit Hilfe der \emph{Gramschen Matrix} notiert: \[ \mathrm{I} = \begin{pmatrix}
	E & F\\
	F & G
	\end{pmatrix} = E du^2 + 2Fdudv + Gdv^2 \] mit
	\[ E=|X_u|^2,\quad F = \langle X_u,X_v \rangle,\quad G = |X_v|^2.  \]
	


Dann gilt für $ (u,v) \in M$ : \[ I \big|_{(u,v)}(\begin{pmatrix}
x_1 \\
y_1
\end{pmatrix},\begin{pmatrix}
x_2 \\
y_2
\end{pmatrix}) := \skal{ d_{(u,v)}X(\begin{pmatrix}
x_1 \\
y_1
\end{pmatrix}),  d_{(u,v)}X(\begin{pmatrix}
x_2 \\
y_2
\end{pmatrix}) } \] \[= \skal{ X_u(u,v)x_1 + X_v(u,v)y_1, X_u(u,v)x_2 + X_v(u,v)y_2 } \]
\[ = E(u,v)x_1x_2 + F(u,v)(x_1y_2 + x_2y_1) + G(u,v)y_1y_2 \]
\[ = \begin{pmatrix}
x_1 \\
y_1
\end{pmatrix}^t \begin{pmatrix}
E & F \\ 
F & G
\end{pmatrix}\Bigg|_{(u,v)} \begin{pmatrix}
x_2\\
y_2
\end{pmatrix} \]

\end{remark}

\begin{example}
	
	\begin{enumerate}
		
		\item Ein \emph{Zylinder}
			\[ (u,v) \mapsto X(u,v) := O + e_1x(u) + e_2y(u) +e_3v \]
		hat induzierte Metrik \[ I = (x'^2 + y'^2)du^2 + dv^2. \]
		Insbesondere: Ist $ u \mapsto O+e_1x(u) + e_2y(u) $ bogenlängenparametrisiert, so ist $ X $ \emph{isometrisch},
		\[I = du^2 + dv^2\]
		
		\item Das \emph{Helicioid} \[ (r,v) \mapsto O + e_1r\cos(v)+e_2r\sin(v) + e_3v \]
		hat Metrik \[ I\big|_{(r,v)}=dr^2 + (1+r^2)dv^2. \]
		Mit einer Umparametrisierung $r = r(u) = \sinh(u)$ erhält man \[ I \big|_{(u,v)}= \cosh^2(u)(du^2+dv^2), \] d.h., $X$ wird \emph{konform} (winkeltreu).
		
	\end{enumerate}
	
\end{example}

\begin{definition}
	
	Eine Parametrisierte Fläche $ X : M \rightarrow \E $ heißt 
	\begin{enumerate}
		\item \emph{konform}, falls $ E = G, F = 0 $
		\item \emph{isometrisch}, falls $ E = G = 1, F= 0 $.
	\end{enumerate}
	
\end{definition}


\begin{remark}
	
	Für Kurven: Jede Kurve kann isometrisch/nach Bogenlänge umparametrisiert werden(\ref*{umpar}).  Für Flächen: Im Allgemeinen gibt es keine isometrische (Um-)Parametrisierung.
	
\end{remark}

\begin{theorem}
	
	Jede Fläche kann lokal konform (um-)parametrisiert werden. 
	
\end{theorem}

\begin{proof}
	
	Ist echt cool laut Jeromin (braucht bissi so Fana und so \ldots).
	Falls der Leser Zeit hat, wird ihm nahegelegt den Beweis nachzuschauen.
	Hier ein Link zu einem Beweis dieser Tatsache:

 	\href{https://thibaultlefeuvre.files.wordpress.com/2017/02/coord_isotherm.pdf}{Link zu PDF hier klicken.}
 	
	\end{proof}

\begin{remark}
	
	Dieser Satz ist die Grundlage, um (reelle) Flächen als komplexe Kurve zu intepretieren. Eine weitreichende Betrachtungsweise \ldots
\end{remark}

\begin{remark}
	
	Um den Satz zu verstehen:\\
	''lokal'' heißt, dass -- für jeden Punkt $ (u,v) \in M $ -- der Definitionsbereich $M$ so eingeschränkt werden kann -- auf eine Umgebung des Punktes$ (u,v) $ -- dass die Behauptung wahr ist;
	''Umparametrisierung'' wie für Kurven definiert:
\end{remark}

\begin{definition}
	
	Eine \emph{Umparametrisierung} einer parametrisierten Fläche $ X: M \rightarrow \E $ ist eine neue parametrisierte Fläche 
		\[ \widetilde{X}=X\circ(u,v), \quad \widetilde{M} \rightarrow \E, \]
	mit einem \emph{Diffeomorphismus}: 
		\[ (u,v): \widetilde{M} \rightarrow M, \] 
	d.h., eine glatte $ (C^\infty) $ Bijektion mit glatter Inverser $ (u,v)^{-1}:M \rightarrow \widetilde{M}. $
		
\end{definition}

\begin{remark}
	
	Für \[(x,y)\mapsto \widetilde{X}(x,y) = X(u(x,y),v(x,y)) \in \E^3  \] gilt (Kettenregel) \[ \widetilde{X}_x = (X_u\circ (u,v))\cdot u_x + (X_v \circ (u,v)) \cdot v_x \]
	\[ \widetilde{X}_y = (X_u \circ (u,v))\cdot u_y + (Y_v \circ (u,v)) \cdot v_y \]
	und somit 
		\[ \widetilde{X}_x \times \widetilde{X}_y = ((X_u\times X_v)\circ (u,v))\cdot (u_xv_y - u_yv_x), \]
	d.h., $ \widetilde{X} $ ist regulär.
	
\end{remark}

\subsection{Gaußabbildung und Weingartentensor}

\begin{definition}
	
	Eine Fläche $ X : M \rightarrow \E^3 $ hat an jedem Punkt $ X(u,v) $ eine \emph{Tangentialebene} und eine \emph{Normalgerade}:
		\[ \mathcal{T}(u,v) := X(u,v) + [ \{ X_u(u,v), X_v(u,v) \} ], \]
		\[ \mathcal{N}(u,v) := X(u,v) + [ \{ (X_u \times X_v)(u,v) \} ]; \]
	dies entspricht einer orthogonalen Zerlegung
		\[ V= [ \{ X_u(u,v),X_v(u,v) \} ] \oplus_\perp [ \{ (X_u \times X_v)(u,v) \} ] \]
	von $ V $ in einen \emph{Tangentialraum} und einen \emph{Normalraum} von $X$ am Punkt $X(u,v).$
	
\end{definition}
	
\begin{definition}
	
	Das \emph{Tangential-} bzw. \emph{Normalenbündel} einer Fläche $ X: M \rightarrow \E $ ist gegeben durch die Abbildung 
		\[ (u,v) \mapsto T_{(u,v)}X := [ \{X_u(u,v),X_v(u,v)\} ], \]
		\[ (u,v) \mapsto N_{(u,v)}X := \{ X_u(u,v),X_v(u,v) \}^{\perp}. \]
	Eine Abbildung $ Y : M \rightarrow V $ heißt 
	
	\begin{itemize}
		
		\item \emph{Tangentialfeld} entlang $ X $, falls 
			\[ \forall (u,v) \in M: ~ Y(u,v) \in T_{(u,v)}X \],
			
		\item \emph{Normalenfeld} entlang $ X $, falls
			\[ \forall(u,v) \in M: ~ Y(u,v) \in N_{(u,v)}X. \]
		
	\end{itemize}

	Die \emph{Gaußabbildung} einer Fläche $ X: M \rightarrow \E^3 $ ist das Einheitsnormalenfeld (ENF):
		\[ N:= \dfrac{X_u \times X_v}{\abs{X_u \times X_v}}: M \rightarrow V. \]
	
\end{definition}
	
\begin{example}
	\textbf{Roationsfläche:}
	Für jede Rotationsfläche
		\[ (u,v) \mapsto X(u,v) := O + e_1r(u)\cos(v) + e_2 r(u) \sin(v) + e_3h(u) \]
	ist jede Profilkurve $ v \equiv \mathrm{const} $ der orthogonale Schnitt der Fläche mit der Ebene $ x\sin(v) = y \cos(v) $ der Meridiankurve; die Gaußabbildung erhält man also durch $ \dfrac{\pi}{2} $ Drehung des ETFs (''Einheitstangentialfeldes'') in der Ebene der Kurve 
		\[ N(u,v) = \{ -(e_1\cos(v) + e_2\sin(v))h'(u)+e_3r'(u) \} \dfrac{1}{\sqrt{(r'^2 + h'^2)}(u)}. \]
	Überprüfung des Vorzeichens:
	\[ \det \pmat{r'\cos & -r\sin & -h'\cos \\ r'\sin & r\cos & -h'\sin \\ h' & 0 & r'} = h'^2r + r'^2r = r(r'^2 + h'^2) > 0. \]
	
\end{example}
	
\begin{remark}
	
	Die Gaußabbildung einer Fläche ist ein geometrisches Objekt, d.h.,
	nach einer Euklid. Bewegung $ \widetilde{X} = \widetilde{O} + A(X-O), $ liefert
		\[ \widetilde{N}=\dfrac{\widetilde{X}_u \times \widetilde{X}_v}{|\widetilde{X}_u \times \widetilde{X}_v|} = \dfrac{AX_u \times A X_v}{\abs{AX_u \times AX_v}} = \dfrac{A(X_u \times X_v)}{\abs{A(X_u \times X_v)}} = A \dfrac{X_u \times  X_v}{\abs{X_u \times X_v}} = AN. \]
	Das vorletzte Gleichheitszeichen gilt, weil für $A \in \mathrm{SO}(3)$ gilt $ \abs{A}=1 $. 
	Eine Spiegelung liefert $ \widetilde{N}~=~-AN, $ d.h., wechselt das Vorzeichen -- was auch eine (ordnungsumkehrende) Umparametrisierung tut, z.B.: $ (u,v) \mapsto(v,u) $.
	Demnach ist $ N $ ''geometrisch'' bis auf Vorzeichen.
	
\end{remark}

\begin{remark}
	
	Ordnungsprobleme tauchen in unserem Setting mit parametrisierten (!) Flächen nicht auf: Die Gaußabbildung einer parametrisierten Fläche ist wohldefiniert; eine nicht-orientierbare Fläche (e.g. Möbiusband) kann durch eine doppelt überlagerte Parametrisierung beschrieben werden.
	
\end{remark}


\textbf{Erinnerung:} Die Normalkrümmung $ \kappa_n $ eines Streifens $(X,N) $ ist definiert durch 
	\[ 0=N'^T + T\abs{X'} \kappa_n = N'^T + X'\kappa_n, \]
wobei $ t \mapsto N'^T(t) \in T_tX $ den Tangentialanteil von $ N' $ bezeichnet, i.e.
	\[ N'^T = N' - \nabla^\perp N = T\skal{T,N'}. \]

\begin{lemma, definition}
	
	Die Ableitung der Gaußabbildung ist tangentialwertig,
	\[ \forall (u,v) \in M : ~ d_{(u,v)}N : \R^2 \rightarrow T_{(u,v)}X. \]
	Damit können wir den \emph{Formoperator} von $ X $ am Punkt $ (u,v) \in M $ definieren:
	\[ \s\big|_{(u,v)}:= - d_{(u,v)}N \circ (d_{(u,v)}X)^{-1} \in \mathrm{End(T_{(u,v)}X)}. \]
	
\end{lemma, definition}

\begin{proof}
	
	\begin{enumerate}
		
		\item Die Ableitung von N ist tangentialwertig. Nämlich:
			\[ 1 \equiv \abs{N}^2 \Rightarrow 0 = 2 \skal{N,dN} \Rightarrow \forall (u,v) \in M \forall \pmat{x \\y} \in \R^2 : d_{(u,v)}N(\pmat{x \\y}) \in T_{(u,v)X}. \]
		
		\item  $\s$ ist wohldefiniert:
		Da für $ (u,v) \in M, d_{(u,v)}X: \R^2 \rightarrow V $ injektiv ist, liefert dies einen Isomorphismus
			\[ d_{(u,v)}X: \R^2 \rightarrow T_{(u,v)}X \subseteq V, \]
		der invertiert werden kann, um eine lineare Abbildung
		
			\[ (d_{(u,v)}X)^{-1}: T_{(u,v)}X \rightarrow \R^2 \]
		
		zu erhalten.
		
		\item $ \s\big|_{(u,v)} $ ist Endomorphismus:
		Als Verkettung linearer Abbildungen
			\[ T_{(u,v)}X \xrightarrow{(d_{(u,v)}X)^{-1}} \R^2 \xrightarrow{-d_{(u,v)}N} T_{(u,v)}X \]
		
	\end{enumerate}
	
\end{proof}

\begin{remark}
	Die Abbildung $\mapsto \s \big|_{(u,v)} \in End(T_{(u,v)}X)$ liefert ein Endomorphismenfeld $\s$, welches man auch \emph{Weingartentensorfeld} nennt.
\end{remark}

\begin{remark}
	Da $\left( X_u(u,v) , X_v(u,v) \right)$ eine Basis von $T_{(u,v)}X$ ist, kann $\s \big|_{(u,v)}$ durch die Werte auf der Basis bestimmt werden:
	\[ \s X_u = -dN \circ (dX)^{-1}(X_u)= 
	-dN \circ \pmat{1\\0} = -N_u \]
	und 
	\[ \s X_v = -N_v. \]
\end{remark}

\begin{lemma}
	$\s \big|_{(u,v)} \in End(T_{(u,v)}X)$ ist symmetrisch für jedes $(u,v) \in M$.
\end{lemma}

\begin{proof}
	Wir verifizieren Symmetrie auf der Basis $(X_u(u,v),X_v(u,v))$ von $T_{(u,v)}X$:
	
	Da $N \perp X_u,X_v$ erhalten wir 
	\[ 0=\skal{X_u,N}_v = \skal{X_{uv}, N} + \skal{X_u,N_v}
	= \skal{X_{vu},N}- \skal{X_u,\s X_v}. \]
	Ebenfalls
	\[ 0=\skal{X_v,N}_u 
	= \skal{X_{vu},N}- \skal{X_v, \s X_u}. \]
	Also 
	\[ \skal{X_u,\s X_v} = \skal{\s X_u,X_v}. \]
\end{proof}


\begin{remark}
	Wie bei Streifen kann der Formoperator analog zu $\kappa_n$ durch die Gleichung 
	\[ 0=dN + \s \circ dX = dN^T + S\circ dX \] beschrieben werden.
	Der Formoperator ''kodiert'' also die Krümmung einer Fläche.
\end{remark}

\begin{definition}
	Sei $\s$ der Formoperator der Fläche $X : M \to \E^3$, dann heißt
	\begin{itemize}
		\item $H=\frac 12 \mathrm{tr} \s $ die \emph{mittlere Krümmung} von $X$,
		\item $K=\det S$ die \emph{Gauß Krümmung} von $X$ und 
		\item die Eigenwerte $\kappa^\pm = H \pm \sqrt{H^2-K}$ und die \emph{Eigenrichtungen} von $\s$ sind die \emph{Hauptkrümmungen} bzw. \emph{Hauptrichtungen} von $X$. 
	\end{itemize}
\end{definition}

\begin{remark} Es gilt
	$H=\frac 12 (\kappa^+ +\kappa^-)$ -- daher auch der Name \emph{mittlere Krümmung}.
\end{remark}

\begin{example}
	Eine Rotationsfläche parametrisiert nach Bogenlänge ist
	\[ X(u,v)= O+e_1r(u)\cos v +e_2 r(u) \sin v + e_3h(v), \]
	mit $r'^2+h'^2=1$. Damit folgt $r'r''+h'h''=0$. Mit der Gaußabbildung
	\[ N(u,v)= -e_1h'(u)\cos v - e_2h'(u) \sin v + e_3v'(u) \]
	bekommt man
	\[ N_v + X_v\frac {h'}r = (e_1\sin v - e_2\cos v)\left( h' - r\frac {h'}r \right)=0, \]
	\[ N_u + X_u\left( r'h''-r''h' \right)\]
	\[= (e_1 \cos v +e_2\sin v ) \left( h''+r'(r'h''-r''h') \right) + e_3 (r''+h'(r'h''-r''h'))
	= 0. \]
	Also liefern $X_u$ und $X_v$ Krümmungsrichtungen zu Hauptrichtungen
	\[ \kappa^+ = r'h''-r''h' \quad \text{ und }\quad
	\kappa^- = \frac {h'}r. \]		
\end{example}


\begin{remark}
	Formoperatoren und Krümmungen sind geometrische Objekte:
	\begin{itemize}
		\item Ist $\tilde X = X \circ (u,v)$ eine Umparametrisierung und $\tilde N = N \circ (u,v)$ so ist 
		\[ \tilde{\s} = -d\tilde N \circ (d\tilde X)^{-1} = -\left( d_{(u,v)}N \circ d(u,v) \right) \circ \left( d_{(u,v)X} \circ d(u,v)  \right)^{-1} \]
		\[ = -d_{(u,v)}N \circ d(u,v) \circ d(u,v)^{-1} \circ \left( d_{(u,v)}X \right)^{-1}
		= \s\big |_{(u,v)}, \]
		also insbesondere
		\[ \tilde H = H \circ (u,v), \quad \text{ und }\quad
		\tilde K = K \circ (u,v), \quad \text{ etc.} \]
		
		\item Ist $\tilde X=\tilde O+A(X-O)$ mit $A \in SO(3)$ so bekommt man 
		\[ \tilde{\s}= -d\tilde N \circ \left( d\tilde X \right)^{-1}
		= -A\circ dN \circ (dX)^{-1} \circ A^{-1}
		= A\circ \s \circ A^{-1}, \]
		insbesondere also
		\[ \tilde H = H, \quad \text{ und }\quad
		\tilde K = K, \quad \text{ etc.} \]
		Die Krümmungsrichtungen werden mit der Fläche gedreht:
		\[ \ker (\mathrm{id} \kappa^\pm -\tilde{\s}) = A\ker (\mathrm{id} \kappa^\pm-\s). \]
	\end{itemize}
\end{remark}	
	
\begin{definition}
	
	Ein Punkt $X(u,v)$ einer Fläche heißt 
	\begin{itemize}
		
		\item \emph{Nabelpunkt} (umbilic), falls $\kappa^+(u,v) = \kappa^-(u,v) \quad (\Longleftrightarrow (H^2- K)(u,v)=0), $
		\item \emph{Flachpunkt} (flatpoint), falls $\kappa^+(u,v) = \kappa^-(u,v)=0.$
		
	\end{itemize}

	
\end{definition}

\begin{remark}
	
	Ein Punkt $X(u,v)$ ist Nabelpunkt bzw. Flachpunkt, falls
	\[ \mathcal{S}\big|_{(u,v)} = H(u,v)\mathrm{id}_{T_{(u,v)}X} \text{ bzw. } \mathcal{S}\big|{(u,v)}=0. \]
	
\end{remark}

\begin{example}
	
	Falls $  (u,v) \mapsto X(u,v) $ Werte in einer (festen) Ebene
		\[ \pi = \{ X \in \E^3 ~|~ \skal{X-0,n} = d \} \] 
	annimmt, d.h., $ \skal{dX,n}=0 $. Also ist $ N \equiv \pm n $, weshalb $\s \equiv 0$ und jeder Punkt der Fläche ist ein Flachpunkt.\\
	Umgekehrt: Ist jeder Punkt einer Fläche $ X $ ein Flachpunkt, $ \s \equiv 0 $, so folgt $ N \equiv \mathrm{const.} $, also nimmt $ X $ Werte in einer (festen) Ebene an: Mit $ O \in \E^3 $ beliebig \[ 0 = \skal{dX,N} = d\skal{X-O,N} - \skal{X-O,dN} \Rightarrow \mathrm{const.} \equiv \skal{X-O,N}, \] wobei $ \skal{dX,N} $ eine Abbildung $ ((u,v), \pmat{x \\ y}) \mapsto \skal{d_{(u,v)}X\pmat{x\\y},N(u,v)} $ ist.

	\textbf{Matrixdarstellung}: Mit $(X_u,X_v)$ als tangentiales Basisfeld kann der Weingartentensor als Matrix geschrieben werden:
	$ (\s X_u, \s X_v) = (-N_u, -N_v) = (X_u,X_v) \pmat{s_{11} & s_{12} \\ s_{21} & s_{22}} $
	
	also, vermittels der Skalarprodukte der Basisvektoren $ X_u,X_v $, 
	\[\pmat{-\skal{X_u,N_u} & - \skal{X_u,N_v} \\ - \skal{X_v,N_u} & - \skal{X_v,N_v} } = \pmat{E &F \\ F &G} \pmat{s_{11} & s_{12} \\ s_{21} & s_{22}}, \] d.h., als die Gram Matrix der zweiten Fundamentalform:
	
	
\end{example}

\begin{lemma, definition}
	 Die \emph{zweite Fundamentalform} einer Fläche $ X: M \rightarrow \E^3 $ ist definiert als \[ \mathrm{II} := \skal{dX,dN}, \] wobei $ \mathrm{II}: ((u,v), \pmat{x \\ y},\pmat{\widetilde{x} \\ \widetilde{y}}) \mapsto \skal{d_{(u,v)}X \pmat{x\\y},d_{(u,v)}N \pmat{\widetilde{x}\\\widetilde{y}}}; $
	 an jedem Punkt $ (u,v) $ erhält man so eine symmetrische Bilinearform 
	 
	 \[ \mathrm{II}\big|_{(u,v)}: \R^2 \times \R^2 \rightarrow \R, \quad (\pmat{x_1 \\ y_1},\pmat{x_2 \\ y_2}) \mapsto - \skal{d_{(u,v)}X\pmat{x_1 \\ y_1},d_{(u,v)}N\pmat{x_2 \\ y_2}}. \]
	 
\end{lemma, definition}

\begin{proof}
	Dass $ \mathrm{II}\big|_{(u,v)} $ Bilinearform ist, ist klar. Weiters gilt mit der Leibniz-Regel: 
		\[ {\color{ForestGreen} \mathrm{II}(\pmat{1 \\ 0},\pmat{0 \\ 1}) = }-\skal{X_u,N_v}= \skal{X_{uv},N}= \skal{X_{vu},N} {=} -\skal{X_v,N_u} {\color{ForestGreen} = \mathrm{II}(\pmat{0 \\ 1},\pmat{1 \\ 0}) }, \] 
	also ist $ \mathrm{II} $ symmetrisch an jedem Punkt. 
\end{proof}

\begin{remark}
	
	Ist die erste Fundamentalform gegeben, so kann die zweite Fundamentalform aus dem Weingartentensor berechnet werden - und umgekehrt.
	
	\textbf{Warnung:} Obwohl der Weingartentensor symmetrisch ist, ist seine Darstellungsmatrix (bzgl. $ (X_u,X_v) $) normalerweise \textbf{nicht} symmetrisch:
	
	\[ \pmat{s_{11} & s_{12} \\ s_{21} & s_{22}} = \frac{1}{EG-F^2} \pmat{G & -F \\ -F & E} \pmat{e & f \\ f & g} = \frac{1}{EG-F^2} \pmat{Ge-Ff & Gf-Fg \\ Ef-Fe & Eg-Ff} \]
	
\end{remark}

\begin{remark}
	Die Gauß-Krümmung ist gegeben durch: \[ K = \frac{eg-f^2}{EG-F^2} \]
\end{remark}

\subsection{Kovariante Ableitung und Krümmungstensor}

\begin{definition}
	
	Die \emph{kovariante Ableitung} eines Tangentialfeldes $ Y:M\rightarrow V $ entlang $ X:M\rightarrow \E $ ist der Tangentialteil seiner Ableitung 
		\[ \nabla Y := (dY)^T = dY - \skal{dY,N}N, \] wobei $ \nabla $ den \emph{Levi-Civita Zusammenhang} entlang $ X $ bezeichnet.
	
\end{definition}

\begin{remark}
	
	Wegen $ \skal{X_{uu},N}= -\skal{X_u,N_u}= e $ usw. bekommt man 
		\[ \nabla_{\dfrac{\partial}{\partial u}}X_u = X_{uu}-Ne \text{ und } \nabla_{\dfrac{\partial}{\partial v}} = X_{uv}-Nf \]
		\[ \nabla_{\dfrac{\partial}{\partial v}}X_v = X_{vv}-Ng \text{ und } \nabla_{\dfrac{\partial}{\partial u}} = X_{uv}-Nf \]
	
	Wobei $$ \nabla_{\dfrac{\partial}{\partial u}}Y := Y_u - \skal{Y_u,N}N $$
	
\end{remark}

\begin{lemma}
	
	Der L-C Zusammenhang erfüllt die Leibniz-Regel,
		\[ \nabla(Yx) = (\nabla Y)x + Ydx \text{ für } x \in C^\infty(M) \] 
	und ist \emph{metrisch}, d.h., 
		\[ d\skal{Y,Z} = \skal{\nabla Y,Z} + \skal{Y,\nabla Z} \]
	
\end{lemma}

\begin{proof}
	
	Die Leibniz-Regel gilt, da für $x \in C^\infty(M)$:
		\[ \nabla(Yx) = dYx + Ydx- \skal{dYx,N}N - \skal{Ydx,N}N = (\nabla Y)x + Ydx, \]
	weil $ Y $ ein Tangentialfeld ist, ist $ \skal{Ydx,N} = 0. $ 
	
	Er ist metrisch, da:
		\[  \skal{\nabla Y,Z} + \skal{Y,\nabla Z} = \skal{dY,Z} -\skal{dY,N}\skal{N,Z} + \skal{Y,dZ} - \skal{Y,N}\skal{dZ,N} \]
		\[ = \skal{dY,Z} + \skal{Y,dZ} = d\skal{Y,Z}. \]
	
\end{proof}

\textbf{Matrixdarstellung:} Da die kovariante Ableitung eines TVFs $ Y $ tangential ist, erhält man für $ Y = X_v :$

\begin{equation} \label{Eq:Gamma}
\begin{split}
		 \left(\nabla_{\dfrac{\partial}{\partial u}}X_u,\nabla_{\dfrac{\partial}{\partial u}}X_v \right) &= (X_u,X_v)\Gamma_1  \\
		 \left(\nabla_{\dfrac{\partial}{\partial v}}X_u,\nabla_{\dfrac{\partial}{\partial v}}X_v \right) &= (X_u,X_v)\Gamma_2 
\end{split}
\end{equation} 
	mit 
		\[ \Gamma_i = \pmat{\Gamma_{i1}^1 & \Gamma_{i2}^1 \\ \Gamma_{i1}^2 & \Gamma_{i2}^2}. \]
		
	Also gilt für eine allgemeine TVF $ Y=X_ux+X_vy=dX\pmat{x \\ y} $, 
		\[ \nabla_{\dfrac{\partial}{\partial u}}Y= \nabla_{\dfrac{\partial}{\partial u}}\left ( \left( X_u,X_v \right) \pmat{x \\y}    \right ) = \left( \nabla_{\dfrac{\partial}{\partial u}}X_u,\nabla_{\dfrac{\partial}{\partial u}}X_v \right) \pmat{x \\y} + X_u,X_v \pmat{x_u \\ y_u} \]

		\[ = \left( X_u, X_v \right) \Gamma_1\pmat{x \\ y} + \left( X_u, X_v \right) \pmat{x_u \\ y_u} \] 
		
		\begin{equation} \label{Eq:Gammaallg}
		\begin{split}
			&= \left( X_u, X_v \right) \left( \dfrac{\partial}{\partial u} +\Gamma_1 \right) \pmat{x \\ y}  \quad \text{ und ebenso }
		 \\ 
		 \nabla_{\dfrac{\partial}{\partial v}}Y &= \left( X_u, X_v \right) \left( \dfrac{\partial}{\partial v} +\Gamma_2 \right) \pmat{x \\ y}, 
		 \end{split}
		\end{equation} 
	oder, anders gesagt:
		\[ \nabla_{\dfrac{\partial}{\partial u}} \circ dX = dX \circ \left( \dfrac{\partial}{\partial u} + \Gamma_1 \right)  \]
	und genauso
		\[  \nabla_{\dfrac{\partial}{\partial v}} \circ dX = dX \circ \left( \dfrac{\partial}{\partial v} + \Gamma_2 \right).   \]	
	
	Man bemerke: mit $ \pmat{x \\ y} = \pmat{1 \\ 0} \cong \dfrac{\partial}{\partial u} $ erhält man
		\[ \nabla_{\dfrac{\partial}{\partial u}}\left( \left( X_u,X_v \right) \pmat{1 \\ 0} \right) = \nabla_{\dfrac{\partial}{\partial u}}X_u = X_u \Gamma_{11}^1 + X_v\Gamma_{11}^2 = \left( X_u,X_v \right) \Gamma_1\pmat{1 \\ 0}   \] 
	und
		\[ \nabla_{\dfrac{\partial}{\partial v}} \left( \left( X_u,X_v \right) \pmat{1 \\ 0} \right) = \nabla_{\dfrac{\partial}{\partial v}}X_u = \left( X_u,X_v \right) \Gamma_2\pmat{1 \\ 0} \]
	und ebenso für  $ \pmat{x \\ y} = \pmat{0 \\ 1} \cong \dfrac{\partial}{\partial v} $, also ist \ref*{Eq:Gamma} ein Spezialfall von \ref*{Eq:Gammaallg}
	
	\begin{remark}
		\textbf{WARUNUNG:} $ \nabla_{\dfrac{\partial}{\partial u}} $ und $ \nabla_{\dfrac{\partial}{\partial v}} $ sind \emph{Differentialoperatoren} (keine Endomorphismen) - trotzdem benutzen wir Matrizen, um sie zu beschreiben.
	\end{remark}

\begin{lemma, definition}
	
	$\Gamma_{ij}^k$ heißen die \emph{Christoffel-Symbole} von $ X $; sie sind symmetrisch: 
		\[ \Gamma_{ij}^k = \Gamma_{ji}^k \]
	
\end{lemma, definition}

\begin{proof}
	
		\[ X_u\Gamma_{12}^1 + X_v \Gamma_{12}^2 = \nabla_{\dfrac{\partial}{\partial u}}X_v = \left( X_{vu} \right)^T = \left( X_{uv} \right)^T = \nabla_{\dfrac{\partial}{\partial v}}X_u = X_u\Gamma_{21}^1 + X_v\Gamma_{21}^2   \]
	also $ \Gamma_{12}^1 = \Gamma_{21}^1 $ und $ \Gamma_{12}^2 = \Gamma_{21}^2 $.
\end{proof}

\begin{theorem}[Koszul's Formeln]
	
	Mit Matrizen 
		\[ I = \pmat{E &f \\ F &G} \text{ und } J = \pmat{0 &-1 \\ 1 & 0 } \]
	gilt: 
		\[ \frac12 I_u - \frac{E_v - F_u}{2}J = I\Gamma_1, \]
		\[ \frac12 I_v + \frac{G_u - F_v}{2}J = I\Gamma_2. \]
		
\end{theorem}

\begin{proof}
	
	Multipliziere $ \nabla_{\dfrac{\partial}{\partial u}}X_u = X_u\Gamma_{11}^1 + X_v\Gamma_{11}^2 $ mit $ Y_u $ und $ X_v $, um zu erhalten:
		\[ E \Gamma_{11}^1 + F \Gamma_{11}^2 = \skal{X_u, \nabla_{\dfrac{\partial}{\partial u}}X_u} = \frac12 \skal{X_u,X_u}_u = \frac12 E \]
	
		\[ F\Gamma_{11}^1 + G \Gamma_{11}^2 = \skal{X_v, \nabla_{\dfrac{\partial}{\partial u}}X_u} = \skal{X_v,X_u}_u - \skal{\nabla_{\dfrac{\partial}{\partial u}}X_v,X_u} = \skal{X_v,X_u}_u - \skal{\nabla_{\dfrac{\partial}{\partial v}}X_u,X_u} \]
		\[ = F_u - \frac12 E_v. \]
	
	Insgesamt also
		\[ \pmat{E &F \\F &G} \pmat{\Gamma_{11}^1 & \Gamma_{12}^1 \\ \Gamma_{11}^2 & \Gamma_{12}^2} = \frac12 \pmat{E_u \\ F_u} - \dfrac{E_v - F_u}{2} \pmat{0 \\ 1} \]
		\[ = \left( \frac12 \pmat{E_u & F_u \\ F_u & G_u} - \dfrac{E_v - F_u}{2} \pmat{0 &-1 \\ 1 & 0 } \right).  \]
	
	Die anderen Teile der Gleichung  folgen ebenso.
	
\end{proof}

\begin{corollary}
	Die Christoffel - Symbole $ \Gamma_{ij}^k $ hängen nur von der induzierten Metrik ab.
\end{corollary}

\begin{example}
	
	Für eine isometrische Parametrisierung, $  E = G = 1 $ und $  F = 0 $, erhält man mit den Konszul-Formeln $ \Gamma_{ij}^k = 0 $.
	
\end{example}


\begin{definition}
	
	Für ein TVf $ Y : M \rightarrow V $ entlang $ X: M \rightarrow \E $ definieren wir den \emph{(Riemannschen) Krümmungstensor} $ R $ von $X$ durch 
		\[ RY := \nabla_{\dfrac{\partial}{\partial u}}\nabla_{\dfrac{\partial}{\partial v}}Y - \nabla_{\dfrac{\partial}{\partial v}}\nabla_{\dfrac{\partial}{\partial u}}Y. \]
	
\end{definition}

\begin{remark}
	
	Dies ist eine vereinfachte Form, für Flächen, des ''wahren'' Krümmungstensors.
	
\end{remark}

\begin{lemma}
	
	$ R $ ist ein schiefsymmetrischer Tensor des Tangentialbündels $ TX $, d.h., 
	$R\big|_{(u,v)} \in \mathrm{End}(T_{(u,v)}X) $ ist schiefsymmetrisch für jedes $ (u,v) \in M $, und 
		\[ R(Yx) = (RY)x \text{ für } x \in C^\infty(M). \]
	
\end{lemma}

\begin{proof}
	
	$R\big|_{(u,v)} \in \mathrm{End(T_{(u,v)}X)}$ ist klar.
	Schiefsymmetrie:
		\[\frac12 (\abs{Y}^2)_{uv} = \skal{Y \nabla_{\dfrac{\partial}{\partial u}}Y}_v = \skal{Y, \nabla_{\dfrac{\partial}{\partial v}}\nabla_{\dfrac{\partial}{\partial u}}Y} + \skal{\nabla_{\dfrac{\partial}{\partial v}}Y,\nabla_{\dfrac{\partial}{\partial u}}Y} \]
		\[ = \skal{Y, \nabla_{\dfrac{\partial}{\partial u}}\nabla_{\dfrac{\partial}{\partial v}}Y} + \skal{\nabla_{\dfrac{\partial}{\partial u}}Y, \nabla_{\dfrac{\partial}{\partial v}}Y} = \frac12 \left( \abs{Y}^2\right)  \]
		\[ \Rightarrow \skal{Y,RY} = \frac12 \left(\abs{Y}^2\right)_{vu}- \frac12\left(\abs{Y}^2\right)_{uv}=0 \]
	Tensoreigenschaft:
		\[ \nabla_{\dfrac{\partial}{\partial u}}\nabla_{\dfrac{\partial}{\partial v}}(Yx)= \nabla_{\dfrac{\partial}{\partial u}}\left(\left( \nabla_{\dfrac{\partial}{\partial v}}Y \right)x + Yx_v\right) \]
		\[ = \left( \nabla_{\dfrac{\partial}{\partial u}} \nabla_{\dfrac{\partial}{\partial v}} Y \right)x + \underbrace{\left( \nabla_{\dfrac{\partial}{\partial v}}Y \right)x_u + \left( \nabla_{\dfrac{\partial}{\partial u}}Y \right)x_v + Yx_{uv}}_{\text{symmetrisch in } \dfrac{\partial}{\partial u} \text{ und } \dfrac{\partial}{\partial v}} \]
	also
		\[ R(Yx)=(RY)x. \]
	
\end{proof}

%\pagebreak

\section{Kurven}
\subsection{Parametrisierung und Bogenlänge}

Wiederholung: Ein Euklidischer Raum $\mathcal{E}$ ist:
\begin{enumerate}
	\item Ein affiner Raum $(\mathcal{E},V,\tau)$ 
	\item über einem Euklid. Vektorraum $(V,<,>)$ .
\end{enumerate}

	Dabei: $\tau: V\times \mathcal{E} \rightarrow \mathcal{E}, \quad  (v,X) \mapsto \tau_v(X)=:X+v$ genügt
	\begin{enumerate}
		\item $\tau_0 = id_{\mathcal{E}}$ und $\forall v,w \in V ~ \tau_v \circ \tau_w = \tau_{v+w}$
		\item $\forall X,Y \in \mathcal{E} \exists! v \in V ~ \tau_v(x) = Y$ ((d.h. $\tau  \emph{ ist einfach transitiv}$)).
	\end{enumerate}
	
	
\begin{definition}
	Eine \emph{(parametrisierte-) Kurve} ist eine Abbildung \[X: I \rightarrow \mathcal{E}\] auf einem offenen Intervall $I \subseteq \R$, die regulär ist (d.h. $\forall t \in I ~ X'(t) \not = 0$).
	Wir nennen $X$ auch Parametrisierung der Kurve $\mathcal{C} = X(I)$.
\end{definition}

\begin{remark}
	Alle Abbildungen in dieser VO sind beliebig oft differenzierbar (d.h. $C^{\infty}$).
\end{remark}

\begin{example}
	Eine \emph{(Kreis-) Helix} mit Radius $r>0$ und Ganghöhe $h$ ist die Kurve
	\[X: \R \rightarrow \mathcal{E}^3, \quad t \mapsto X(t) := O + e_1r\cos(t) + e_2rsin(t) + e_3ht. \]
\end{example}

\begin{definition}
	\emph{Umparametrisierung} einer param. Kurve $X: I \rightarrow \mathcal{E}$ ist eine param. Kurve
	\[\widetilde{X}: \widetilde{I} \rightarrow \mathcal{E}, \quad s \mapsto \widetilde{X}(s)=X(t(s)),\]
	wobei $t: \widetilde{I} \rightarrow I$ eine surjektive, reguläre Abbildung ist.

\end{definition}

Motivation: Für eine Kurve $t \mapsto X(t)$ 
\begin{enumerate}
	\item X'(t) ist \emph{Geschwindigkeit(-svektor)} (''velocity''),
	\item |X'(t)| ist (skalare) Geschwindigkeit (''speed'').
\end{enumerate}

Rekonstruktion durch Integration:
\[X(t)= X(o) + \int_{o}^{t}X'(t)dt\]

und die Länge des Weges von $X(0)$ nach $X(t)$:
\[s(t) = \int_{o}^{t}|X'(t)|dt\]

\begin{definition}
	
	Die \emph{Bogenlänge} einer Kurve $X: I \rightarrow \mathcal{E}$ ab $X(o)$ für $o \in I$, ist
	\[s(t) := \int_{o}^{t}|X'(t)|dt\] (wobei $\int_{o}^{s}|X'(t)|dt$ auch als $\int_{o}^{t} ds$ geschrieben wird)
	
\end{definition}

\begin{remark}
	Dies ist tatsächlich die Länge des Kurvenbogens zwischen $X(o)$ und $X(t)$, wie man z.B. durch polygonale Approximation beweist (s. Ana2 VO)
	Also: Die Bogenlänge zwischen zwei Punkten ist \emph{invariant} ("ändert sich nicht")
	unter Umparametrisierung.
\end{remark}

\begin{lemma, definition}\label{umpar}
	Jede Kurve $t \mapsto X(t)$ kann man nach Bogenlänge (um-) parametrisieren, d.h. so, dass sie konstante Geschwindigkeit $1$ ($|X'(t)|\equiv 1$) hat.
	Dies ist die \emph{Bogenlängenparametrisierung} und üblicherweise notiert $s \mapsto X(s)$ diesen Zusammenhang.
\end{lemma, definition}
\begin{proof}
	Wähle $o \in I$ und bemerke \[s'(t) = |X'(t)| > 0.\]
	Also ist $t \mapsto s(t)$ streng monoton wachsend, kann also invertiert werden, um $t= t(s)$ zu erhalten: Damit erhält man für 
	\[\widetilde{X}:=X\circ t\]
	\[|\widetilde{X}'(s)|= |X'(t(s))| \cdot |t'(s)| = \frac{s'(t)}{s'(t)}= 1,\]
	d.h. $\widetilde{X}$ ist nach Bogenlänge parametrisiert. (nämlich durch Division mit der Inversen.)
\end{proof}

\begin{remark}
	Eine Bogenlängenparametrisierung ist eindeutig bis auf Wahl von $o$ und Orientierung.
\end{remark}

\begin{example}
	Eine Helix \[t \mapsto X(t) = O + e_1r\cos(t)+e_2r\sin(t)+ e_3ht\]
	hat Bogenlänge \[s(t) = \int_{0}^{t} \sqrt{r^2 + h^2}dt = \sqrt{r^2+h^2} \cdot t\]
	und somit Bogenlängenparametrisierung \[s \mapsto \widetilde{X}(s)= O + e_1r\cos\frac{s}{\sqrt{r^2+h^2}}+e_2r\sin\frac{s}{\sqrt{r^2+h^2}}+ e_3\frac{hs}{\sqrt{r^2+h^2}}.\]
\end{example}

\begin{remark, example}
	
	Üblicherweise ist es nicht möglich eine Bogenlängenparam. in elem. Funktionen anzugeben: Eine Ellipse\[t \mapsto O + e_1a\cos(t)+e_2b\sin(t) ~ (a>b>0)\]
	hat Bogenlänge
	\[s(t) = \int_{0}^{t} \sqrt{b^2 + ( a^2-b^2)\sin(t)}dt,\]
	dies ist ein elliptisches Integral, also nicht mit elem. Funktionen invertierbar.

\end{remark, example}

\subsection{Streifen und Rahmen}

\begin{definition}
	
	Sei $X: \R \supseteq I \rightarrow \E$ eine parametrisierte Kurve.
	Die \emph{Tangente} an einem Punkt $X(t)$, wird durch den Punkt und seinen \emph{Tangentialvektor} $X'(t)$ beschrieben. $\mathcal{T}(t)=X(t) + [X'(t)]$ notiert diese Gerade.
	Die Ebene $\mathcal{N}(t)=X(t)+ \{X'(t)\}^\perp $ heißt \emph{Normalebene}.
	
	Alternativ können wir sagen: Wir erhalten Tangente, bzw. Normalebene, durch legen des \emph{Tangentialraumes} $[X'(t)]$ bzw. \emph{Normalraumes} $\{X'(t)\}^\perp$ durch den Punkt $X(t)$.
	
\end{definition}

\begin{definition}
	Das \emph{Tangential-} und \emph{Normalbündel} einer Kurve $X:I \rightarrow \E^3$ werden durch die folgenden Abbildungen definiert:
	\[I \ni t \mapsto T_tX := [X'(t)]\subseteq V \text{ bzw.}\]
	\[I \ni t \mapsto N_tX:= \{ X'(t) \}^\perp. \]
	
	eine Abbildung $Y: I \rightarrow V$ heißt
	\begin{enumerate}
		\item \emph{Tangentialfeld} entlang $X$, falls \[ \forall t \in I: Y(t) \in T_tX \]
		\item \emph{Normalenfeld} entlang $X$, falls \[  \forall t \in I: Y(t) \in N_tX \]
	\end{enumerate}
\end{definition}

\begin{remark, definition}

Jede Kurve hat ein \textbf{(und nur ein!)} harmonisches \emph{Einheitstangentenfeld} (ETF)
\[ T: I \rightarrow V, \quad t \mapsto \frac{X'(t)}{|X'(t)|} \]

Aber -- \textbf{es gibt haufenweise Normalenfelder.}

\end{remark, definition}

\begin{definition}
	Ein \emph{Streifen} (''ribbon'') ist ein Paar $(X,N)$, wobei $$X:I \rightarrow \E$$ eine Kurve und $$ N: I \rightarrow V $$ ein \emph{Einheitsnormalenfeld} (ENF) ist, d.h.,
	\[N\perp T \text{ und } |N|=1. \]
\end{definition}

\begin{remark, definition}
	(Im dreidimensionalen Raum können wir folgendes sagen:)
	Ein Streifen ist also eine Kurve mit einer ''vertikalen Richtung''.
	Weiters erhält man eine ''seitwärts Richtung'' durch die \emph{Binormale} $$B:=T\times N : I\rightarrow V.$$ (Hier ist $T \times N$ das ''bekannte'' Kreuzprodunkt)
	
\end{remark, definition}

\begin{lemma, definition}
	
	Der \emph{(angepasste) Rahmen} eines Streifens $(X,N): I\rightarrow\E^3\times S^2$ ist eine Abbildung \[ F=(T,N,B): I\rightarrow SO(V) \] seine \emph{Strukturgleichungen} sind von der Form \[ F' = F \phi \text{ mit } \phi =|X'| \begin{pmatrix}
	0 & - \kappa_n & \kappa_g \\
	\kappa_n & 0 & -\tau\\
	-\kappa_g & \tau & 0
	\end{pmatrix}, \] wobei\begin{enumerate}
		\item $\kappa_n$ die \emph{Normalkrümmung}
		\item $\kappa_g$ die \emph{geodätische Krümmung}, und
		\item $\tau$ die \emph{Torsion} des Streifens $(X,N)$ \\ bezeichnen.
		
	\end{enumerate}

\begin{proof}
	Da $F: I \rightarrow SO(V)$, gilt \[F^tF \equiv id\] und daher \[ 0 = (F^tF)' = F'^tF + F^tF' = (F\phi)^tF + F^tF\phi = \phi^tF^tF + F^tF\phi = \phi^t+ \phi  ~,\] d.h., $\phi:I \rightarrow o(V)$ ist schiefsymmetrisch. Insbesondere: Es gibt Funktionen $\kappa_n, \kappa_g, \tau$, so dass $\phi$ von der behaupteten Form ist.
\end{proof}

\textbf{Wiederholung:}
$$O(V) = \{ A \in End(V) ~ | ~ A^tA\equiv id \}$$
$$SO(V) = \{ A \in O(V) ~ | ~\det(A) = 1 \}$$
$$o(V) = \{ B \in End(V) ~ | ~ B^t +B \equiv 0 \}$$

\begin{remark}
	Krümmung und Torsion eines Streifens sind \emph{geometrische Invarianten} des Streifens, d.h., sie sind unabhängig von Position und (in gewisser Weise) Parametrisierung des Streifens.
	
	\begin{enumerate}
		\item ist $(\widetilde{X}, \widetilde{N}) = (\widetilde{O} + A(X-O), AN)$ mit $O,\widetilde{O} \in \E$ und $A \in SO(V)$ eine Euklidsche Bewegung des Streifens $(X,N)$, so sind $\widetilde{T}= AT$ und $\widetilde{B}= AT\times AN = A(T\times N) = AB$, also $\widetilde{F}=AF$ und damit $\widetilde{\phi} = \widetilde{F}^t\widetilde{F}'= F^tA^tAF' = \phi$. (Da $A \in SO(V))$
		
		\item ist $s \mapsto (\widetilde{X}, \widetilde{N})(s) = (X,N)(t(s))$ eine \emph{orientierungstreue Umparametrisierung}, d.h., $t' >0$, von $t \mapsto (X,N)(t)$, so gilt
		\[\widetilde{\phi}(s) = \widetilde{F}^t(s)\widetilde{F}'(s) = F^t(t(s))F'(t(s))\cdot t'(s) =\phi(t(s)) \cdot t'(s) \] und \[ |\widetilde{X}'(s)| = |X'(t(s))|\cdot|t'(s)| =  |X'(t(s))|\cdot t'(s) \] und damit $\widetilde{\kappa_n}(s)= \kappa_n(t(s))$ usw.
	\end{enumerate}
\end{remark}

\begin{lemma}
	Für einen Streifen $(X,N)$ gilt
	\begin{align*}
		\kappa_n 	= -\frac {\skal{N',T}}{\abs{X'}} = \frac {\skal{N,T'}}{\abs{X'}}, &&
		\kappa_g 	= -\frac {\skal{B,T'}}{\abs{X'}} = \frac {\skal{B',T}}{\abs{X'}}, &&
		\tau  		= -\frac {\skal{N,B'}}{\abs{X'}} = \frac {\skal{N',B}}{\abs{X'}} .
	\end{align*}
\end{lemma}
\begin{proof}
	Dies folgt sofort aus der Definition von $\kappa_n,\kappa_g$ und $\tau$ und aus der Orthonormalität von $T,N$ und $B$.
\end{proof}

\begin{remark, definition}
	Ist ein Streifen $(\widetilde{X},\widetilde{N})$ gegeben durch eine \emph{Normalrotation} eines Streifens $(X,N)$, d.h., $\widetilde{X}, \widetilde{N} = (X,N \cos \varphi + B \sin \varphi)$ mit $\varphi: I \rightarrow \R$, so gilt
	\begin{equation*}
	\begin{pmatrix} 
		\widetilde{\kappa_n}\\
		\widetilde{\kappa_g}
	\end{pmatrix}
	=
	 \begin{pmatrix} 
	 \cos \varphi & - \sin \varphi \\
	 \sin \varphi & \cos \varphi
	 \end{pmatrix}
	 \begin{pmatrix}
	 \kappa_n\\
	 \kappa_g
	 \end{pmatrix}
	\end{equation*} und
	$$\widetilde{\tau} = \tau + \frac{\varphi'}{|X'|}  .$$
	
	
\end{remark, definition}
	
\end{lemma, definition}

\begin{example}
	
	\begin{enumerate}
		 \item \textbf{Helix}: Betrachte den Streifen $(X,N)$ mit $$t \mapsto X(t) = o + e_1r\cos t + e_2r \sin t + e_3ht$$ und 
		$t \mapsto N(t) = -(e_1\cos t + e_2 \sin t)$.
		Für $$T(t) = (-e_1r\sin t + e_2r \cos t + e_3h) \frac{1}{\sqrt{r^2 + h^2}}$$ und
		$$B(t) = (e_1h\sin t - e_2h \cos t + e_3r) \frac{1}{\sqrt{r^2 + h^2}}$$ bekommt man
		$F = (T,N,B): \R \rightarrow SO(V)$ und damit \[ T' = N \cdot \frac{r}{\sqrt{r^2 + h^2}} \]		\[ N' = T \cdot \frac{-r}{\sqrt{r^2 + h^2}} + B \cdot \frac{h}{\sqrt{r^2 + h^2}} \]	
		\[B'= \frac{-h}{\sqrt{r^2 +h^2}}\]
		also (mit $|X'|=\sqrt{r^2 + h^2}$),
		\begin{align*}
		\kappa_n  = \frac{r}{r^2 + h^2}, &&
		\kappa_g  = 0, &&
		\tau = \frac{h}{r^2+ h^2}. 
		\end{align*} 
		
		\item \textbf{sphärische Kurve}: Sei $s\mapsto X(s) \in \E^3$ eine bogenlängenparametrisierte Kurve, d.h. mit Mittelpunkt $O \in \E^3$ und Radius $r>0$, der Sphäre gilt:
		\[ |X-O|^2 \equiv r^2 \text{ und } |X'|^2 \equiv 1  .\]
		Bemerke:
		$\langle X',X-O \rangle = \frac 12 (|X-O|^2)' = 0  .$
		Also liefert $N :=(X-O)\frac 1r $ ein ENF. Damit berechnen wir
		\begin{align*}
			\kappa_n &= - \langle N', T \rangle \equiv \frac 1r\\
			\kappa_g &= - \langle B, T' \rangle = - \frac 1r \langle X' \times (X-O), X'' \rangle = \frac{det(X-O,X',X'')}{r}\\
			\tau &= \langle N',B\rangle = \frac{1}{r^2} \langle X',X'\times (X-O) \rangle \equiv 0  .
			\end{align*}
		 
	\end{enumerate}

\end{example}

\begin{remark}
	$\kappa_g \equiv 0$ im ersten Bsp. und $\tau \equiv 0$ im zweiten Bsp.
\end{remark}

\begin{theorem}[Fundamentalsatz für Streifen]
	Seien $$\kappa_n, \kappa_g, \tau: I \rightarrow \R, \quad s \mapsto \kappa_n(s), \kappa_g(s), \tau(s)$$ gegeben. Dann gibt es eine bogenlängenparametrisierte Kurve $$X: I \rightarrow \E$$ und ein ENF $$N: I \rightarrow V  ,$$ so dass $\kappa_n, \kappa_g, \tau$ Normal- bzw. geodätische Krümmung und Torsion des Streifens $(X,N)$ sind.\\ Dieser Streifen ist bis auf Euklid. Bewegung eindeutig.
\end{theorem}

\begin{proof}
	
	Wähle $o \in I$ und $F_o \in SO(V)$ beliebig und fest.
	Nach Satz von Picard-Lindelöf hat das AWP 
		\[ F' = F\phi, ~ F(o)= F_o \] mit 
	\[ \phi =\begin{pmatrix}
		0 & -\kappa_n & \kappa_g \\
		\kappa_n & 0 & -\tau \\
		-\kappa_g & \tau & 0
	\end{pmatrix}\]
	eine eindeutige Lösung $F = (T,N,B) : I \rightarrow End(V)$.
	Nun zeigen wir, dass $F$ ein Rahmen ist:
	\begin{enumerate}
		\item $(FF^t)' = F( \phi  + \phi^t) F^t \equiv 0 $ also $FF^t \equiv id,$ und $F:I \rightarrow O(V)$
		\item $\det: O(V) \rightarrow \{\pm1\}$ ist stetig, also $\det F: I \rightarrow \{ \pm 1 \}$ konstant und somit $$\det F =\det F_o = 1  ,$$ also $F: I \rightarrow SO(V).$
	\end{enumerate}
	
	Insbesondere $|T| \equiv 1$ und man erhält eine bogenlängenparametrisierte Kurve
		\[X: I \rightarrow \E^3, t \mapsto O + \int_{o}^{t} T(s) ds  . \]
		
	$(X,N)$ mit $F=(T,N,B)$ liefert einen Streifen, Krümmung und Torsion wie behauptet.\\
	Eindeutigkeit bis auf Euklid. Bewegung folgt aus der Eindeutigkeit in Picard-Lindelöf und jener der Integration.
\end{proof}

\subsection{Normalzusammenhang \& Paralleltransport}

\begin{definition}

Für ein Normalenfeld kann man die Ableitung $N' = {\color{red}N' - \langle N',T  \rangle T} +  {\color{ForestGreen}\langle N',T  \rangle T} $ in {\color{red}Normal-} und {\color{ForestGreen}Tangentialanteil} zerlegen.

\end{definition}

\begin{definition}
	Ein Normalenfeld $N: I \rightarrow V $ entlang $X: I \rightarrow \E$ heißt \emph{parallel}, falls $\nabla^\perp N := N'-  \langle N',T\rangle T = 0,$ wobei $\nabla^\perp $ den \emph{Normalzusammenhang} entlang $X$ bezeichnet.
\end{definition}

\begin{remark}
	Hier wird \underline{nicht} $| N | = 1$ angenommen.
\end{remark}

\begin{lemma}
	Der Normalzshg. ist \emph{metrisch}, d.h., $$ \langle N_1,N_2 \rangle ' = \langle \nabla^\perp N_1,N_2 \rangle + \langle N_1, \nabla^\perp N_2 \rangle; $$
	
	parallele Normalenfelder haben konstante Länge und schließen konstante Winkel ein.
\end{lemma}

\begin{proof}
		
		Für Normalenfelder $N_1, N_2 : I \rightarrow V$ entlang $X: I\rightarrow \E$ gilt:
		\[ \langle \nabla^\perp N_1,N_2 \rangle + \langle N_1,\nabla^\perp N_2 \rangle = \langle N_1'- \langle N_1,T \rangle T,N_2 \rangle + \langle N_1,N_2' - \langle N_2',T \rangle T \rangle\]
		\[ =\langle N_1',N_2 \rangle + \langle N_1,N_2' \rangle = \langle N_1,N_2 \rangle ' .\]
		Insbesondere sind $N_1,N_2$ parallel, so ist $\langle N_1,N_2 \rangle ' = 0$
		
		Damit: 
		
		\begin{enumerate}
			\item  ist $N$ parallel, so gilt $$ (|N|^2)' = 2 \langle N,\nabla^\perp N \rangle = 0$$
			
			\item  sind $N_1, N_2$ parallel, so ist der Winkel $\alpha$ zwischen $N_1,N_2$
			$$ \alpha = \arccos \frac{\langle N_1,N_2 \rangle}{|N_1||N_2|}= const. $$
		\end{enumerate}
		
\end{proof}

\begin{example}
	Für einen Kreis $t \mapsto X(t) = O + (e_1 \cos t + e_2 \sin t)r$ ist $t\mapsto N(t) := e_1 \cos t + e_2 \sin t$ ein paralleles Normalenfeld.
\end{example}


\begin{remark}
	Ist $(X,N)$ ein Krümmungsstreifen, $\tau \equiv 0,$ so ist $N$ parallel. Aus $ N' = (-\kappa_n T + \tau B)\abs{X'} $ folgt
	\[  \nabla^\perp N  = (-\kappa_n T + \tau B)\abs{X'} +\kappa_nT = B\tau\abs{X'}= 0. \]
	Andererseits: Ist $N: I \rightarrow V$ parallel entlang $X: I \rightarrow \E$ , so ist $ (X,\frac{N}{|N|}) $ Krümmungsstreifen (falls $N \not = 0$).
\end{remark}

\begin{remark}
	Ist $N$ parallel längs $X$, so auch $B = T\times N.$
\end{remark}

\begin{remark}
	Ist $(X,N)$ durch eine Normalrotation von $(X, \widetilde{N})$ gegeben, d.h. \[ (X,N) = (X,\widetilde{N} \cos \varphi + \widetilde{B}\sin \varphi) \] mit $\varphi: I \rightarrow \R$, so gilt 
	\[ \tau = \widetilde{\tau} + \frac{\varphi '}{|X'|}; \]
	folglich: Man erhält einen Krümmungsstreifen bzw. ENF $N: I \rightarrow V$ einer Kurve $X: I \rightarrow \E$ durch \[ N = \widetilde{N} \cos \varphi + \widetilde{B} \sin \varphi \text{ mit } \varphi(t) = \varphi_o - \int_{o}^{t} \tau(t) ds. \]
	
	Wobei $\varphi_o$ eine Integrationskonstante ist und eine konstante Normaldrehung liefert und $ds$ für $|X'| dt$ -- das Bogenlängenelement -- steht.
	
	Da konstante Skalierungen eines parallelen Normalenfeldes parallel ist, folgt:
	
	
\end{remark}

\begin{lemma}
	Sei $X: I \rightarrow \E$ eine Kurve, $o \in I$ und $N_o \in N_oX$. Dann existiert ein eindeutiges paralleles Normalenfeld $N: I \rightarrow V$ mit $N(o) = N_o$
\end{lemma}

\begin{proof}
	der Beweis folgt aus der Bemerkung darüber. Allerdings nur für Dimension 3. Der Beweis gilt auch sonst, dann braucht man allerdings Picard-Lindelöf
\end{proof}

\begin{example}
	Für das ''radikale'' ENF $\widetilde{N} = - (e_1 \cos t + e_2 \sin t)$ der Helix
	$$ X = O + e_1 r \cos t + e_2 r \sin t + e_3 h t$$
	ist $\widetilde{\tau} = \frac{h}{r^2 + h^2}$. Also liefert \[ N(t) := \widetilde{N}(t) \cos (\frac{ht}{\sqrt{r^2 + h^2}}) + \widetilde{B}(t) \sin (-\frac{ht}{\sqrt{r^2 + h^2}}) \] ein paralleles Normalenfeld.

\end{example}

\begin{lemma, definition}
	Parallele Normalenfelder entlang $X: I \rightarrow \E$ definieren eine lineare Isometrie von $N_oX$ nach $N_tX$. Diese Isometrie heißt \emph{Paralleltransport} entlang $X$.
\end{lemma, definition}

\begin{remark}
	Dies erklärt den Begriff ''Zusammenhang'' für $\nabla^\perp$:$\nabla^\perp$ liefert einen Zusammenhang zwischen Normalräumen einer Kurve.
\end{remark}

\begin{proof}
	Wähle $N_o \in N_oX$; nach Lemma vorher gibt es ein eindeutiges(!) paralleles NF $N : I \rightarrow V$ entlang $X$ mit $N(o) = N_o$; also definiere durch 
	\[\pi_t: N_oX \rightarrow N_t X,~ N_o \mapsto N(t)  \] eine wohldefinierte Abbildung. Da die Gleichung $\nabla^\perp N = 0$ linear ist, sind konstante(!) Linearkombinationen von Lösungen wieder Lösungen -- also ist $\pi_t$ linear.
\end{proof}

\subsection{Frenet Kurven}

Wir diskutieren $\kappa_g \equiv 0$ (vorheriger Abschnitt $\tau \equiv 0$).

Bemerke: ist $(\widetilde{X},\widetilde{N}) =(X,N \cos \varphi + B \sin \varphi )$ Normalrotation eines Streifens $(X,N)$, so gilt 
\begin{equation*}
\begin{pmatrix} 
\widetilde{\kappa_n}\\
\widetilde{\kappa_g}
\end{pmatrix}
=
\begin{pmatrix} 
\cos \varphi & - \sin \varphi \\
\sin \varphi & \cos \varphi
\end{pmatrix}
\begin{pmatrix}
\kappa_n\\
\kappa_g
\end{pmatrix}
\end{equation*}

insbesondere $\widetilde{\kappa}_n = - \kappa_g $ und  $\widetilde{\kappa}_g = \kappa_n$
für $(\widetilde{X}, \widetilde{N}) = (X,B)$. % N und B werden vertauscht und damit vertauscht sich \kappa_g und \kappa_n

\begin{definition}
	$X: I \rightarrow \E^3$ heißt \emph{Frenet Kurve}, falls \[ \forall t \in I: (X' \times X'')(t) \not = 0. \]
\end{definition}

\begin{remark}
	In diesem Kapitel wird stets der 3-dimensionale Raum angenommen.
\end{remark}

\begin{remark}
	Die Frenet-Bedingung ist invariant unter Umparametrisierung.
\end{remark}

\begin{lemma, definition}
	Ist $X: I \rightarrow \E^3$ Frenet, so gilt \[ \forall t \in I : T'(t) \not = 0 \] und $\frac{T'}{|T'|} =: N$ definiert ein ENF: Dies ist die \emph{Hauptnormale} von $X$.
\end{lemma, definition}

\begin{proof}
	Mit der Frenet-Bedingung:
	\[0 \not = X' \times X''= X' \times (T|X'|)' = X' \times T|X'|' + X'\times T'|X'| \Rightarrow T' \not = 0 \]
	
	Weiters: \[ 0 = (1)' = (|T|^2)' = 2\langle T,T' \rangle, \]
	also definiert $N = \frac{T'}{|T'|}$ ein ENF.
\end{proof}

\begin{lemma, definition}
	Ist $X: I \rightarrow \E^3$ Frenet mit Hauptnormale $N: I \rightarrow V$, so sind die Strukturgleichungen des \emph{Frenet Rahmens} $F = (T,N,B)$ der Kurve die \emph{Frenet-Serret Gleichungen} $F' = F\phi$ mit $\phi = |X'|\begin{pmatrix}
	0 & - \kappa & 0\\
	\kappa & 0 & - \tau \\
	0 & \tau & 0
	\end{pmatrix}$
	 mit der \emph{Krümmung} $\kappa > 0$ und \emph{Torsion} $\tau$ der Frenet Kurve $X$.
\end{lemma, definition}

\begin{remark}
	Für eine Frenet Kurve (mit Hauptnormale) gilt also $\kappa_n = \kappa > 0$ und $\kappa_g = 0$.
\end{remark}

\begin{proof}
	Für einen Frenet Rahmen gilt 
	\[ \kappa_g = - \frac{\langle T', B \rangle}{|X'|} = - \frac{\langle N, B \rangle |T'|}{|X'|} = 0, \]
	
	\[ \kappa_n = \frac{\langle T', N \rangle}{|X'|} = \frac{\langle N, N \rangle |T'|}{|X'|} >0. \]
\end{proof}

\begin{example}
	Eine Helix
	
	\[ X = O + e_1 r \cos t + e_2 r \sin t + e_3 h \]
	
	hat Hauptnormalenfeld (s. Kapitel 1.2) \[ t \mapsto N(t) = - (e_1 \cos t + e_2 \sin t) \] und Krümmung und Torsion 
	\[ \kappa \equiv \frac{r}{r^2 +h^2} \text{ bzw. } \tau \equiv \frac{h}{r^2+h^2} . \]
\end{example}

\begin{remark}
	Krümmung und Torsion einer Frenet Kurve sind \[ \kappa = \frac{|X' \times X''|}{|X'|^3} \text{ bzw. }  \tau = \frac{\det(X',X'',X''')}{|X' \times X''|^2} \]
	
	Insbesondere: Krümmung und Torsion hängen nur von der Kurve ab (daher: ''Krümmung'' und ''Torsion der Kurve'').
\end{remark}

\begin{theorem}[Fundamentalsatz für Frenet Kurven]
	Für zwei Funktionen \[ s \mapsto \kappa(s), \tau(s) \text{ mit } \forall s \in I: \kappa(s) > 0 \] gibt es eine Bogenlängenparametrisierte Frenet Kurve $X: I \rightarrow \E ^3$ mit Krümmung $\kappa$ und Torsion $\tau$. 
	Weiters: $X$ ist eindeutig bis auf Euklid. Bewegung.
\end{theorem}

\begin{proof}
	Nach dem Fundamentalsatz für Streifen existiert bogenlängen-parametrisierte Kurve $X: I \rightarrow \E^3$ und ENF $N: I \rightarrow V$ so dass der Streifen $ (X,N)$ Krümmung und Torsion \[  \kappa_n = \kappa, \kappa_g = 0, \tau = \tau,  \] d.h.\[F' = F\phi \text{ mit } \phi = \begin{pmatrix}
	0 & - \kappa & 0\\
\kappa & 0 & - \tau \\
0 & \tau & 0
	\end{pmatrix} \] hat.
	
	Insbesondere $T' = N\kappa \not = 0$, daher:
	\begin{enumerate}
	\item $X$ ist Frenet, da
	\[ X' \times X'' = T \times T' = T \times N\kappa \not = 0 \]
	und \item  $N$ ist Hauptnormalenfeld, da \[ N = T'\frac{1}{\kappa} = \frac{T'}{|T'|}. \]
	
	\end{enumerate}
\end{proof}

\begin{remark}
	Einen einfacheren Fundamentalsatz gibt es für ebene Kurven. (Aufgabe: Formulieren und -- ohne Picard-Lindelöf-- beweisen!)
\end{remark}

\begin{example}
	Seien $\kappa > 0, \tau \in \R$ Zahlen. Nach dem Fundamentalsatz existiert (eind. bis auf Eukl. Bewegung) eine bogenlängenparametrisierte Frenet Kurve mit Krümmung $\kappa$ und Torsion $\tau$. Andererseits: \[ \R \ni s \mapsto X(s) = O + e_1 r \cos \frac{s}{\sqrt{r^2+h^2}} + e_2 r \sin \frac{s}{\sqrt{r^2+h^2}} + e_3 \frac{hs}{\sqrt{r^2 + h^2}}\] ist bogenlängenparam. Frenet Kurve mit Krümmung $\kappa$ und Torsion $\tau$ für \[ r= \frac{\kappa}{\kappa^2 + \tau^2} \text{ und } h = \frac{ \tau}{\kappa^2 + \tau^2}. \]
	
	Damit haben wir bewiesen:
\end{example}

\begin{theorem}[Klassifikation der Helices]
	Eine Frenet Kurve ist genau dann Helix, wenn sie konstante Krümmung und Torsion hat.
\end{theorem}

\begin{example}
	
	Falls $  (u,v) \mapsto X(u,v) $ Werte in einer (festen) Ebene annimmt,
	\[ \pi = \{ X \in \E^3 ~|~ \skal{X-0,n} = d \} \] d.h., $ \skal{dX,n}=0 $, so ist $ N \equiv \pm n $, demnach also $\S \equiv 0$ und jeder Punkt der Fläche ist Flachpunkt. 
	
\end{example}


\pagebreak

\section{Flächen}

\subsection{Parametrisierung \& Metrik}
\begin{definition}
	

Eine Abbildung \[ X:\R^2 \supseteq M \rightarrow \E \] heißt \emph{parametrisierte Fläche}, falls $M$ offen und zusammenhängend ist und $X$ regulär ist, d.h., $  \forall(u,v) \in M~:~d_{(u,v)}X: \R^2 \rightarrow V $ ist injektiv.
Wir sagen auch: $  X  $ ist eine \emph{Parametrisierung} der \emph{Fläche} $ X(M) \subseteq \E $.
Wobei $ d_{(u,v)} $ definiert ist über:
	\[d_{(u,v)}X(\pmat{x \\y})= X_u(u,v)\cdot x + X_v(u,v) \cdot y. \]

\end{definition}

\begin{remark}
	
 Äquivalent zur letzten Forderung ist die Forderung, dass die Jacobi-Matrix maximalen Rang hat. Diese braucht aber eine Festgelegte Basis, was oft zu Schwierigkeiten bei Berechnungen führt und wird daher von Prof. Jeromin nicht empfohlen.

\end{remark}

\begin{remark}
	Einmal mehr sind alle geforderten Abbildungen so oft differenzierbar, wie wir das wünschen.
\end{remark}

\begin{remark}
	
	$ d_{(u,v)} : \R^2 \rightarrow V$ ist die Ableitung am Punkt $ (u,v) \in M $, \[X(u+x,v+y) \approx X(u,v) + d_{(u,v)}  X(\begin{pmatrix}
	x\\
	y
	\end{pmatrix}) = X(u,v) + X_u(u,v)\cdot x + X_v(u,v)\cdot y,  \]
	wir können also identifizieren: \[ d_{(u,v)}X \cong (X_u,X_v)(u,v), \]
	bzw., nach Wahl einer Basis von V, mit der Jacobi-Matrix am Punkt $ (u,v) $.
	
\end{remark}

\begin{example}
	


Ein \emph{Helicoid} $ X: \R^2 \rightarrow \E^3 $ ist die \emph{(Regel-)Fläche}  
\[ \R^2 \ni (r,v) \mapsto O + e_1r \cos(v) + e_2 r \sin(v) + e_3 v \in \E^3. \]

Wir zeigen, dass $(X_r,X_v)(r,v)$ linear unabhängig für alle $ (r,v) \in \R^2 $ sind:

\[ X_r(r,v) = e_1\cos(v) + e_2 \sin(v) \neq 0 \]
\[ X_v(r,v) = -e_1r \sin(v) + e_2 r \cos(v) + e_3 \neq 0 \]

und da $ X_v(r,v) $ von $e_3 \neq 0$  abhängt sind die beiden linear unabhängig.

\end{example}

\begin{example}
	
	Eine übliche Parametrisierung von $ \mathbb{S}^2 \subseteq \E^3 $ (mit Mittelpunkt $O \in \E^3$) ist  
	\[ (u,v) \mapsto O + e_1\cos(u)\cos(v) + e_2\cos(u)\sin(v)+e_3\sin(u) \]
	liefert keine parametrisierte Fläche, da die Sphäre an den Polen nicht regulär ist.
	Dies ist also nur eine Parametrisierung auf $ M = (-\frac{\pi}{2},\frac{\pi}{2}) \times \R $
	
	Insbesondere kann man sogar zeigen, dass es keine (reguläre) Parametrisierung der (ganzen) Sphäre gibt (''Hairy Ball Theorem'' bzw. ''Satz vom Igel'').
	
	$ \mathbb{S}^2 $ ist also \textbf{keine} Fläche im Sinne der Definition. Dieses ''Problem'' wird später gelöst.
\end{example}

\begin{lemma, definition}
	
	Die \emph{induzierte Metrik} oder \emph{erste Fundamentalform} einer parametrisierten Fläche $ X: M \rightarrow \E$ ist definiert durch
	\[ \mathrm{I} := \langle dX,dX \rangle. \]
	Für jeden Punkt $ (u,v) \in M $ liefert 
	\[ \R^2 \times \R^2 \ni \left( \pmat{x_1\\y_1}, \pmat{x_2\\y_2} \right)  \mapsto \mathrm{I} \big|_{(u,v)}(\begin{pmatrix}
	x_1 \\
	y_1
	\end{pmatrix},\begin{pmatrix}
	x_2 \\
	y_2
	\end{pmatrix}) := \left\langle d_{(u,v)}X(\begin{pmatrix}
	x_1 \\
	y_1
	\end{pmatrix}),  d_{(u,v)}X(\begin{pmatrix}
	x_2 \\
	y_2
	\end{pmatrix})  \right\rangle \]
	eine positiv definite, symmetrische Bilinearform.
\end{lemma, definition}

\begin{proof}
	%Das darf ich wohl machen.. \#getrektluka
	Zu zeigen ist, dass $I\big| _{(u,v)}$ für jeden Punkt $(u,v)\in M$ eine positiv definite symmetrische Bilinearform ist. 
	
	Weil $ I\big| _{(u,v)} $ eine Komposition aus linearen Funktionen und einer Bilinearform ist, ist auch $I\big| _{(u,v)}$ linear. Die Symmetrie ist ebenfalls leicht ersichtlich. Fehlt noch die positive Definitheit.
	
	Sei $\pmat{x\\y} \neq 0$ beliebig, so gilt
		\[ I\big| _{(u,v)}(\pmat{x\\y},\pmat{x\\y})
			= \skal{d_{(u,v)}X(\pmat{x\\y}),d_{(u,v)}X(\pmat{x\\y})}> 0. \]
	Die letzte Ungleichung gilt, weil $d_{(u,v)}X$ injektiv und linear ist, daher bildet nur $0$ auf $0$ ab. Also ist $d_{(u,v)}X(\pmat{x\\y})\neq 0$. Der Rest folgt, weil $\skal{.,.}$ positiv definit ist.
\end{proof}

\begin{remark}
	
	$I$ wird oft mit Hilfe der \emph{Gramschen Matrix} notiert: \[ \mathrm{I} = \begin{pmatrix}
	E & F\\
	F & G
	\end{pmatrix} = E du^2 + 2Fdudv + Gdv^2 \] mit
	\[ E=|X_u|^2,\quad F = \langle X_u,X_v \rangle,\quad G = |X_v|^2.  \]
	


Dann gilt für $ (u,v) \in M$ : \[ I \big|_{(u,v)}(\begin{pmatrix}
x_1 \\
y_1
\end{pmatrix},\begin{pmatrix}
x_2 \\
y_2
\end{pmatrix}) := \skal{ d_{(u,v)}X(\begin{pmatrix}
x_1 \\
y_1
\end{pmatrix}),  d_{(u,v)}X(\begin{pmatrix}
x_2 \\
y_2
\end{pmatrix}) } \] \[= \skal{ X_u(u,v)x_1 + X_v(u,v)y_1, X_u(u,v)x_2 + X_v(u,v)y_2 } \]
\[ = E(u,v)x_1x_2 + F(u,v)(x_1y_2 + x_2y_1) + G(u,v)y_1y_2 \]
\[ = \begin{pmatrix}
x_1 \\
y_1
\end{pmatrix}^t \begin{pmatrix}
E & F \\ 
F & G
\end{pmatrix}\Bigg|_{(u,v)} \begin{pmatrix}
x_2\\
y_2
\end{pmatrix} \]

\end{remark}

\begin{example}
	
	\begin{enumerate}
		
		\item Ein \emph{Zylinder}
			\[ (u,v) \mapsto X(u,v) := O + e_1x(u) + e_2y(u) +e_3v \]
		hat induzierte Metrik \[ I = (x'^2 + y'^2)du^2 + dv^2. \]
		Insbesondere: Ist $ u \mapsto O+e_1x(u) + e_2y(u) $ bogenlängenparametrisiert, so ist $ X $ \emph{isometrisch},
		\[I = du^2 + dv^2\]
		
		\item Das \emph{Helicioid} \[ (r,v) \mapsto O + e_1r\cos(v)+e_2r\sin(v) + e_3v \]
		hat Metrik \[ I\big|_{(r,v)}=dr^2 + (1+r^2)dv^2. \]
		Mit einer Umparametrisierung $r = r(u) = \sinh(u)$ erhält man \[ I \big|_{(u,v)}= \cosh^2(u)(du^2+dv^2), \] d.h., $X$ wird \emph{konform} (winkeltreu).
		
	\end{enumerate}
	
\end{example}

\begin{definition}
	
	Eine Parametrisierte Fläche $ X : M \rightarrow \E $ heißt 
	\begin{enumerate}
		\item \emph{konform}, falls $ E = G, F = 0 $
		\item \emph{isometrisch}, falls $ E = G = 1, F= 0 $.
	\end{enumerate}
	
\end{definition}


\begin{remark}
	
	Für Kurven: Jede Kurve kann isometrisch/nach Bogenlänge umparametrisiert werden(\ref*{umpar}).  Für Flächen: Im Allgemeinen gibt es keine isometrische (Um-)Parametrisierung.
	
\end{remark}

\begin{theorem}
	
	Jede Fläche kann lokal konform (um-)parametrisiert werden. 
	
\end{theorem}

\begin{proof}
	
	Ist echt cool laut Jeromin (braucht bissi so Fana und so \ldots).
	Falls der Leser Zeit hat, wird ihm nahegelegt den Beweis nachzuschauen.
	Hier ein Link zu einem Beweis dieser Tatsache:

 	\href{https://thibaultlefeuvre.files.wordpress.com/2017/02/coord_isotherm.pdf}{Link zu PDF hier klicken.}
 	
	\end{proof}

\begin{remark}
	
	Dieser Satz ist die Grundlage, um (reelle) Flächen als komplexe Kurve zu intepretieren. Eine weitreichende Betrachtungsweise \ldots
\end{remark}

\begin{remark}
	
	Um den Satz zu verstehen:\\
	''lokal'' heißt, dass -- für jeden Punkt $ (u,v) \in M $ -- der Definitionsbereich $M$ so eingeschränkt werden kann -- auf eine Umgebung des Punktes$ (u,v) $ -- dass die Behauptung wahr ist;
	''Umparametrisierung'' wie für Kurven definiert:
\end{remark}

\begin{definition}
	
	Eine \emph{Umparametrisierung} einer parametrisierten Fläche $ X: M \rightarrow \E $ ist eine neue parametrisierte Fläche 
		\[ \widetilde{X}=X\circ(u,v), \quad \widetilde{M} \rightarrow \E, \]
	mit einem \emph{Diffeomorphismus}: 
		\[ (u,v): \widetilde{M} \rightarrow M, \] 
	d.h., eine glatte $ (C^\infty) $ Bijektion mit glatter Inverser $ (u,v)^{-1}:M \rightarrow \widetilde{M}. $
		
\end{definition}

\begin{remark}
	
	Für \[(x,y)\mapsto \widetilde{X}(x,y) = X(u(x,y),v(x,y)) \in \E^3  \] gilt (Kettenregel) \[ \widetilde{X}_x = (X_u\circ (u,v))\cdot u_x + (X_v \circ (u,v)) \cdot v_x \]
	\[ \widetilde{X}_y = (X_u \circ (u,v))\cdot u_y + (Y_v \circ (u,v)) \cdot v_y \]
	und somit 
		\[ \widetilde{X}_x \times \widetilde{X}_y = ((X_u\times X_v)\circ (u,v))\cdot (u_xv_y - u_yv_x), \]
	d.h., $ \widetilde{X} $ ist regulär.
	
\end{remark}

\subsection{Gaußabbildung und Weingartentensor}

\begin{definition}
	
	Eine Fläche $ X : M \rightarrow \E^3 $ hat an jedem Punkt $ X(u,v) $ eine \emph{Tangentialebene} und eine \emph{Normalgerade}:
		\[ \mathcal{T}(u,v) := X(u,v) + [ \{ X_u(u,v), X_v(u,v) \} ], \]
		\[ \mathcal{N}(u,v) := X(u,v) + [ \{ (X_u \times X_v)(u,v) \} ]; \]
	dies entspricht einer orthogonalen Zerlegung
		\[ V= [ \{ X_u(u,v),X_v(u,v) \} ] \oplus_\perp [ \{ (X_u \times X_v)(u,v) \} ] \]
	von $ V $ in einen \emph{Tangentialraum} und einen \emph{Normalraum} von $X$ am Punkt $X(u,v).$
	
\end{definition}
	
\begin{definition}
	
	Das \emph{Tangential-} bzw. \emph{Normalenbündel} einer Fläche $ X: M \rightarrow \E $ ist gegeben durch die Abbildung 
		\[ (u,v) \mapsto T_{(u,v)}X := [ \{X_u(u,v),X_v(u,v)\} ], \]
		\[ (u,v) \mapsto N_{(u,v)}X := \{ X_u(u,v),X_v(u,v) \}^{\perp}. \]
	Eine Abbildung $ Y : M \rightarrow V $ heißt 
	
	\begin{itemize}
		
		\item \emph{Tangentialfeld} entlang $ X $, falls 
			\[ \forall (u,v) \in M: ~ Y(u,v) \in T_{(u,v)}X \],
			
		\item \emph{Normalenfeld} entlang $ X $, falls
			\[ \forall(u,v) \in M: ~ Y(u,v) \in N_{(u,v)}X. \]
		
	\end{itemize}

	Die \emph{Gaußabbildung} einer Fläche $ X: M \rightarrow \E^3 $ ist das Einheitsnormalenfeld (ENF):
		\[ N:= \dfrac{X_u \times X_v}{\abs{X_u \times X_v}}: M \rightarrow V. \]
	
\end{definition}
	
\begin{example}
	\textbf{Roationsfläche:}
	Für jede Rotationsfläche
		\[ (u,v) \mapsto X(u,v) := O + e_1r(u)\cos(v) + e_2 r(u) \sin(v) + e_3h(u) \]
	ist jede Profilkurve $ v \equiv \mathrm{const} $ der orthogonale Schnitt der Fläche mit der Ebene $ x\sin(v) = y \cos(v) $ der Meridiankurve; die Gaußabbildung erhält man also durch $ \dfrac{\pi}{2} $ Drehung des ETFs (''Einheitstangentialfeldes'') in der Ebene der Kurve 
		\[ N(u,v) = \{ -(e_1\cos(v) + e_2\sin(v))h'(u)+e_3r'(u) \} \dfrac{1}{\sqrt{(r'^2 + h'^2)}(u)}. \]
	Überprüfung des Vorzeichens:
	\[ \det \pmat{r'\cos & -r\sin & -h'\cos \\ r'\sin & r\cos & -h'\sin \\ h' & 0 & r'} = h'^2r + r'^2r = r(r'^2 + h'^2) > 0. \]
	
\end{example}
	
\begin{remark}
	
	Die Gaußabbildung einer Fläche ist ein geometrisches Objekt, d.h.,
	nach einer Euklid. Bewegung $ \widetilde{X} = \widetilde{O} + A(X-O), $ liefert
		\[ \widetilde{N}=\dfrac{\widetilde{X}_u \times \widetilde{X}_v}{|\widetilde{X}_u \times \widetilde{X}_v|} = \dfrac{AX_u \times A X_v}{\abs{AX_u \times AX_v}} = \dfrac{A(X_u \times X_v)}{\abs{A(X_u \times X_v)}} = A \dfrac{X_u \times  X_v}{\abs{X_u \times X_v}} = AN. \]
	Das vorletzte Gleichheitszeichen gilt, weil für $A \in \mathrm{SO}(3)$ gilt $ \abs{A}=1 $. 
	Eine Spiegelung liefert $ \widetilde{N}~=~-AN, $ d.h., wechselt das Vorzeichen -- was auch eine (ordnungsumkehrende) Umparametrisierung tut, z.B.: $ (u,v) \mapsto(v,u) $.
	Demnach ist $ N $ ''geometrisch'' bis auf Vorzeichen.
	
\end{remark}

\begin{remark}
	
	Ordnungsprobleme tauchen in unserem Setting mit parametrisierten (!) Flächen nicht auf: Die Gaußabbildung einer parametrisierten Fläche ist wohldefiniert; eine nicht-orientierbare Fläche (e.g. Möbiusband) kann durch eine doppelt überlagerte Parametrisierung beschrieben werden.
	
\end{remark}


\textbf{Erinnerung:} Die Normalkrümmung $ \kappa_n $ eines Streifens $(X,N) $ ist definiert durch 
	\[ 0=N'^T + T\abs{X'} \kappa_n = N'^T + X'\kappa_n, \]
wobei $ t \mapsto N'^T(t) \in T_tX $ den Tangentialanteil von $ N' $ bezeichnet, i.e.
	\[ N'^T = N' - \nabla^\perp N = T\skal{T,N'}. \]

\begin{lemma, definition}
	
	Die Ableitung der Gaußabbildung ist tangentialwertig,
	\[ \forall (u,v) \in M : ~ d_{(u,v)}N : \R^2 \rightarrow T_{(u,v)}X. \]
	Damit können wir den \emph{Formoperator} von $ X $ am Punkt $ (u,v) \in M $ definieren:
	\[ \s\big|_{(u,v)}:= - d_{(u,v)}N \circ (d_{(u,v)}X)^{-1} \in \mathrm{End(T_{(u,v)}X)}. \]
	
\end{lemma, definition}

\begin{proof}
	
	\begin{enumerate}
		
		\item Die Ableitung von N ist tangentialwertig. Nämlich:
			\[ 1 \equiv \abs{N}^2 \Rightarrow 0 = 2 \skal{N,dN} \Rightarrow \forall (u,v) \in M \forall \pmat{x \\y} \in \R^2 : d_{(u,v)}N(\pmat{x \\y}) \in T_{(u,v)X}. \]
		
		\item  $\s$ ist wohldefiniert:
		Da für $ (u,v) \in M, d_{(u,v)}X: \R^2 \rightarrow V $ injektiv ist, liefert dies einen Isomorphismus
			\[ d_{(u,v)}X: \R^2 \rightarrow T_{(u,v)}X \subseteq V, \]
		der invertiert werden kann, um eine lineare Abbildung
		
			\[ (d_{(u,v)}X)^{-1}: T_{(u,v)}X \rightarrow \R^2 \]
		
		zu erhalten.
		
		\item $ \s\big|_{(u,v)} $ ist Endomorphismus:
		Als Verkettung linearer Abbildungen
			\[ T_{(u,v)}X \xrightarrow{(d_{(u,v)}X)^{-1}} \R^2 \xrightarrow{-d_{(u,v)}N} T_{(u,v)}X \]
		
	\end{enumerate}
	
\end{proof}

\begin{remark}
	Die Abbildung $\mapsto \s \big|_{(u,v)} \in End(T_{(u,v)}X)$ liefert ein Endomorphismenfeld $\s$, welches man auch \emph{Weingartentensorfeld} nennt.
\end{remark}

\begin{remark}
	Da $\left( X_u(u,v) , X_v(u,v) \right)$ eine Basis von $T_{(u,v)}X$ ist, kann $\s \big|_{(u,v)}$ durch die Werte auf der Basis bestimmt werden:
	\[ \s X_u = -dN \circ (dX)^{-1}(X_u)= 
	-dN \circ \pmat{1\\0} = -N_u \]
	und 
	\[ \s X_v = -N_v. \]
\end{remark}

\begin{lemma}
	$\s \big|_{(u,v)} \in End(T_{(u,v)}X)$ ist symmetrisch für jedes $(u,v) \in M$.
\end{lemma}

\begin{proof}
	Wir verifizieren Symmetrie auf der Basis $(X_u(u,v),X_v(u,v))$ von $T_{(u,v)}X$:
	
	Da $N \perp X_u,X_v$ erhalten wir 
	\[ 0=\skal{X_u,N}_v = \skal{X_{uv}, N} + \skal{X_u,N_v}
	= \skal{X_{vu},N}- \skal{X_u,\s X_v}. \]
	Ebenfalls
	\[ 0=\skal{X_v,N}_u 
	= \skal{X_{vu},N}- \skal{X_v, \s X_u}. \]
	Also 
	\[ \skal{X_u,\s X_v} = \skal{\s X_u,X_v}. \]
\end{proof}


\begin{remark}
	Wie bei Streifen kann der Formoperator analog zu $\kappa_n$ durch die Gleichung 
	\[ 0=dN + \s \circ dX = dN^T + S\circ dX \] beschrieben werden.
	Der Formoperator ''kodiert'' also die Krümmung einer Fläche.
\end{remark}

\begin{definition}
	Sei $\s$ der Formoperator der Fläche $X : M \to \E^3$, dann heißt
	\begin{itemize}
		\item $H=\frac 12 \mathrm{tr} \s $ die \emph{mittlere Krümmung} von $X$,
		\item $K=\det S$ die \emph{Gauß Krümmung} von $X$ und 
		\item die Eigenwerte $\kappa^\pm = H \pm \sqrt{H^2-K}$ und die \emph{Eigenrichtungen} von $\s$ sind die \emph{Hauptkrümmungen} bzw. \emph{Hauptrichtungen} von $X$. 
	\end{itemize}
\end{definition}

\begin{remark} Es gilt
	$H=\frac 12 (\kappa^+ +\kappa^-)$ -- daher auch der Name \emph{mittlere Krümmung}.
\end{remark}

\begin{example}
	Eine Rotationsfläche parametrisiert nach Bogenlänge ist
	\[ X(u,v)= O+e_1r(u)\cos v +e_2 r(u) \sin v + e_3h(v), \]
	mit $r'^2+h'^2=1$. Damit folgt $r'r''+h'h''=0$. Mit der Gaußabbildung
	\[ N(u,v)= -e_1h'(u)\cos v - e_2h'(u) \sin v + e_3v'(u) \]
	bekommt man
	\[ N_v + X_v\frac {h'}r = (e_1\sin v - e_2\cos v)\left( h' - r\frac {h'}r \right)=0, \]
	\[ N_u + X_u\left( r'h''-r''h' \right)\]
	\[= (e_1 \cos v +e_2\sin v ) \left( h''+r'(r'h''-r''h') \right) + e_3 (r''+h'(r'h''-r''h'))
	= 0. \]
	Also liefern $X_u$ und $X_v$ Krümmungsrichtungen zu Hauptrichtungen
	\[ \kappa^+ = r'h''-r''h' \quad \text{ und }\quad
	\kappa^- = \frac {h'}r. \]		
\end{example}


\begin{remark}
	Formoperatoren und Krümmungen sind geometrische Objekte:
	\begin{itemize}
		\item Ist $\tilde X = X \circ (u,v)$ eine Umparametrisierung und $\tilde N = N \circ (u,v)$ so ist 
		\[ \tilde{\s} = -d\tilde N \circ (d\tilde X)^{-1} = -\left( d_{(u,v)}N \circ d(u,v) \right) \circ \left( d_{(u,v)X} \circ d(u,v)  \right)^{-1} \]
		\[ = -d_{(u,v)}N \circ d(u,v) \circ d(u,v)^{-1} \circ \left( d_{(u,v)}X \right)^{-1}
		= \s\big |_{(u,v)}, \]
		also insbesondere
		\[ \tilde H = H \circ (u,v), \quad \text{ und }\quad
		\tilde K = K \circ (u,v), \quad \text{ etc.} \]
		
		\item Ist $\tilde X=\tilde O+A(X-O)$ mit $A \in SO(3)$ so bekommt man 
		\[ \tilde{\s}= -d\tilde N \circ \left( d\tilde X \right)^{-1}
		= -A\circ dN \circ (dX)^{-1} \circ A^{-1}
		= A\circ \s \circ A^{-1}, \]
		insbesondere also
		\[ \tilde H = H, \quad \text{ und }\quad
		\tilde K = K, \quad \text{ etc.} \]
		Die Krümmungsrichtungen werden mit der Fläche gedreht:
		\[ \ker (\mathrm{id} \kappa^\pm -\tilde{\s}) = A\ker (\mathrm{id} \kappa^\pm-\s). \]
	\end{itemize}
\end{remark}	
	
\begin{definition}
	
	Ein Punkt $X(u,v)$ einer Fläche heißt 
	\begin{itemize}
		
		\item \emph{Nabelpunkt} (umbilic), falls $\kappa^+(u,v) = \kappa^-(u,v) \quad (\Longleftrightarrow (H^2- K)(u,v)=0), $
		\item \emph{Flachpunkt} (flatpoint), falls $\kappa^+(u,v) = \kappa^-(u,v)=0.$
		
	\end{itemize}

	
\end{definition}

\begin{remark}
	
	Ein Punkt $X(u,v)$ ist Nabelpunkt bzw. Flachpunkt, falls
	\[ \mathcal{S}\big|_{(u,v)} = H(u,v)\mathrm{id}_{T_{(u,v)}X} \text{ bzw. } \mathcal{S}\big|{(u,v)}=0. \]
	
\end{remark}

\begin{example}
	
	Falls $  (u,v) \mapsto X(u,v) $ Werte in einer (festen) Ebene
		\[ \pi = \{ X \in \E^3 ~|~ \skal{X-0,n} = d \} \] 
	annimmt, d.h., $ \skal{dX,n}=0 $. Also ist $ N \equiv \pm n $, weshalb $\s \equiv 0$ und jeder Punkt der Fläche ist ein Flachpunkt.\\
	Umgekehrt: Ist jeder Punkt einer Fläche $ X $ ein Flachpunkt, $ \s \equiv 0 $, so folgt $ N \equiv \mathrm{const.} $, also nimmt $ X $ Werte in einer (festen) Ebene an: Mit $ O \in \E^3 $ beliebig \[ 0 = \skal{dX,N} = d\skal{X-O,N} - \skal{X-O,dN} \Rightarrow \mathrm{const.} \equiv \skal{X-O,N}, \] wobei $ \skal{dX,N} $ eine Abbildung $ ((u,v), \pmat{x \\ y}) \mapsto \skal{d_{(u,v)}X\pmat{x\\y},N(u,v)} $ ist.

	\textbf{Matrixdarstellung}: Mit $(X_u,X_v)$ als tangentiales Basisfeld kann der Weingartentensor als Matrix geschrieben werden:
	$ (\s X_u, \s X_v) = (-N_u, -N_v) = (X_u,X_v) \pmat{s_{11} & s_{12} \\ s_{21} & s_{22}} $
	
	also, vermittels der Skalarprodukte der Basisvektoren $ X_u,X_v $, 
	\[\pmat{-\skal{X_u,N_u} & - \skal{X_u,N_v} \\ - \skal{X_v,N_u} & - \skal{X_v,N_v} } = \pmat{E &F \\ F &G} \pmat{s_{11} & s_{12} \\ s_{21} & s_{22}}, \] d.h., als die Gram Matrix der zweiten Fundamentalform:
	
	
\end{example}

\begin{lemma, definition}
	 Die \emph{zweite Fundamentalform} einer Fläche $ X: M \rightarrow \E^3 $ ist definiert als \[ \mathrm{II} := \skal{dX,dN}, \] wobei $ \mathrm{II}: ((u,v), \pmat{x \\ y},\pmat{\widetilde{x} \\ \widetilde{y}}) \mapsto \skal{d_{(u,v)}X \pmat{x\\y},d_{(u,v)}N \pmat{\widetilde{x}\\\widetilde{y}}}; $
	 an jedem Punkt $ (u,v) $ erhält man so eine symmetrische Bilinearform 
	 
	 \[ \mathrm{II}\big|_{(u,v)}: \R^2 \times \R^2 \rightarrow \R, \quad (\pmat{x_1 \\ y_1},\pmat{x_2 \\ y_2}) \mapsto - \skal{d_{(u,v)}X\pmat{x_1 \\ y_1},d_{(u,v)}N\pmat{x_2 \\ y_2}}. \]
	 
\end{lemma, definition}

\begin{proof}
	Dass $ \mathrm{II}\big|_{(u,v)} $ Bilinearform ist, ist klar. Weiters gilt mit der Leibniz-Regel: 
		\[ {\color{ForestGreen} \mathrm{II}(\pmat{1 \\ 0},\pmat{0 \\ 1}) = }-\skal{X_u,N_v}= \skal{X_{uv},N}= \skal{X_{vu},N} {=} -\skal{X_v,N_u} {\color{ForestGreen} = \mathrm{II}(\pmat{0 \\ 1},\pmat{1 \\ 0}) }, \] 
	also ist $ \mathrm{II} $ symmetrisch an jedem Punkt. 
\end{proof}

\begin{remark}
	
	Ist die erste Fundamentalform gegeben, so kann die zweite Fundamentalform aus dem Weingartentensor berechnet werden - und umgekehrt.
	
	\textbf{Warnung:} Obwohl der Weingartentensor symmetrisch ist, ist seine Darstellungsmatrix (bzgl. $ (X_u,X_v) $) normalerweise \textbf{nicht} symmetrisch:
	
	\[ \pmat{s_{11} & s_{12} \\ s_{21} & s_{22}} = \frac{1}{EG-F^2} \pmat{G & -F \\ -F & E} \pmat{e & f \\ f & g} = \frac{1}{EG-F^2} \pmat{Ge-Ff & Gf-Fg \\ Ef-Fe & Eg-Ff} \]
	
\end{remark}

\begin{remark}
	Die Gauß-Krümmung ist gegeben durch: \[ K = \frac{eg-f^2}{EG-F^2} \]
\end{remark}

\subsection{Kovariante Ableitung und Krümmungstensor}

\begin{definition}
	
	Die \emph{kovariante Ableitung} eines Tangentialfeldes $ Y:M\rightarrow V $ entlang $ X:M\rightarrow \E $ ist der Tangentialteil seiner Ableitung 
		\[ \nabla Y := (dY)^T = dY - \skal{dY,N}N, \] wobei $ \nabla $ den \emph{Levi-Civita Zusammenhang} entlang $ X $ bezeichnet.
	
\end{definition}

\begin{remark}
	
	Wegen $ \skal{X_{uu},N}= -\skal{X_u,N_u}= e $ usw. bekommt man 
		\[ \nabla_{\dfrac{\partial}{\partial u}}X_u = X_{uu}-Ne \text{ und } \nabla_{\dfrac{\partial}{\partial v}} = X_{uv}-Nf \]
		\[ \nabla_{\dfrac{\partial}{\partial v}}X_v = X_{vv}-Ng \text{ und } \nabla_{\dfrac{\partial}{\partial u}} = X_{uv}-Nf \]
	
	Wobei $$ \nabla_{\dfrac{\partial}{\partial u}}Y := Y_u - \skal{Y_u,N}N $$
	
\end{remark}

\begin{lemma}
	
	Der L-C Zusammenhang erfüllt die Leibniz-Regel,
		\[ \nabla(Yx) = (\nabla Y)x + Ydx \text{ für } x \in C^\infty(M) \] 
	und ist \emph{metrisch}, d.h., 
		\[ d\skal{Y,Z} = \skal{\nabla Y,Z} + \skal{Y,\nabla Z} \]
	
\end{lemma}

\begin{proof}
	
	Die Leibniz-Regel gilt, da für $x \in C^\infty(M)$:
		\[ \nabla(Yx) = dYx + Ydx- \skal{dYx,N}N - \skal{Ydx,N}N = (\nabla Y)x + Ydx, \]
	weil $ Y $ ein Tangentialfeld ist, ist $ \skal{Ydx,N} = 0. $ 
	
	Er ist metrisch, da:
		\[  \skal{\nabla Y,Z} + \skal{Y,\nabla Z} = \skal{dY,Z} -\skal{dY,N}\skal{N,Z} + \skal{Y,dZ} - \skal{Y,N}\skal{dZ,N} \]
		\[ = \skal{dY,Z} + \skal{Y,dZ} = d\skal{Y,Z}. \]
	
\end{proof}

\textbf{Matrixdarstellung:} Da die kovariante Ableitung eines TVFs $ Y $ tangential ist, erhält man für $ Y = X_v :$

\begin{equation} \label{Eq:Gamma}
\begin{split}
		 \left(\nabla_{\dfrac{\partial}{\partial u}}X_u,\nabla_{\dfrac{\partial}{\partial u}}X_v \right) &= (X_u,X_v)\Gamma_1  \\
		 \left(\nabla_{\dfrac{\partial}{\partial v}}X_u,\nabla_{\dfrac{\partial}{\partial v}}X_v \right) &= (X_u,X_v)\Gamma_2 
\end{split}
\end{equation} 
	mit 
		\[ \Gamma_i = \pmat{\Gamma_{i1}^1 & \Gamma_{i2}^1 \\ \Gamma_{i1}^2 & \Gamma_{i2}^2}. \]
		
	Also gilt für eine allgemeine TVF $ Y=X_ux+X_vy=dX\pmat{x \\ y} $, 
		\[ \nabla_{\dfrac{\partial}{\partial u}}Y= \nabla_{\dfrac{\partial}{\partial u}}\left ( \left( X_u,X_v \right) \pmat{x \\y}    \right ) = \left( \nabla_{\dfrac{\partial}{\partial u}}X_u,\nabla_{\dfrac{\partial}{\partial u}}X_v \right) \pmat{x \\y} + X_u,X_v \pmat{x_u \\ y_u} \]

		\[ = \left( X_u, X_v \right) \Gamma_1\pmat{x \\ y} + \left( X_u, X_v \right) \pmat{x_u \\ y_u} \] 
		
		\begin{equation} \label{Eq:Gammaallg}
		\begin{split}
			&= \left( X_u, X_v \right) \left( \dfrac{\partial}{\partial u} +\Gamma_1 \right) \pmat{x \\ y}  \quad \text{ und ebenso }
		 \\ 
		 \nabla_{\dfrac{\partial}{\partial v}}Y &= \left( X_u, X_v \right) \left( \dfrac{\partial}{\partial v} +\Gamma_2 \right) \pmat{x \\ y}, 
		 \end{split}
		\end{equation} 
	oder, anders gesagt:
		\[ \nabla_{\dfrac{\partial}{\partial u}} \circ dX = dX \circ \left( \dfrac{\partial}{\partial u} + \Gamma_1 \right)  \]
	und genauso
		\[  \nabla_{\dfrac{\partial}{\partial v}} \circ dX = dX \circ \left( \dfrac{\partial}{\partial v} + \Gamma_2 \right).   \]	
	
	Man bemerke: mit $ \pmat{x \\ y} = \pmat{1 \\ 0} \cong \dfrac{\partial}{\partial u} $ erhält man
		\[ \nabla_{\dfrac{\partial}{\partial u}}\left( \left( X_u,X_v \right) \pmat{1 \\ 0} \right) = \nabla_{\dfrac{\partial}{\partial u}}X_u = X_u \Gamma_{11}^1 + X_v\Gamma_{11}^2 = \left( X_u,X_v \right) \Gamma_1\pmat{1 \\ 0}   \] 
	und
		\[ \nabla_{\dfrac{\partial}{\partial v}} \left( \left( X_u,X_v \right) \pmat{1 \\ 0} \right) = \nabla_{\dfrac{\partial}{\partial v}}X_u = \left( X_u,X_v \right) \Gamma_2\pmat{1 \\ 0} \]
	und ebenso für  $ \pmat{x \\ y} = \pmat{0 \\ 1} \cong \dfrac{\partial}{\partial v} $, also ist \ref*{Eq:Gamma} ein Spezialfall von \ref*{Eq:Gammaallg}
	
	\begin{remark}
		\textbf{WARUNUNG:} $ \nabla_{\dfrac{\partial}{\partial u}} $ und $ \nabla_{\dfrac{\partial}{\partial v}} $ sind \emph{Differentialoperatoren} (keine Endomorphismen) - trotzdem benutzen wir Matrizen, um sie zu beschreiben.
	\end{remark}

\begin{lemma, definition}
	
	$\Gamma_{ij}^k$ heißen die \emph{Christoffel-Symbole} von $ X $; sie sind symmetrisch: 
		\[ \Gamma_{ij}^k = \Gamma_{ji}^k \]
	
\end{lemma, definition}

\begin{proof}
	
		\[ X_u\Gamma_{12}^1 + X_v \Gamma_{12}^2 = \nabla_{\dfrac{\partial}{\partial u}}X_v = \left( X_{vu} \right)^T = \left( X_{uv} \right)^T = \nabla_{\dfrac{\partial}{\partial v}}X_u = X_u\Gamma_{21}^1 + X_v\Gamma_{21}^2   \]
	also $ \Gamma_{12}^1 = \Gamma_{21}^1 $ und $ \Gamma_{12}^2 = \Gamma_{21}^2 $.
\end{proof}

\begin{theorem}[Koszul's Formeln]
	
	Mit Matrizen 
		\[ I = \pmat{E &f \\ F &G} \text{ und } J = \pmat{0 &-1 \\ 1 & 0 } \]
	gilt: 
		\[ \frac12 I_u - \frac{E_v - F_u}{2}J = I\Gamma_1, \]
		\[ \frac12 I_v + \frac{G_u - F_v}{2}J = I\Gamma_2. \]
		
\end{theorem}

\begin{proof}
	
	Multipliziere $ \nabla_{\dfrac{\partial}{\partial u}}X_u = X_u\Gamma_{11}^1 + X_v\Gamma_{11}^2 $ mit $ Y_u $ und $ X_v $, um zu erhalten:
		\[ E \Gamma_{11}^1 + F \Gamma_{11}^2 = \skal{X_u, \nabla_{\dfrac{\partial}{\partial u}}X_u} = \frac12 \skal{X_u,X_u}_u = \frac12 E \]
	
		\[ F\Gamma_{11}^1 + G \Gamma_{11}^2 = \skal{X_v, \nabla_{\dfrac{\partial}{\partial u}}X_u} = \skal{X_v,X_u}_u - \skal{\nabla_{\dfrac{\partial}{\partial u}}X_v,X_u} = \skal{X_v,X_u}_u - \skal{\nabla_{\dfrac{\partial}{\partial v}}X_u,X_u} \]
		\[ = F_u - \frac12 E_v. \]
	
	Insgesamt also
		\[ \pmat{E &F \\F &G} \pmat{\Gamma_{11}^1 & \Gamma_{12}^1 \\ \Gamma_{11}^2 & \Gamma_{12}^2} = \frac12 \pmat{E_u \\ F_u} - \dfrac{E_v - F_u}{2} \pmat{0 \\ 1} \]
		\[ = \left( \frac12 \pmat{E_u & F_u \\ F_u & G_u} - \dfrac{E_v - F_u}{2} \pmat{0 &-1 \\ 1 & 0 } \right).  \]
	
	Die anderen Teile der Gleichung  folgen ebenso.
	
\end{proof}

\begin{corollary}
	Die Christoffel - Symbole $ \Gamma_{ij}^k $ hängen nur von der induzierten Metrik ab.
\end{corollary}

\begin{example}
	
	Für eine isometrische Parametrisierung, $  E = G = 1 $ und $  F = 0 $, erhält man mit den Konszul-Formeln $ \Gamma_{ij}^k = 0 $.
	
\end{example}


\begin{definition}
	
	Für ein TVf $ Y : M \rightarrow V $ entlang $ X: M \rightarrow \E $ definieren wir den \emph{(Riemannschen) Krümmungstensor} $ R $ von $X$ durch 
		\[ RY := \nabla_{\dfrac{\partial}{\partial u}}\nabla_{\dfrac{\partial}{\partial v}}Y - \nabla_{\dfrac{\partial}{\partial v}}\nabla_{\dfrac{\partial}{\partial u}}Y. \]
	
\end{definition}

\begin{remark}
	
	Dies ist eine vereinfachte Form, für Flächen, des ''wahren'' Krümmungstensors.
	
\end{remark}

\begin{lemma}
	
	$ R $ ist ein schiefsymmetrischer Tensor des Tangentialbündels $ TX $, d.h., 
	$R\big|_{(u,v)} \in \mathrm{End}(T_{(u,v)}X) $ ist schiefsymmetrisch für jedes $ (u,v) \in M $, und 
		\[ R(Yx) = (RY)x \text{ für } x \in C^\infty(M). \]
	
\end{lemma}

\begin{proof}
	
	$R\big|_{(u,v)} \in \mathrm{End(T_{(u,v)}X)}$ ist klar.
	Schiefsymmetrie:
		\[\frac12 (\abs{Y}^2)_{uv} = \skal{Y \nabla_{\dfrac{\partial}{\partial u}}Y}_v = \skal{Y, \nabla_{\dfrac{\partial}{\partial v}}\nabla_{\dfrac{\partial}{\partial u}}Y} + \skal{\nabla_{\dfrac{\partial}{\partial v}}Y,\nabla_{\dfrac{\partial}{\partial u}}Y} \]
		\[ = \skal{Y, \nabla_{\dfrac{\partial}{\partial u}}\nabla_{\dfrac{\partial}{\partial v}}Y} + \skal{\nabla_{\dfrac{\partial}{\partial u}}Y, \nabla_{\dfrac{\partial}{\partial v}}Y} = \frac12 \left( \abs{Y}^2\right)  \]
		\[ \Rightarrow \skal{Y,RY} = \frac12 \left(\abs{Y}^2\right)_{vu}- \frac12\left(\abs{Y}^2\right)_{uv}=0 \]
	Tensoreigenschaft:
		\[ \nabla_{\dfrac{\partial}{\partial u}}\nabla_{\dfrac{\partial}{\partial v}}(Yx)= \nabla_{\dfrac{\partial}{\partial u}}\left(\left( \nabla_{\dfrac{\partial}{\partial v}}Y \right)x + Yx_v\right) \]
		\[ = \left( \nabla_{\dfrac{\partial}{\partial u}} \nabla_{\dfrac{\partial}{\partial v}} Y \right)x + \underbrace{\left( \nabla_{\dfrac{\partial}{\partial v}}Y \right)x_u + \left( \nabla_{\dfrac{\partial}{\partial u}}Y \right)x_v + Yx_{uv}}_{\text{symmetrisch in } \dfrac{\partial}{\partial u} \text{ und } \dfrac{\partial}{\partial v}} \]
	also
		\[ R(Yx)=(RY)x. \]
	
\end{proof}

\pagebreak


\section{Curves on surfaces}
\subsection{Natural ribbon \& special lines on surfaces}

\begin{definition}
	Let $X: \R^2 \supset M \to \E^3$ a surface and $I \ni t \mapsto X(u(t),v(t))$ with a map $(u,v):I \to M$ defines a curve on the surface $X$ as soon as $X \circ (u,v)$ is regular:
		\[ \forall t \in I: (X_uu' + X_vv') (t) \neq 0 \quad \Longleftrightarrow \quad
			\pmat{u'\\v'}(t) \neq 0 \]
	since $d_{(u,v)} X: \R^2 \to \R^3$ injects.
\end{definition}

\begin{example}
	The parameter lines of a surface are the curves
		\[ t \mapsto X(u,v+t), t \mapsto X(u+t,v). \]
\end{example}

\begin{remark, definition}
	If $t \mapsto X(u(t),v(t))$ is a curve on a surface $X: M \to \E^3$ than $T_t(X \circ (u,v)) \subset T_{(u(t),v(t))}X$
	or equivalently, the unit tangent field is always tangential to the surface
		\[ T= \frac {X_uu' + X_vv'}{\sqrt{Eu'^2 + 2Fu'v' + Gv'^2}}. \]
	Thus the Gauss map $N$ of $X$ yields a unit normal vectorfield for $X_0(u,v)$ 
		\[ I \ni t \mapsto N(u(t),v(t)). \]
	Hence this defines the \emph{natural ribbon} of the curve. The corresponding frame is called the \emph{Darboux frame}.
\end{remark, definition}

\begin{definition}
	A curve $t \mapsto X(u(t),v(t))$ on a surface $X: M \to \E^3$ is called
	\begin{itemize}
		\item a \emph{curvature line} if its natural ribbon is a curvature ribbon, i.e., $\tau = 0$,
		\item  an \emph{asymptotic line} if its natural ribbon is an asymtotic ribbon, i.e., $\kappa_n = 0$.
		\item  an \emph{per-geodesic line} if its natural ribbon is an geodesic ribbon, i.e., $\kappa_g = 0$.
	\end{itemize}
\end{definition}

\begin{remark}
	A curve is a curvature line iff the Gauss map of $X$ is parallel along the curve.
\end{remark}

\begin{theorem}[Joachimsthal's theorem]
	Suppose two surfaces intersect along a curve that is a curvature line for one of the two surfaces. Then it is a curvature line for the other surface iff the two surfaces intersect at a constant angle.
\end{theorem}

\begin{proof}
	Exercise.
\end{proof}

\begin{definition}
	Rodriges' equation: The curve $t \mapsto X(u(t),v(t))$ is a curvature line iff 
		\[ 0 = (dN + \kappa  dX) \pmat{u'\\v'} \]
	where $\kappa$ is a principle curvature of $X$ at $(u,v)= (u(t),v(t))$ and $dX \pmat{u'\\v'}$ is the corresponding curve direction.
\end{definition}

\begin{proof}
	The structure equation of the natural ribbon yield
		\[ \nabla^\perp (N \circ (u,v)) = (N \circ (u,v))' - \skal{N \circ (u,v)',T}T
			= N_uu' + N_vv'+ \kappa_n\circ (u,v)(X_uu' + X_vv') \]
		\[ = (dN + \kappa_n dX)\pmat{u'\\v'}. \]
	Therefore $t \mapsto (X,N)(u(t),v(t))$ is a curvature ribbon iff $(dN + \kappa_n dX) \pmat{u'\\v'}=0$.
	On the other hand $dN = -\s \circ dX$. Therefore
		\[ (dN + \kappa_n dX)\pmat{u'\\v'} = (-\s + \kappa_n id ) \circ dX \pmat{u'\\v'} =0 \]
	iff $\kappa_n$ is a principle curvature and $dX \pmat{u'\\v'}$ is the corresponding curve direction. 
\end{proof}

\begin{example}
	Let $X$ be a surface of revolution with Gauss map $N$ (sec 2.2)
		\[ X(u,v)= O + e_1 r(u) \cos v + e_2 r(u) \sin v + e_3 h(u). \]
	and 
		\[ N(u,v) = -e_1 h'(u) \cos v - e_2 h'(u) \sin v + e_3 r'(u) \]
	we deduce
		\[ N_u || X_u \ and \ N_v || X_v \]
	Hence the parameter line of $X$ are curvature lines.
\end{example}

\begin{theorem, definition}
	$X: M \to \E^3$ is a \emph{curvature line parametrisation} if all parameter lines are curvature lines. Any surface admits locally away form umbilics, a curvature line (re-)parametrisation.
\end{theorem, definition}

\begin{remark}
	Suppose $X$ is a curvature line parametrisation then $(X_u,X_v)$ diagonalizes the shape operator, cause these are the Eigenvalues,
		\[ \s X_u || X_u \quad \s X_v || X_v. \]
	Hence, as $\s$ is symmetric, $X_u \perp X_v$ and $N_u = -\s X_u \perp X_v$, or equivalently, $F = f=0$ where
		\[ I = Edu^2 + 2Fdudv + Gdv^2 \]
	and
		\[ II = edu^2 + 2fdudv + gdv^2. \]
	Conversely, if $f=F=0$, then $X$ is a curvature line parametrisation. Look at the matrix representation of $\s$.
\end{remark}

\begin{lemma}
	The normal curvature of a curve $t \mapsto X(u(t),v(t))$ on a surface is given by 
		\[ \kappa_n = \frac {II(\pmat{u'\\v'},\pmat{u'\\v'})}{I (\pmat{u'\\v'},\pmat{u'\\v'})}. \]
\end{lemma}

\begin{proof}
	The normal curvature of a ribbon $(X,N)$ is given by
		\[ \kappa_n = \frac 1{\abs{X'}} \skal{T',N} = \frac 1{\abs{X'}^2} \skal{X'',N} = -\frac 1{\abs{X'}^2} \skal{X',N'}. \]
	Applying the chain rule yields the result
		\[ X' = X_uu' + X_vv', \quad N' = N_uu' + N_vv'. \]
\end{proof}

\begin{remark, definition}
	The normal curvature $\kappa_n$ for a curve on a surface depends only on the tangent direction (and not on $u'', v''$). Thus we also call it the ''normal curvature $\kappa_n$ of a tangent direction''.
\end{remark, definition}

\begin{theorem}[Euler's theorem]
	The normal curvatures $\kappa_n$ at a point on a surface satisfy 
		\[ \min \{ \kappa^+,\kappa^- \} \leq \kappa_n (\theta) = \kappa^+ \cos^2 \theta + \kappa^- \sin^2\theta \leq \max \{\kappa^+,\kappa^- \},  \]
	where $\kappa^\pm$ are the principle curvatures and $\theta$ is the angle between the tangent direction of $\kappa_n(\theta)$ and the curvature direction of $\kappa^+$.
\end{theorem}

\begin{proof}
	Exercise.
\end{proof}

\begin{corollary}
	The principle curvatures can be characterised as the extremal values of the normal curvature at a point on a surface.
\end{corollary}

\begin{corollary}
	If $t \mapsto (\pmat{u'\\v'},\pmat{u'\\v'})$ is an asymptotic line, i.e. $\kappa_n = 0$, of $X$ iff 
		\[ eu'^2 + 2fu'v' + gv'^2 = 0. \]
\end{corollary}

\begin{example}
	The helicoid
		\[ X(u,v) = O + e_1 \sinh u \cos v + e_2 \sinh u \sin v + e_3 v. \]
	Then
		\[ II = -2dudv. \]
	Hence the parameter lines of $X$ are asymptotic lines 
		\[ (II(\pmat{1\\0}, \pmat{1\\0})) = II(\pmat{0\\1}, \pmat{0\\1}) = 0 \]
		\[ t \mapsto X(u,t) = O + e_1r \cos t + e_2 r \sin t + e_3 t, \]
	where $r=\sinh u$.
\end{example}

\begin{lemma}
	Fix a point $X(u,v)$ on a parametrised surface that an asymptotic line passes through $X(u,v)$  in two, one or no directions, depending on the sign of the Gauss curvature $K(u,v)$
\end{lemma}

\subsection{Geodesic and exponential map}

\begin{definition}
	The \emph{covariant derivative} of a tangent field $Y: I \to R^3$ along a curve $t \mapsto X(u(t),v(t))$ on a surface $X:M \to \E^3$ is the tangential part of its derivative 
		\[ \frac {D}{dt} Y : = Y' - N \skal{Y',N}.  \]
	A geodesic is an acceleration free curve $t \mapsto C(t)=X(u(t),v(t))$ on a surface ,i.e,
		\[ \frac D{dt} C' = 0 \]
\end{definition}

\begin{example}
	Circular helices as geodesies a circular cylinders
		\[ t \mapsto C(t)= O+ e_1 r \cos t + e_2 r\sin t + e_3 h t = X(ht,t)  \]
	is a geodesic on the cylinder of radius $r>0$, $h\in \R$ constant.
	
		\[ C'(t) = -e_1 r\sin t + e_2 r \cos t +e_3 h \]
		\[ C''(t)= -e_1 r\cos t - e_2 r\sin t \perp X_u(ht,t), X_v(ht,t) \]
	Therefore
		\[ \frac D{dt} C' = 0 \]
\end{example}

\begin{theorem}
	
	Geodesics are the constant speed pre-geodesic lines ($ \kappa_g \equiv 0 $)
	
\end{theorem}

\begin{proof}
	Firstly, every geodesic has constant speed by the Leibniz' rule.
	
		\[ \skal{C',C'}' = 2\skal{C'',C'}= 2\skal{\dfrac{D}{dt}C',C'} \equiv 0. \]
		
	Secondly, assume $ |C'|= \mathrm{const.} $, then
		\[ \dfrac{C''}{|C'|^2} = \dfrac{T'}{|C'|}= \dfrac{1}{|C'|}(|C'|\kappa_nN - |C'| \kappa_gB) || N \Leftrightarrow \kappa_g = 0 \Leftrightarrow C \text{ is pre-geodesic line. } \]
		
\end{proof}

\begin{theorem}[Clairaut's theorem]
	For a geodesic on a surface of revolution the quantity $ r\sin(\theta) \equiv \mathrm{const.} $ where $ r = r(s) $ is the distance from the axis and $ \theta(s) $ is the angle the geodesic makes with the profile curves.
	
\end{theorem}

\begin{proof}Consider the surface of revolution:
		$$ X(u,v) = O +  e_1r(u)\cos(v) + e_2r(u)\sin(v) + e_3h(u); $$
	$$ N(u,v) = -e_1g'(u)\cos(v) - e_2h'(u)\sin(v) +e_3r'(u) $$
	$ C(s) = X(u(s),v(s))$ be a geodesic on a surface of revolution , w.l.o.g., arc length parametrized.
	Set $ C_t(s) = O + A(t)(C(s)-O) = X(u(s),v(s)+t) $ where $ A(t) $ is given in matrix form by
		\[ A(t) = \pmat{\cos(t) & -\sin(t) & 0 \\ \sin(t) & \cos(t) & 0 \\ 0 & 0 & 1 } \in \mathrm{SO}(3). \]
	Note that $ \forall t \in \R, C_t$ is an arclength parametrized geodesic and
			\[ \dfrac{\partial}{\partial t} C_t(s) = \dfrac{\partial}{\partial t}X(u(s),v(s)+t) = X_v(u(s),v(s)+t \] and the normal of the normal ribbon for $ C_t $ is $ s \mapsto N(u(s),v(s)+t $.
	Set $ Y(s) := \dfrac{\partial}{\partial t}\big|_{t=0}C_t(s) = X_v(u(s),v(s)). $
	\[|Y(s)| = \widetilde{r}(u(s)) = r(s). \] 
	%Observe that, $ Y(s) = (-e_1 \sin(v(s)) + e_2\cos(v(s)))r(s) $.
	Therefore the angle $ \theta = \theta(s) $ between $ C $ and the profile curve satisfies
		\[ r\sin(\phi) = r \cos(\dfrac{\pi}{2} - \theta) = r\dfrac{\skal{C',Y}}{|C'||Y|}= \skal{C',Y} = \skal{\dfrac{\partial}{\partial s}C_t,\dfrac{\partial}{\partial t}C_t}\bigg|_{t=0}. \]
	We want to show $ \dfrac{\partial}{\partial s} \skal{ \dfrac{\partial}{\partial s} C_t , \dfrac{\partial}{\partial t} C_t } = 0: $
		\[ \dfrac{\partial}{\partial s} \skal{ \dfrac{\partial}{\partial s} C_t , \dfrac{\partial}{\partial t} C_t } = \skal{ \dfrac{\partial^2}{\partial^2 s} C_t , \dfrac{\partial}{\partial t} C_t} + \skal{ \dfrac{\partial}{\partial s} C_t , \dfrac{\partial}{\partial s} \dfrac{\partial}{\partial t} C_t } \]
%	Because $ \dfrac{D}{\partial s} C_t \equiv 0 $ we obtain $ \dfrac{\partial^2}{\partial^2s}C_t || N $ and $ \dfrac{\partial}{\partial t}C_t \perp N. $ Hence   
	$ C_t $ is geodesic and thus $ \dfrac{\partial^2}{\partial^2s} C_t || N(u(s),v(s)+t) $ and $ \dfrac{\partial}{\partial t} C_t(s)=X_v(u(s),v(s)+t). $ 
	Hence
		\[ \skal{ \dfrac{\partial^2}{\partial^2 s} C_t , \dfrac{\partial}{\partial t} C_t } = 0 \] 
	and furthermore
		\[ \dfrac{\partial}{\partial s} \skal{ \dfrac{\partial}{\partial s} C_t , \dfrac{\partial}{\partial t} C_t } = 0 \text{ fot all $t \in \R$} \] 
	and
		\[ \dfrac{\partial}{\partial s}(r \sin(\theta)) = 0. \]
\end{proof}

\begin{remark}
	
	The proof can be generalized for surfaces invariant with respect to 1-parameter families of isometries.
	
\end{remark}

\begin{remark, example}
	Clairaut's theorem only provides a necessary condition for a geodesic, not a sufficient one.
	For example: one sheeted hyperboloid
		\[ (u,v) \mapsto O + e_1\cosh(u) \cos(v) + e_2 \cosh(u) \sin(v) + e_3 \sinh(u) \]
	Straight line $ C(t) = O + e_1 + (e_2 +e_3)t $ is a geodesic in $ X $
		\[r \sin (\theta) = \skal{\dfrac{C'}{|C'|},Y} = \dfrac{\cosh(u) \cos(v)}{\sqrt{2}}= \dfrac{1}{\sqrt{2}}. \]
	$ Y(s) = (-e_1\sin(v(s)) + e_2\cos(v(s)))\cosh(u) $
	On the other hand, every circle of latitude in $ X $ satisfies $ r\sin(\theta) \equiv \cosh(u) \equiv \mathrm{const.} $
	but in general these are not geodesic.	
\end{remark, example}

\textbf{Differential equations of a geodesic:}
	
	Let $ Y(t) = X_u(u(t),v(t))x(t) + X_v(u(t),v(t))y(t) $ be a tangent field along a curve $ t \mapsto C(t) = X(u(t),v(t)). $
	Compute the covariant derivative
	\begin{align*}
		\dfrac{D}{\partial t} Y 
			&= X_ux' + (\nabla_{\!\!{\partial\over \partial u}} X_u u' + \nabla_{\!\!{\partial\over \partial v}} X_u v')x + X_vy' +  (\nabla_{\!\!{\partial\over \partial u}} X_v u' + \nabla_{\!\!{\partial\over \partial v}} X_v v')y \\[2mm]
		 &= X_ux' + (X_u\Gamma_{11}^1 u' + X_v\Gamma_{11}^2u' + X_u \Gamma_{21}^1v' + X_v \Gamma_{21}^2v')x \\
		 &\phantom{=} +  X_vy' + (X_u\Gamma_{12}^1 u' + X_v\Gamma_{12}^2u' + X_u \Gamma_{22}^1v' + X_v \Gamma_{22}^1v')y \\[2mm]
		 &= X_u(x' + (\Gamma_{11}^1u' + \Gamma_{21}^1v')x + (\Gamma_{12}^1 u' + \Gamma_{22}^2v')y) + X_v(y' + (\Gamma_{11}^2u' + \Gamma_{21}^2v')x + (\Gamma_{12}^2 u' + \Gamma_{22}^2v')y)
	\end{align*}
	Now, let $ Y = C' = X_u u' + X_v v' $, we get
		\[ \dfrac{D}{\partial t} C' = X_u(u'' + \Gamma_{11}^1u'^2 + 2 \Gamma_{12}^1 u'v' +\Gamma_{22}^1 v'^2) + X_v(v'' + \Gamma_{11}^2u'^2 + 2\Gamma_{12}^2u'v' + \Gamma_{22}^2v'^2). \]
	Then we learn that:
	
\begin{definition}
	\emph{Geodesic Equation:} \label{Eq: geodesic}
		$ t \mapsto C(t) = X(u(t),v(t)) $ is a geodesic if and only if
			\[ 0 = u'' + \Gamma_{11}^1u'^2 + 2 \Gamma_{12}^1 u'v' +\Gamma_{22}^1 v'^2 \]
		and
		\[ 0 = v'' + \Gamma_{11}^2u'^2 + 2\Gamma_{12}^2u'v' + \Gamma_{22}^2v'^2. \]
		
		We can also write as
		\begin{align*}
			0&= u'' +(u',v')\Gamma^1\pmat{u'\\v'}\\[2mm]
			0&= v'' +(u',v')\Gamma^2\pmat{u'\\v'}
		\end{align*}
		with
			\[ \Gamma^i = \pmat{\Gamma_{11}^i & \Gamma_{12}^i\\
								\Gamma_{21}^i & \Gamma_{22}^i}. \]
		
\end{definition}

\begin{remark}
	
	For a geodesic curve $ t \mapsto C(t) = X(u(t),v(t)) $ we can compute the geodesic curvature of the Darboux frame
		\[ \dfrac{d}{\partial t} \dfrac{C'}{|C'|} = - B |C'|\kappa_g. \]
	Take the cross product of this with $ T = \dfrac{C'}{|C'|} $
		\[ - N |C'|\kappa_g = \dfrac{D}{\partial t}\left(\dfrac{C'}{|C'|}\right) \times \dfrac{C'}{|C'|} = \dfrac{1}{|C'|^2} \dfrac{D}{\partial t}C' \times C' \] and thus
		\[ N\kappa_g = - \dfrac{1}{|C'|^3}\dfrac{D}{\partial t} C' \times C'. \]
		Comparing $ X_u \times X_v $ terms
		
		\[ \kappa_g = \frac {\sqrt{EG-F^2}}{\sqrt{Eu'^2+2Fu'v'+Gv'^2}} \det \pmat{
			u' & u'' + \Gamma_{11}^1 u'^2 + 2\Gamma_{12}^1u'v' + \Gamma_{22}^1v'^2\\
			v' & v'' + \Gamma_{11}^2 u'^2 + 2\Gamma_{12}^2u'v' + \Gamma_{22}^2v'^2}.
		\]
		
	
\end{remark}

\begin{corollary}
	Geodesics can be determined by the induced metric $ \mathrm{I} $ alone.
\end{corollary}

\begin{example}
	
	Geodesics on a circular cylinder are the straight lines after developing onto a plane: circular helices.
	I.e. \ref*{Eq: geodesic} holds for exactly those curves.
	
\end{example}

\begin{corollary}
	
	Given a point $ (u_0,v_0) \in M $ and a direction $ Y = d_{(u_0,v_0)}X(\pmat{x_0 \\ y_0})  $
	There exists a unique (maximal) geodesic $ t \mapsto C_Y(t) = X(u(t),v(t)) $ on $ X: M \rightarrow \E^3 $ such that
		\[ (u(0),v(0)) = (u_0,v_0) \text{ and } (u'(0),v'(0)) = (x_0,y_0) \label{Eq:initial conditions}. \]
	
\end{corollary}

\begin{remark}
	
	The initial conditions \ref*{Eq:initial conditions} say that an initial point and a tangential direction are given on the surface, if $ X(u_0,v_0) $ is a double point \ref*{Eq:initial conditions} also specifies which leaf of the surface $ C_Y(t) $ lives on.
	
\end{remark}

\begin{proof}
	
	We are going to use Picard-Lindelöf.
	Let $ (w_1,w_2,w_3,w_4) = (u,v,u',v') $. Thus we have the system
		\[ w_1' = w_3 \]
		\[ w_2' = w_4 \]
		And because of \ref*{Eq: geodesic}
		\[ w_3' = -\Gamma_{11}^1w_3^2 - 2 \Gamma_{12}^1 w_3w_4 -\Gamma_{22}^1 w_4^2 \]
		\[ w_4' = -\Gamma_{11}^2w_3^2 - 2 \Gamma_{12}^2 w_3w_4 -\Gamma_{22}^2 w_4^2. \]
		So, the initial conditions \ref*{Eq:initial conditions} imply that $ (w_1,w_2,w_3,w_4)(0) = (u_0,v_0,x_0,y_0) $ and we can use Picard-Lindelöf theorem by which the result follows.
	
\end{proof}

\begin{lemma}
	
	$ C_{Ys}(t) = C_Y(st) $ for $ s \in (0,1) $
	\label{lemma:I}
\end{lemma}

\begin{proof}
	
	Suppose $ C_Y: I \rightarrow \E^3 $ is the geodesic satisfying \ref*{Eq:initial conditions}, then (for an interval around $ 0 $)
		\[ \dfrac{D}{\partial t} (C_Y(st)' = \dfrac{D}{\partial t} C_Y'(st)s = (\dfrac{D}{\partial t} C') (st)s^2 = 0 \] 
	and also
		\[ (C_Y(st))'(0) = C_Y'(s0)s = Ys \]
	while
		\[ C_Y(S0) = C_Y(0). \]
	By the uniqueness, $ C_{Ys}(t) = C_Y(st) $ for $ t \in I $
\end{proof}

\begin{remark}
	By the smooth dependence of solutions $C_Y$ of the initial value problem, we obtain a smooth map 
		\[ R \times T_{(u_0,v_0)}X \ni (t,Y) \mapsto C_Y(t) \in \E^3, \]
	which is defined on an open neighbourhood $I \times U$ of $(0,0) \in R \times T_{(u_0,v_0)}X$ with star shaped $U$ and , w.l.o.g., $I \supset [0,1]$.
	
	Consider all unit tangent vectors $Y \in T_{(u_0,v_0)}X$. Then by Picard-Lindelöf theorem there exists a $\varepsilon_Y>0$ such that $C_Y$ is defined on $(-\varepsilon_Y, \varepsilon_Y)$.
	Let $Y_{min}$ be the direction for witch $\varepsilon_{Y_{min}}$ is the smallest possible $\varepsilon_Y$. 
	
	If $\varepsilon_{Y_{min}}<1$, then $C_{Y\frac {\varepsilon_min}2} = C_Y(\frac {\varepsilon_min}2 t)$ is defined on $[0,1]$, for all $Y$. let $ U \subset B_{\frac {\varepsilon_min}2}(0)$.
	\end{remark}

\begin{lemma, definition}
	Given a point $X(u_0,v_0)$ on a surface $X:M \to \E^3$
		\[ Y \mapsto \exp(Y):= C_Y(1) \]
		
	defines a smooth map on an open neighbourhood $U$ of $O \in T_{(u_0,v_0)}X$ with 		
		\[d_0\exp = id_{T_{(u_0,v_0)}X},\]
	with $d_0= \frac d{dt} \big |_{t=0} $.
	$\exp$ is called the exponential map of $X$ at $X(u_0,v_0)$.
\end{lemma, definition}

\begin{proof}
	$\exp$ is a smooth dependence of solutions of $IVP$s. Now we compute $d_0\exp$ using directional derivatives. Let $Y \in T_{(u_0,v_0)}X$
		\[ d_0\exp(Y) 
			= \frac d{dt} \big |_{t=0} \exp(tY) 
			= \frac d{dt}\big |_{t=0} C_{Yt}(1) 
			= \frac d{dt}\big |_{t=0} C_Y(t) 
			= Y. \]
	Therefore $d_0\exp = id_{T_{(u_0,v_0)}X}$.
\end{proof}

\begin{remark}
	Thus $\exp : T_{(u_0,v_0)}X \supset U \to X(M)$ yields a local diffeomorphism and in particular a reparametrisation of $X$ around $X(u_0,v_0)$.  
\end{remark}

\subsection{Geodesic polar coordinates and Minding's Theorem}

\begin{definition}
	A reparametrisation of a surface by geodesic polar coordinates $(r,\theta)$ around a point $X(0,0)$ of a surface is given by the map
		\[ (r,\theta) \mapsto \exp(e_1 r \cos\theta + e_2 r \sin \theta)
			=C_{e_1 r \cos\theta + e_2 r \sin \theta}(1), \]
	where $(e_1,e_2)$ form an orthonormal basis of $T_{(0,0)}X$.
	
	For fixed $\theta$, $r \mapsto X(r,\theta) = C_{e_1 r \cos\theta + e_2 r \sin \theta}(1) = C_{e_1 \cos\theta + e_2 \sin \theta}(r)$ and $r \leq1$.
	So parameter lines $\theta = const$ are geodesic.
	
	We let $r \leq 1$ in contrast to the Lemma \ref{lemma:I} cause we expect $[0,1] \subset I$.
\end{definition}

\begin{remark}
	This parametrisation is regular at $r=0$, however it is regular on $(0,\varepsilon) \times \R$ for some $\varepsilon>0$.
\end{remark}

\begin{lemma}
	In geodesic polar coordinates $(r,\theta)$ the induced metric is given by 
		\[ \mathrm I = dr^2 + Gd\theta^2  \]
	where 
		\[ \sqrt{G}\big |_{r=0} = 0 \text{ and }\frac {\partial \sqrt G}{\partial r} \big |_{r=0} = 1. \]
\end{lemma}

\begin{proof}
	This proof is technical, see the notes.
\end{proof}

\begin{example}
	In geodesic polar coordinates $(r,\theta)$ the Gauss curvature is 
		\[ K= - \frac {(\sqrt{G})_{rr}}{\sqrt G}. \]
\end{example}

\begin{corollary}
	Geodesics are locally the shortest distance between two points.
\end{corollary}

\begin{proof}
	Let $X: M \to \E^3$ parametrised by geodesic polar coordinates around $X(0,0)$. Let $c(t)= X(r(t),\theta(t))$ be a curve between $X(0,0)$ and $X(r(0),\theta(1))$. Then 
		\[ \int_0^1 \abs{C'(\tilde t)} \ d\tilde t
			= \int_0^1 \sqrt{r'^2 + G(r,\theta)\theta'^2}\ d\tilde t
			\geq \int_0^1 r'\ d\tilde t
			= r(1). \]
			
	Equality holds iff $\theta'=0$ or $\theta = const$.
	
	Therefore $C$ is a parameter line of geodesic polar coordinates and thus geodesic (up to reparametrisation of the function $r$).
\end{proof}

\begin{remark}
	The surface $\R^2 \setminus \{(0,0)\}$ is an example of why we need locality in the corollary above. Cause there is no shortest distance on that surface between $(1,0)$ and $(-1,0)$.
\end{remark}

\begin{theorem}[Minding's theorem]
	Any two surfaces with the same \emph{constant} Gauss curvature are locally isometric, i.e., there exists local parametrisations $X_1$ and $X_2$ such that $I_1=I_2$. 
\end{theorem}

\begin{proof}
	For surface $X: M \to \E^3$ parametrised by geodesic polar coordinates around $X(0,0)$ we have
		\[ I=dr^2+Gd\theta^2 \text{ with }
			\sqrt G \big|_{r=0} \text{ and } \frac {\partial \sqrt G}{\partial r} \big|_{r=0}=1\]
	and $K=-\frac {(\sqrt G)_{rr}}{\sqrt G}$.
	Therefore $\sqrt G_{rr}+K\sqrt G = 0$ and $K$ is constant.
	Hence we have an initial value problem (for fixed $\theta$)
		\[ (\sqrt G)_{rr} + K\sqrt G=0, \quad \sqrt G\big|_{r=0}=0, \ \frac {\partial \sqrt G}{\partial r} \big|_{r=0}=1.  \]
	The unique solution is
		\[ \sqrt G= \begin{cases}
			\frac 1{\sqrt K} \sin(\sqrt K r), & K>0\\
			r, & K=0\\
			\frac 1{\sqrt{-K}} \sinh(\sqrt{-K}r), & K<0.
		\end{cases} \]
	Thus $G$ is determined by $K$ and thus so is $I$. Thus any two surfaces with the same constant Gauss curvature have the same induced metric.

\end{proof}
\pagebreak

\section{Manifolds}
\textbf{Motivation:} Some problems occur with our definition of curves and surfaces:
\begin{itemize}
	\item The sphere is not a surface because no regular parametrisation  (harry (Potter? WTF Jojo?) ball theorem)
	\item The hyperbola is no a surface, because it has two components thus it can not be parametrised by a single regular map on an open interval.
\end{itemize}

The notion of a submanifold resolves these problems at the expense of imposing other restrictions.

\subsection{Submanifolds of $\E^n$}
There are several equivalent characterisations of submanifolds in $\E^n$.

\begin{definition}[1. A submanifols can be locally flattened]
	$M \subset \E^n$ is called a $k$-dimensional submanifold of $\E^n$ if for all $p \in M$ there exists a diffeomorphism $\phi: U \to \tilde U$, where $U \subset \E^n$ is an open neighbourhood of $p$ and $\tilde U \subset \R^n$ is an open neighbourhood of $0$ such that
		\[ \phi(M \cap U)= \tilde U \cap (\R^k \times \{0\}), \]
	where $\R^n = \R^k \oplus \R^{n-k}$.
\end{definition}

\begin{definition}[2. A submanifold is locally a level set]
	$M \subset \E^n$ is a $k$-dimensional submanifold of $\E^n$ if for all $p \in M$ there exist open neighbourhood $U \subset \E^n$ of $p$ and a submersion $F. U \to \R^{n-k}$ such that
		\[ M \cap U = F^{-1}\{0\}. \]
	Where $dpF: \R^n \to \R^{n-k}$ surjects for all $p \in U$.
\end{definition}

\begin{remark}
	In the definition above th is sufficient to require that $dpF: \R^n \to \R^{n-k}$ surjects: if $dpF$ surjects then since $p \mapsto dpF$ is continuous, $dpF$ surjects by the inertia principle on some open neighbourhood $\tilde U \subset U$ of $p$.
\end{remark}

\begin{definition}[3. A submanifold can be locally parametrised]
	$M \subset \E^n$ is a $k$-dimensional submanifold of $\E^n$ if for all $p \in M$ there exists an immersion $X: V \to U$ from an open neighbourhood $V \subset \R^k$ of $0$ to an open neighbourhood $U \subset \E^n$ of $p$ such that
		\[ M \cap U = X(V) \]
	and $X:V \to M \cap U$ is a homeomorphism (with respect to the induced topology on $M \cap U$).
	
	A homeomorphism is continuous and bijective.
\end{definition}

\begin{remark} We get the following exlusions:	
	\begin{itemize}
		\item $X$ being an immersion excludes ''kinks'' such as the singularity of the nilparabola.
		\item $X$ being injective excludes self intersections.
		\item Continuity of $X^{-1}$ excludes ''T-junctions''.
	\end{itemize}
\end{remark}

\begin{proof}
	Proof of equivalence of these definitions:
	
	For $\R^n = \R^k \oplus \R^{n-k}$ we define the submersions
		\[ \pi_1: \R^k \oplus \R^{n-k} \to \R^k, (x,y) \mapsto x, \]
		\[ \pi_2: \R^k \oplus \R^{n-k} \to \R^k, (x,y) \mapsto y. \]
	First we proof $1.$ implies $2.$:
	
	Let $F:= \pi_2 \circ \phi: U \to \R^{n-k}$. $F$ is a submersion.
	
	Secondly we proof $1.$ implies $3.$:
	
	With $V= \pi_1(\tilde U) \subset \R^k$ we can have
		\[ X:= \phi^{-1}\big|_v: V \to U \]
	is the necessary map.
	If you are bored, you can check that this is an homeomorphism.
	
	Now we proof $3.$ implies $1.$:
	
	Let $X: \R^k \supset V \to \E^n$ parametrisation of $M \cap U = X(V)$. Assume that $X(0)= p$. Let $(t_1, \ldots,t_{n-k})$ be an orthonormal basis of $d_0X(\R^k)^\perp \subset \R^n$. Define
		\[ C \times \R^{n-k}: (x,y) \mapsto \psi(x,y)= X(x)+ \sum_{i=1}^{n-k} t_iy_i, \quad y=(y_1,\ldots,y_{n-k}). \]
	Then 
		\[ d_0\psi(v,w) = \underbrace{d_0X(v)}_{\in d_0X(\R^k)} + \underbrace{\sum_{i=1}^{n-k} t_i w_i}_{\in (d_0X(\R^k))^\perp} = 0 \]
	iff $w_i=0$ for all $i$ and $v=0$ or $(v,w)=0$. 
	
	Then we use the inverse mapping theorem, $\psi$ has a smooth local inverse
		\[ \phi=(\psi\big|_{\tilde U})^{-1}: \psi(\tilde U) \to \tilde U \]
	where $\tilde U \subset V \times \R^k$ open neighbourhood of $0$. Without loss of generality, assume that $\psi(\tilde U) \subset U$ (otherwise take the intersection with U).
	Now, $q \in M \cap \psi(\tilde U)$ implies there exists a $x \in V$ such that $q= X(x)= \psi(x,0) \in \psi(\tilde U)$. On the other hand 
		\[ (x,0) \in \tilde U \Rightarrow \psi(x,0) = X(x) \in M \]
	with means that $q=X(x) \in M \cap \psi(\tilde U)$.
	
	After replacing $\psi(\tilde U)$ with $U$, then $\phi(U \cap M)= \tilde U \cap (\R^k \times \{0\})$.
	
	2. $\Rightarrow$ 1. $ F: U \rightarrow \R^{n-k} $ submersion. Let $ t_1, \dots ,t_n $ be an orthonormal basis of $ \R^n $ such that $ t_1,\dots,t_k $ is an orthonormal basis of $\mathrm{ker} d_pF$. Write $  \R^{n} = \skal{e_1,\dots,e_k} \oplus \R^{n-k} $ then $ q \in U \Rightarrow q = p + \sum_{i = 1}^{n}t_iq_i $ and 
		\[ \phi: U \rightarrow \R^n \quad, \quad \phi(q) = \sum_{i = 1}^{k} e_iq_i + F(q) \]
		\[d_p\phi (v) = \sum_{i=1}^{k}e_iv_i + d_pF(v) = 0 \text{ for } v = \sum_{i = 1}^{k}t_iv_i\] is equivalent to
		$d_pF(v) = 0$ and thus $ v \in \mathrm{ker}d_pF $
		and also to
		\[ \sum_{i=1}^{k}e_iv_i = 0 \text{ thus } \forall i v_i = 0 \Leftrightarrow v = 0 \] 
		Thus $ d_p\phi $ is invertable and by the Inverse Mapping Theorem
			\[ \phi : U \rightarrow \phi(U) \text{ is a diffeomorphism (maybe after shrinking $U$)}. \]
		Now, $ q \in M \cap U \Leftrightarrow F(q) = 0 \Leftrightarrow \phi(q) \in \phi(U) \cap ( \R^k \times \{0\} ) $.
		Thus $ \phi(M \cap U) = \phi(U) \cap (R^k \times \{0\}) $
\end{proof}  

\begin{example}
	\begin{enumerate} The following are manifolds:
		\item Plane: $ \pi = \{ p \in \E^3 : \skal{p-O,n} = d \}, $
		$ F : \E^3 \rightarrow \R, F(q) = \skal{p-O,n}-d $ then $ \pi = F^{-1}(\{0\}) $.
		Also, $ d_pF(v) = \skal{v,n} $ and hence $ d_pF \not \equiv 0 $. Thus $ d_pF $ surjects and $F$ is a submersion.
		
		\item Sphere: $ S = \{ p \in \E^3 ~|~ \skal{p-O,p-O} = r^2 \} $ and $ S $ is implicitly given by the function $ F : \E^3 \rightarrow \R, F(p) = \skal{p-O,p-O}-r^2 $. Here $ d_pF(v) = 2 \skal{v,p-O} $ and $ d_pF $ surjects as long as $p \not \equiv 0$ which does not happen in $ S $.
		$ F\big|_{\E^3 \setminus \{ 0 \} } $ is a submersion.
		
		\item  Hyperboloids: $O + e_1x + e_2y + e_3z$ such that
	 $ F_\pm(O  + e_1x + e_2y + e_3z) = 0 $ where  
	 	\[F_\pm(O  + e_1x + e_2y + e_3z) = \left(\dfrac{x}{a}\right)^2 + \left(\dfrac{y}{b}\right)^2 +\left(\dfrac{z}{c}\right)^2 \pm 1 \]
	 and
	 	\[ \nabla F = 2\left(e_1 \dfrac{x}{a^2} + e_2 \dfrac{y}{b^2} + e_3 \dfrac{z}{c^2}\right). \]
	 Then $ \nabla F = 0 \Leftrightarrow (x,y,z) = 0 $. Therefore $ F\big|_{\E^3 \setminus \{0\}} $ is a submersion with $\skal{\nabla F,v}= dF(v)$.
	\end{enumerate}
\end{example}

\begin{example}
	A counterexample is the following \emph{Lemniscate}.
	$ O + e_1x + e_2y $, where $ x^4 - x^2 + y^2 = 0 $. A regular parametrization is given by $ t \mapsto O + e_1 x(t) + e_2 y(t) = O + e_1 \sin(t) + e_2\sin(t)\cos(t). $ 
	The curve has a self-intersection at $ (x(t),y(t))=(0,0) $ which is equivalent to $ \forall k \in \mathbb{Z} t = k\pi $. Hence this is not a $ 1$-Dimensional submanifold.
\end{example}

\begin{definition}
	The tangent space of a $k$-dimensional submanifold $M \subset \E^n$ of $p \in M$ is the $k$-dimensional subspace
		\[ T_pM = d_oX(\R^k) \subset \R^n \]
	where $X: \R^k \supset V  \to \E^n$ is a parametrisation of $M$ around $p=X(o)$.
\end{definition}

\begin{remark}
	$T_pM$ is independent of the parametrisation $X$.
	
	Let $\tilde X = X_o\mu$ around $p = \tilde X (\tilde o)$ where $\mu: \tilde V \to V $ is a diffeomorphism with $\mu(\tilde o)=o$. Then
		\[ d_o\tilde X = d_{\tilde o}(X \circ \mu) = (d_oX)\circ d_{\tilde o}\mu. \]
	Therefore
		\[ d_{\tilde o}\tilde X(\R^k) = (d_oX)(d_{\tilde o}\mu(\R^k)) = d_oX(\R^k). \]
\end{remark}

\begin{remark}
	If we have $M=F^{-1}(\{0\})$ is defined as the level set of a submersion $F: U \to \R^{n-k}$ then 
		\[ T_pM = \ker d_pF. \]
		
	This follows from the following: 
	$F \circ X=0$ for any parametrisation $X:V \to \E^n$ around $p=X(o)$. Then the chain rule gives us
		\[ 0 = d_o(F \circ X) = (d_oF)\circ d_oX. \]
		\[ 0= (d_pF)\circ (d_oX(\R^k)) = (d_pF)(T_pM). \]
	Therefore $T_pM \subset \ker d_pF$. But since $F$ is a submersion $\dim (\ker d_pF)=k$ and $\dim(T_pM)=k$.
\end{remark}

\begin{example}
	Lets consider a set of orthogonal maps 
		\[ O(3)=\{ A \in GL(3): AA^T = id \}
		= \{ A \in GL(3): F(A)=0 \} \]
	where $F: GL(3) \to Sym(3), F(A) = AA^T-id$. $ Sym(3) $ is a $6$-dimensional vector space.
	
		\[ d_AF: End(\R^3) \to Sym(3), d_AF(B) = BA^T + AB^T. \]
	This surjects since any element of $Sym(3)$ can be written as $Y+Y^T$ so let $B=YA$. Hence $F$ is a submersion and $O(3)$ is a $3$-dimensional submanifold of $End(\R^3) \approx M_{3 \times 3}$.
		\[ d_AF(B) = 0 \quad \Longleftrightarrow \quad
			BA^T + AB^T=0 \quad\Longleftrightarrow\quad
			BA^T \in \square(3), \]
	where $\square(3)$ is the skew-symmetric-endomorphisms. Therefore
		\[ T_AO(3) = \{ B \in GL(3)= End(\R^3) | BA^T \in \square(3) \} \]
	with is $3$-dimensional and
		\[ T_ASO(3) = T_AO(3), \quad A \in SO(3). \]
	
	This is an example of a Lie-group
\end{example}

\begin{exercise}
	Think about GL($n$) and $SL(3)$.
\end{exercise}

Fun-fact:
	\[ End(\R^n) = Sym(n) \oplus \square(n). \]
	
\subsection{Functions on submanifolds}

Perviosly functions, vector fields etc. were defined on open sets of affine spaces, where notions of differentiability makes sense.
Now we want to consider functions on domains that are submanifolds of $\E^n$ To do this we define derivatives so that the chain rule holds.

\begin{definition}
	A function $\phi: M \to \E$ on a submanifold $M \subset \R^n$ is said to be differentiable at $p \in M$ with derivative
		\[ d_p\phi := d_0(\phi \circ X) \circ (d_oX)^{-1} : T_pM \to \R  \]
	uf $\phi_oX : \R^k \supset V \to \E$ is differentiable at $o$ for some local parametrisation $X: V \to M$ of $M$ around $p$ with $X(o)=p$.
\end{definition}

\begin{remark}
	This definition makes sense as it does not depend on our choice of parametrisation $X$.
	
	Let $\tilde X= X \circ \psi$ is a reparametrisation at $p \in M$ for some diffeomorphism $\psi$ then $\phi \circ\tilde X = \phi  \circ  X\circ\psi$ is differentiable as soon as $\phi \circ X$ is differentiable. Moreover, if we assume $\psi(o)=o$ then 
		\[ d_o(\phi \circ \tilde X) \circ (d_o\tilde X)^{-1}
			= d_o(\phi \circ X \circ \psi) \circ (d_o(X \circ \psi))^{-1}
			 \]
		\[ = d_o(\phi \circ X) \circ d_o\psi \circ (d_o\psi)^{-1} \circ (d_oX)^{-1}
		= d_o(\phi \circ X)  \circ (d_oX)^{-1}. \]
	This definition can be easily generalised to $\E^n$-valued maps and thus to maps between submanifolds.
\end{remark}

\begin{remark}
	Suppose that $\Phi: \E^n \to \E$ is differentiable and $M$ is a submanifold of $\E^n$. Thus $\phi:= \Phi \big|_M :M \to \E$ is differentiable with
		\[ d_p\phi = d_p\Phi \big|_{T_pM} : T_pM \to \R, \quad p \in M. \] 
	
	Let $X: V \to M$ be a parametrisation of $M$ around $p=X(o)$, then $\phi \circ X = \Phi \circ X$ is differentiable and for $\xi = d_oX(x)$ then
		\[ d_p\phi(\xi) = d_o(\phi \circ X) \circ (d_oX)^{-1}(\xi)
			= d_0(\Phi \circ X)(x) = (d_{X(o)} \Phi) \circ d_oX(x)
			= d_p\Phi(\xi). \]  
\end{remark}

\begin{definition}
	Let $\phi: M \to \E$ be differentiable. Then the gradient of $\phi$ at $p \in M$ is the unique vector field $\grad \phi(p) \in T_pM$ with
		\[ d_p\phi(\xi) = \skal{\xi, \grad \phi(p)}, \quad \forall \xi \in T_pM. \]
	Since  $d_p\phi : T_pM \to \R $ , $d_p \phi \in (T_pM)^*$, with Riesz-Fischer $\exists!$ vector $v \in T_pM$ such that $d_p\phi= \skal{.,v}$
\end{definition}

\begin{example}
	Suppose $\R^2 \supset V \in (u,v) \mapsto X(u,v) \in \E^3$ is a parametrised surface with $I= Edu^2 + 2Fdudv + Gdv^2$. Let
		\[ X_u^*:= \frac 1{EG-F^2}(GX_u-FX_v), \quad
			X_v^*:= \frac 1{EG-F^2} (-FX_u +EX_v) \in T_{(u,v)}X. \]
	Cause
		\[ \skal{X_u^*,X_u} = \skal{X_v^*,X_v}=1, \quad
			\skal{X_u^*,X_v} = \skal{X_v^*,X_u} =0 \]
	$X_u^*,X_v^*$ is a dual basis. 
	
	Now let $\phi : M= X(V) \to \E$ be a differentiable function and define $\psi := \phi \circ X$. Then $\xi = d_{(u,v)} X(x)  \in T_{X(u,v)}M$ and
		\[ \skal{\grad \phi(X(u,v)), \xi} = d_{X(u,v)}\phi(\xi) = (d_{(u,v)}\psi)(x). \]
	Thus 
		\[\skal{ \grad \phi \circ X, X_u} = \psi_u, \quad \skal{ \grad \phi \circ X, X_v} = \psi_v.\]
		
	Hence 
		\[ \grad \phi \circ X = X_u^*\psi_u + X_v^*\psi_v
			= \frac {G\psi_u - F\psi_v}{EG-F^2}X_u + \frac {E\psi_v - F\psi_u}{EG-F^2}X_v, \]
	where
		\[ \grad \phi \circ X = V \to TM. \]
\end{example}
	


\pagebreak

\end{document}
