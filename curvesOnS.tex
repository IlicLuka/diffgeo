\section{Natural ribbon \& special lines on surfaces}

\begin{definition}
	Let $X: \R^2 \supset M \to \E^3$ a surface and $I \ni t \mapsto X(u(t),v(t))$ with a map $(u,v):I \to M$ defines a curve on the surface $X$ as soon as $X \circ (u,v)$ is regular:
		\[ \forall t \in I: (X_uu' + X_vv') (t) \neq 0 \quad \Longleftrightarrow \quad
			\pmat{u'\\v'}(t) \neq 0 \]
	since $d_{(u,v)} X: \R^2 \to \R^3$ injects.
\end{definition}

\begin{example}
	The parameter lines of a surface are the curves
		\[ t \mapsto X(u,v+t), t \mapsto X(u+t,v) \]
\end{example}

\begin{remark} %and Def
	If $t \mapsto X(u(t),v(t))$ is a curve on a surface $X: M \to \E^3$ than $T_t(X \circ (u,v)) \subset T_{(u(t),v(t))}X$
	or equivalently, the unit tangent field is always tangential to the surface
		\[ T= \frac {X_uu' + X_vv'}{\sqrt{Eu'^2 + 2Fu'v' + Gv'^2}}. \]
	Thus the Gauss map $N$ of $X$ yields a unit normal vectorfield for $X_0(u,v)$ 
		\[ I \ni t \mapsto N(u(t),v(t)) \]
	Hence this defines the \emph{natural ribbon} or the curve. The corresponding frame is called the \emph{Darboux frame}.
\end{remark}

\begin{definition}
	A curve $t \mapsto X(u(t),v(t))$ on a surface $X: M \to \E^3$ is called
	\begin{itemize}
		\item a \emph{curvature line} if its natural ribbon is a curvature ribbon, i.e., $\tau = 0$,
		\item  an \emph{asymtotic line} if its natural ribbon is an asymtotic ribbon, i.e., $\kappa_n = 0$.
		\item  an \emph{per-geodesic line} if its natural ribbon is an geodesic ribbon, i.e., $\kappa_g = 0$.
	\end{itemize}
\end{definition}

\begin{remark}
	A curve is a curvature line iff the Gauss map of $X$ is parallel along the curve.
\end{remark}

\begin{theorem}[Joachimsthal's theorem]
	Suppose two surfaces intersect along a curve thet is a curvature line for one of the two surfaces. Then it is a curvature line for the other surface iff the two surfaces intersect at a constant angle.
\end{theorem}

\begin{proof}
	Exercise.
\end{proof}

\begin{definition}
	Rodriges' equation: The curve $t \mapsto X(u(t),v(t))$ is a curvature line iff 
		\[ 0 = (dN + \kappa  dX) \pmat{u'\\v'} \]
	where $\kappa$ is a principle curvature od $X$ at $(u,v)= (u(t),v(t))$ and $dX \pmat{u'\\v'}$ is the corresponding curve direction.
\end{definition}

\begin{proof}
	The structure equation of the natural ribbon yield
		\[ \nabla^\perp (N \circ (u,v)) = (N \circ (u,v))' - \skal{N \circ (u,v)',T}T
			= N_uu' + N_vv'+ \kappa_n\circ (u,v)(X_uu' + X_vv') \]
		\[ = (dN + \kappa_n dX)\pmat{u'\\v'}. \]
	Therefore $t \mapsto (X,N)(u(t),v(t))$ is a curvature ribbon iff $(dN + \kappa_n dX) \pmat{u'\\v'}=0$.
	On the other hand $dN = -\s \circ dX$. Therefore
		\[ (dN + \kappa_n dX)\pmat{u'\\v'} = (-\s + \kappa_n id ) \circ dX \pmat{u'\\v'} =0 \]
	iff $\kappa_n$ is a principle curvature and $dX \pmat{u'\\v'}$ is the corresponding curve direction. 
\end{proof}

\begin{example}
	Let $X$ be a surface of revolution with Gauss map $N$ (sec 2.2)
		\[ X(u,v)= O + e_1 r(u) \cos v + e_2 r(u) \sin v + e_3 h(u). \]
	and 
		\[ N(u,v) = -e_1 h'(u) \cos v - e_2 h'(u) \sin v + e_3 r'(u) \]
	we deduce
		\[ N_u || X_u \ and \ N_v || X_v \]
	Hence the parameter line of $X$ are curvature lines.
\end{example}

\begin{definition} (and thm)
	$X: M \to \E^3$ is a \emph{curvature line parametrisation} if all parameter lines are curvature lines. Any surface admits locally away form unbilics, a curvature line (rep-)parametrisation.
\end{definition}

\begin{remark}
	Suppose $X$ is a curvature line parametrisation then $(X_u,X_v)$ diagonalizes the shape operator, cause these are the Eigenvalues,
		\[ \s X_u || X_u \quad \s X_v || X_v. \]
	Hence, as $\s$ is symetric, $X_u \perp X_v$ and $N_u = -\s X_u \perp X_v$, or equivalently, $F = f=0$ where
		\[ I = Edu^2 + 2Fdudv + Gdv^2 \]
	and
		\[ II = edu^2 + 2fdudv + gdv^2. \]
	Conversely, if $f=F=0$, then $X$ is a curvature line parametrisation. Look at the matrix representation of $\s$.
\end{remark}

\begin{lemma}
	The normal curvature of a curve $t \mapsto X(u(t),v(t))$ on a surface is given by 
		\[ \kappa_n = \frac {II(\pmat{u'\\v'},\pmat{u'\\v'})}{I (\pmat{u'\\v'},\pmat{u'\\v'})}. \]
\end{lemma}

\begin{proof}
	The normal curvature of a ribbon $(X,N)$ is given by
		\[ \kappa_n = \frac 1{\abs{X'}} \skal{T',N} = \frac 1{\abs{X'}^2} \skal{X'',N} = -\frac 1{\abs{X'}^2} \skal{X',N'}. \]
	Applying the chain rule yields the result
		\[ X' = X_uu' + X_vv', \quad N' = N_uu' + N_vv'. \]
\end{proof}

\begin{remark}(and def)
	The normal curvature $\kappa_n$ for a curve on a surface depends only on tthe tangent direction and (not on $u'', v''$). Thus we also call it the '' normal curvature $\kappa_n$ of a tangent direction''.
\end{remark}

\begin{theorem}(Euler's theorem)
	The normal curvatures $\kappa_n$ at a point on a surface satisfy 
		\[ \min \{ \kappa^+,\kappa^- \} \leq \kappa_n (\theta) = \kappa^+ \cos^2 \theta + \kappa^- \sin^2\theta \leq \max \{\kappa^+,\kappa^- \},  \]
	where $\kappa^\pm$ are the principle curvatures and $\theta$ is the angle between the tangent direction of $\kappa_n(\theta)$ and the curvature direction of $\kappa^+$.
\end{theorem}

\begin{proof}
	Exercise.
\end{proof}

\begin{corollary}
	The principle curvatures can be characterised as the extremal values of the normal curvature at a point on a surface.
\end{corollary}

\begin{corollary}
	If $t \mapsto (\pmat{u'\\v'},\pmat{u'\\v'})$ is an asymtotic line, i.e. $\kappa_n = 0$, of $X$ iff 
		\[ eu'^2 + 2fu'v' + gv'^2 = 0. \]
\end{corollary}

\begin{example}
	The helicoid
		\[ X(u,v) = O + e_1 \sinh u \cos v + e_2 \sinh u \sin v + e_3 v. \]
	Then
		\[ II = -2dudv. \]
	Hence the parameter line of $X$ are asymtotic lines 
		\[ (II(\pmat{1\\0}, \pmat{1\\0})) = II(\pmat{0\\1}, \pmat{0\\1}) = 0 \]
		\[ t \mapsto X(u,t) = O + e_1r \cos t ü e_2 r \sin t + e_3 t, \]
	where $r=\sinh u$.
\end{example}

\begin{lemma}
	Fix a point $X(u,v)$ on a parametrised surface that an asymptotic line pas through $X(u,v)$  in two, one or no directions, depending on the sign of the Gaus curvature $K(u,v)$
\end{lemma}

\section{Geodesic and exponential map}

\begin{definition}
	The \emph{covariant derivative} of a tangent fiel $Y: I \to R^3$ along a curve $t \mapsto X(u(t),v(t))$ on a surface $X:M \to \E^3$ is the tangential part of its derivative 
		\[ \frac {D}{dt} Y : = Y' - N \skal{Y',N}.  \]
	A geodesic is an acceleration free curve $t \mapsto C(t)=X(u(t),v(t))$ on a surface ,i.e,
		\[ \frac D{dt} C' = 0 \]
\end{definition}

\begin{example}
	Cicular helices as geodesies a cicular cylinders
		\[ t \mapsto C(t)= O+ e_1 r \cos t e_2 r\sin t + e_3 h t = X(ht,t)  \]
	is a geodesic on the cylinder of radius $r>0$, $h\in \R$ constant.
	
		\[ C'(t) = -e_1 r\sin t + e_2 r \cos t +e_3 h \]
		\[ C''(t)= -e_1 r\cos t - e_2 r\sin t \perp X_u(ht,t), X_v(ht,t) \]
	Therefore
		\[ \frac D{dt} C' = 0 \]
\end{example}

\begin{theorem}
	
	Geodesics are the constant speed pre-geodesic lines ($ \kappa_g \equiv 0 $)
	
\end{theorem}

\begin{proof}
	Firstly, every geodesic has constant speed by the Leibniz' rule.
	
		\[ \skal{C',C'}' = 2\skal{C'',C'}= 2\skal{\dfrac{D}{dt}C',C'} \equiv 0. \]
		
	Secondly, assume $ |C'|= \mathrm{const.} $, then
		\[ \dfrac{C''}{|C'|^2} = \dfrac{T'}{|C'|}= \dfrac{1}{|C'|}(|C'|\kappa_nN - |C'| \kappa_gB) || N \Leftrightarrow \kappa_g = 0 \Leftrightarrow C \text{ is pre-geodesic line. } \]
		
\end{proof}

\begin{theorem}[Clairaut's theorem]
	For a geodesic on a surface of revolution the product $ r\sin(\theta) \equiv \mathrm{const.} $ where $ r = r(s) $ is the distance from the axis and $ \theta(s) $ is the angle the geodesic makes with the profile curves.
	
\end{theorem}

\begin{proof}
	
	$ C(s) = O + e_1r(s)\cos(v(s)) + e_2r(s)\sin(v(s)) + e_3r(s)$ be a geodesic on a surface of revolution , wlog, arc length parameterized.
	Set $ C_t(s) = O + A(t)(C(s)-O) $ where $ A(t) $ is given in matrix form by
		\[ \pmat{\cos(t) & -\sin(t) & 0 \\ \sin(t) & \cos(t) & 0 \\ 0 & 0 & 1 } \in \mathrm{SO}(3). \]
	Set $ Y(s) := \dfrac{\partial}{\partial t}\big|_{t=0}C_t(s). $ Notem that $ \forall t, C(t)$ is an arclength parameterized geodesic.
	Observe that, $ Y(s) = (-e_1 \sin(v(s)) + e_2\cos(v(s)))r(s) $.
	Thus
		\[ r\sin(\phi) = r \cos(\dfrac{\pi}{2} - \theta) = r\dfrac{\skal{C',Y}}{|C'|||Y}= \skal{C',Y} = \skal{\dfrac{\partial}{\partial s}C_t,\dfrac{\partial}{\partial t}C_t}\big|_{t=0}. \]
	Since each $ C_t(s) $ is arc length parameterized
		\[ \dfrac{\partial}{\partial s} \skal{ \dfrac{\partial}{\partial s} C_t , \dfrac{\partial}{\partial t} C_t } = \skal{ \dfrac{\partial^2}{\partial^2 s} C_t , \dfrac{\partial}{\partial t} C_t} + \skal{ \dfrac{\partial}{\partial s} C_t , \dfrac{\partial}{\partial s} \dfrac{\partial}{\partial t} C_t } \]
	Because $ \dfrac{D}{\partial s} C_t \equiv 0 $ we obtain $ \dfrac{\partial^2}{\partial^2s}C_t || N $ and $ \dfrac{\partial}{\partial t}C_t \perp N. $ thus   
		\[ \dfrac{\partial}{\partial s} \skal{ \dfrac{\partial}{\partial s} C_t , \dfrac{\partial}{\partial t} C_t } = 0 \]
\end{proof}

\begin{remark}
	
	The proof can be generalized for surfaces invariant with respect to 1-parameter families of isometries.
	
\end{remark}

\begin{remark, example}
	Clairaut's theorem only provides a necessary condition for a geodesic, not a sufficient one.
	For example: one sheeted hyperboloid
		\[ (uv) \mapsto O + e_1\cosh(u) \cos(v) + e_2 \cosh(u) \sin(v) + e_3 \sinh(u) \]
	Straight line $ C(t) = O + e_1 + (e_2 +e_3)t $ is a geodesic in $ X $
		\[r \sin (\theta) = \skal{\dfrac{C'}{|C'|},Y} = \dfrac{\cosh(u) \cos(v)}{\sqrt{2}}= \dfrac{1}{\sqrt{2}}. \]
	$ Y(s) = (-e_1\sin(v(s)) + e_2\cos(v(s)))\cosh(u) $
	On the other hand, every circle of latitude in $ X $ satisfies $ r\sin(\theta) \equiv \cosh(u) \equiv \mathrm{const.} $
	but in general these are not geodesic.	
\end{remark, example}





