\textbf{Motivation:} Some problems occur with our definition of curves and surfaces:
\begin{itemize}
	\item The sphere is not a surface because no regular parametrisation  (harry ball theorem)
	\item The hyperbola is no a surface, because it has two components thus it can not be parametrised by a single regular map on an open interval.
\end{itemize}

The notion of a submanifold resolves these problems at the expense of imposing other restrictions.

\section{Submanifolds of $\E^n$}
There are several equivalent characterisations of submanifolds in $\E^n$.

\begin{definition}[1. A submanifols can be locally flattened]
	$M \subset \E^n$ is called a $k$-dimensional submanifold of $\E^n$ if for all $p \in M$ there exists a diffeomorphism $\phi: U \to \tilde U$, where $U \subset \E^n$ is an open neighbourhood of $p$ and $\tilde U \subset \R^n$ is an open neighbourhood of $0$ such that
		\[ \phi(M \cap U)= \tilde U \cap (\R^k \times \{0\}), \]
	where $\R^n = \R^k \oplus \R^{n-k}$.
\end{definition}

\begin{definition}[2. A submanifold is locally a level set]
	$M \subset \E^n$ is a $k$-dimensional submanifold of $\E^n$ if for all $p \in M$ there exist open neighbourhood $U \subset \E^n$ of $p$ and a submersion $F. U \to \R^{n-k}$ such that
		\[ M \cap U = F^{-1}\{0\}. \]
	Where $dpF: \R^n \to \R^{n-k}$ surjects for all $p \in U$.
\end{definition}

\begin{remark}
	In the definition above th is sufficient to require that $dpF: \R^n \to \R^{n-k}$ surjects: if $dpF$ surjects then since $p \mapsto dpF$ is continuous, $dpF$ surjects by the inertia principle on some open neighbourhood $\tilde U \subset U$ of $p$.
\end{remark}

\begin{definition}[3. A submanifold can be locally parametrised]
	$M \subset \E^n$ is a $k$-dimensional submanifold of $\E^n$ if for all $p \in M$ there exists an immersion $X: V \to U$ from an open neighbourhood $V \subset \R^k$ of $0$ to an open neighbourhood $U \subset \E^n$ of $p$ such that
		\[ M \cap U = X(V) \]
	and $X:V \to M \cap U$ is a homeomorphism (with respect to the induced topology on $M \cap U$).
	
	A homeomorphism is continuous and bijective.
\end{definition}

\begin{remark}
	\begin{itemize}
		\item $X$ being an immersion excludes ''kinks'' such as the singularity of the nilparabola.
		\item $X$ being injective excludes self intersections.
		\item Continuity of $X^{-1}$ excludes ''T-junctions''.
	\end{itemize}
\end{remark}

\begin{proof}
	Proof of equivalence of these definitions:
	
	For $\R^n = \R^k \oplus \R^{n-k}$ we define the submersions
		\[ \pi_1: \R^k \oplus \R^{n-k} \to \R^k, (x,y) \mapsto x, \]
		\[ \pi_2: \R^k \oplus \R^{n-k} \to \R^k, (x,y) \mapsto y. \]
	First we proof $1.$ implies $2.$:
	
	Let $F:= \pi_2 \circ \phi: U \to \R^{n-k}$. $F$ is a submersion.
	
	Secondly we proof $1.$ implies $3.$:
	
	With $V= \pi_1(\tilde U) \subset \R^k$ we can have
		\[ X:= \phi^{-1}\big|_v: V \to U \]
	is the necessary map.
	If you are bored, you can check that this is an homeomorphism.
	
	Now we proof $3.$ implies $1.$:
	
	Let $X: \R^k \supset V \to \E^n$ parametrisation of $M \cap U = X(V)$. Assume that $X(0)= p$. Let $(t_1, \ldots,t_{n-k})$ be an orthonormal basis of $d_0X(\R^k)^\perp \subset \R^n$. Define
		\[ C \times \R^{n-k}: (x,y) \mapsto \psi(x,y)= X(x)+ \sum_{i=1}^{n-k} t_iy_i, \quad y=(y_1,\ldots,y_{n-k}). \]
	Then 
		\[ d_0\psi(v,w) = \underbrace{d_0X(v)}_{\in d_0X(\R^k)} + \underbrace{\sum_{i=1}^{n-k} t_i w_i}_{\in (d_0X(\R^k))^\perp} = 0 \]
	iff $w_i=0$ for all $i$ and $v=0$ or $(v,w)=0$. 
	
	Then we use the inverse mapping theorem, $\psi$ has a smooth local inverse
		\[ \phi=(\psi\big|_{\tilde U})^{-1}: \psi(\tilde U) \to \tilde U \]
	where $\tilde U \subset V \times \R^k$ open neighbourhood of $0$. Without loss of generality, assume that $\psi(\tilde U) \subset U$ (otherwise take the intersection with U).
	Now, $q \in M \cap \psi(\tilde U)$ implies there exists a $x \in V$ such that $q= X(x)= \psi(x,0) \in \psi(\tilde U)$. On the other hand 
		\[ (x,0) \in \tilde U \Rightarrow \psi(x,0) = X(x) \in M \]
	with means that $q=X(x) \in M \cap \psi(\tilde U)$.
	
	After replacing $\psi(\tilde U)$ with $U$, then $\phi(U \cap M)= \tilde U \cap (\R^k \times \{0\})$.
\end{proof}