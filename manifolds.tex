\textbf{Motivation:} Some problems occur with our definition of curves and surfaces:
\begin{itemize}
	\item The sphere is not a surface because no regular parametrisation  (harry (Potter? WTF Jojo?) ball theorem)
	\item The hyperbola is no a surface, because it has two components thus it can not be parametrised by a single regular map on an open interval.
\end{itemize}

The notion of a submanifold resolves these problems at the expense of imposing other restrictions.

\subsection{Submanifolds of $\E^n$}
There are several equivalent characterisations of submanifolds in $\E^n$.

\begin{definition}[1. A submanifols can be locally flattened]
	$M \subset \E^n$ is called a $k$-dimensional submanifold of $\E^n$ if for all $p \in M$ there exists a diffeomorphism $\phi: U \to \tilde U$, where $U \subset \E^n$ is an open neighbourhood of $p$ and $\tilde U \subset \R^n$ is an open neighbourhood of $0$ such that
		\[ \phi(M \cap U)= \tilde U \cap (\R^k \times \{0\}), \]
	where $\R^n = \R^k \oplus \R^{n-k}$.
\end{definition}

\begin{definition}[2. A submanifold is locally a level set]
	$M \subset \E^n$ is a $k$-dimensional submanifold of $\E^n$ if for all $p \in M$ there exist open neighbourhood $U \subset \E^n$ of $p$ and a submersion $F. U \to \R^{n-k}$ such that
		\[ M \cap U = F^{-1}\{0\}. \]
	Where $dpF: \R^n \to \R^{n-k}$ surjects for all $p \in U$.
\end{definition}

\begin{remark}
	In the definition above th is sufficient to require that $dpF: \R^n \to \R^{n-k}$ surjects: if $dpF$ surjects then since $p \mapsto dpF$ is continuous, $dpF$ surjects by the inertia principle on some open neighbourhood $\tilde U \subset U$ of $p$.
\end{remark}

\begin{definition}[3. A submanifold can be locally parametrised]
	$M \subset \E^n$ is a $k$-dimensional submanifold of $\E^n$ if for all $p \in M$ there exists an immersion $X: V \to U$ from an open neighbourhood $V \subset \R^k$ of $0$ to an open neighbourhood $U \subset \E^n$ of $p$ such that
		\[ M \cap U = X(V) \]
	and $X:V \to M \cap U$ is a homeomorphism (with respect to the induced topology on $M \cap U$).
	
	A homeomorphism is continuous and bijective.
\end{definition}

\begin{remark} We get the following exlusions:	
	\begin{itemize}
		\item $X$ being an immersion excludes ''kinks'' such as the singularity of the nilparabola.
		\item $X$ being injective excludes self intersections.
		\item Continuity of $X^{-1}$ excludes ''T-junctions''.
	\end{itemize}
\end{remark}

\begin{proof}
	Proof of equivalence of these definitions:
	
	For $\R^n = \R^k \oplus \R^{n-k}$ we define the submersions
		\[ \pi_1: \R^k \oplus \R^{n-k} \to \R^k, (x,y) \mapsto x, \]
		\[ \pi_2: \R^k \oplus \R^{n-k} \to \R^k, (x,y) \mapsto y. \]
	First we proof $1.$ implies $2.$:
	
	Let $F:= \pi_2 \circ \phi: U \to \R^{n-k}$. $F$ is a submersion.
	
	Secondly we proof $1.$ implies $3.$:
	
	With $V= \pi_1(\tilde U) \subset \R^k$ we can have
		\[ X:= \phi^{-1}\big|_v: V \to U \]
	is the necessary map.
	If you are bored, you can check that this is an homeomorphism.
	
	Now we proof $3.$ implies $1.$:
	
	Let $X: \R^k \supset V \to \E^n$ parametrisation of $M \cap U = X(V)$. Assume that $X(0)= p$. Let $(t_1, \ldots,t_{n-k})$ be an orthonormal basis of $d_0X(\R^k)^\perp \subset \R^n$. Define
		\[ C \times \R^{n-k}: (x,y) \mapsto \psi(x,y)= X(x)+ \sum_{i=1}^{n-k} t_iy_i, \quad y=(y_1,\ldots,y_{n-k}). \]
	Then 
		\[ d_0\psi(v,w) = \underbrace{d_0X(v)}_{\in d_0X(\R^k)} + \underbrace{\sum_{i=1}^{n-k} t_i w_i}_{\in (d_0X(\R^k))^\perp} = 0 \]
	iff $w_i=0$ for all $i$ and $v=0$ or $(v,w)=0$. 
	
	Then we use the inverse mapping theorem, $\psi$ has a smooth local inverse
		\[ \phi=(\psi\big|_{\tilde U})^{-1}: \psi(\tilde U) \to \tilde U \]
	where $\tilde U \subset V \times \R^k$ open neighbourhood of $0$. Without loss of generality, assume that $\psi(\tilde U) \subset U$ (otherwise take the intersection with U).
	Now, $q \in M \cap \psi(\tilde U)$ implies there exists a $x \in V$ such that $q= X(x)= \psi(x,0) \in \psi(\tilde U)$. On the other hand 
		\[ (x,0) \in \tilde U \Rightarrow \psi(x,0) = X(x) \in M \]
	with means that $q=X(x) \in M \cap \psi(\tilde U)$.
	
	After replacing $\psi(\tilde U)$ with $U$, then $\phi(U \cap M)= \tilde U \cap (\R^k \times \{0\})$.
	
	2. $\Rightarrow$ 1. $ F: U \rightarrow \R^{n-k} $ submersion. Let $ t_1, \dots ,t_n $ be an orthonormal basis of $ \R^n $ such that $ t_1,\dots,t_k $ is an orthonormal basis of $\mathrm{ker} d_pF$. Write $  \R^{n} = \skal{e_1,\dots,e_k} \oplus \R^{n-k} $ then $ q \in U \Rightarrow q = p + \sum_{i = 1}^{n}t_iq_i $ and 
		\[ \phi: U \rightarrow \R^n \quad, \quad \phi(q) = \sum_{i = 1}^{k} e_iq_i + F(q) \]
		\[d_p\phi (v) = \sum_{i=1}^{k}e_iv_i + d_pF(v) = 0 \text{ for } v = \sum_{i = 1}^{k}t_iv_i\] is equivalent to
		$d_pF(v) = 0$ and thus $ v \in \mathrm{ker}d_pF $
		and also to
		\[ \sum_{i=1}^{k}e_iv_i = 0 \text{ thus } \forall i v_i = 0 \Leftrightarrow v = 0 \] 
		Thus $ d_p\phi $ is invertable and by the Inverse Mapping Theorem
			\[ \phi : U \rightarrow \phi(U) \text{ is a diffeomorphism (maybe after shrinking $U$)}. \]
		Now, $ q \in M \cap U \Leftrightarrow F(q) = 0 \Leftrightarrow \phi(q) \in \phi(U) \cap ( \R^k \times \{0\} ) $.
		Thus $ \phi(M \cap U) = \phi(U) \cap (R^k \times \{0\}) $
\end{proof}  

\begin{example}
	\begin{enumerate} The following are manifolds:
		\item Plane: $ \pi = \{ p \in \E^3 : \skal{p-O,n} = d \}, $
		$ F : \E^3 \rightarrow \R, F(q) = \skal{p-O,n}-d $ then $ \pi = F^{-1}(\{0\}) $.
		Also, $ d_pF(v) = \skal{v,n} $ and hence $ d_pF \not \equiv 0 $. Thus $ d_pF $ surjects and $F$ is a submersion.
		
		\item Sphere: $ S = \{ p \in \E^3 ~|~ \skal{p-O,p-O} = r^2 \} $ and $ S $ is implicitly given by the function $ F : \E^3 \rightarrow \R, F(p) = \skal{p-O,p-O}-r^2 $. Here $ d_pF(v) = 2 \skal{v,p-O} $ and $ d_pF $ surjects as long as $p \not \equiv 0$ which does not happen in $ S $.
		$ F\big|_{\E^3 \setminus \{ 0 \} } $ is a submersion.
		
		\item  Hyperboloids: $O + e_1x + e_2y + e_3z$ such that
	 $ F_\pm(O  + e_1x + e_2y + e_3z) = 0 $ where  
	 	\[F_\pm(O  + e_1x + e_2y + e_3z) = \left(\dfrac{x}{a}\right)^2 + \left(\dfrac{y}{b}\right)^2 +\left(\dfrac{z}{c}\right)^2 \pm 1 \]
	 and
	 	\[ \nabla F = 2\left(e_1 \dfrac{x}{a^2} + e_2 \dfrac{y}{b^2} + e_3 \dfrac{z}{c^2}\right). \]
	 Then $ \nabla F = 0 \Leftrightarrow (x,y,z) = 0 $. Therefore $ F\big|_{\E^3 \setminus \{0\}} $ is a submersion with $\skal{\nabla F,v}= dF(v)$.
	\end{enumerate}
\end{example}

\begin{example}
	A counterexample is the following \emph{Lemniscate}.
	$ O + e_1x + e_2y $, where $ x^4 - x^2 + y^2 = 0 $. A regular parametrization is given by $ t \mapsto O + e_1 x(t) + e_2 y(t) = O + e_1 \sin(t) + e_2\sin(t)\cos(t). $ 
	The curve has a self-intersection at $ (x(t),y(t))=(0,0) $ which is equivalent to $ \forall k \in \mathbb{Z} t = k\pi $. Hence this is not a $ 1$-Dimensional submanifold.
\end{example}

\begin{definition}
	The tangent space of a $k$-dimensional submanifold $M \subset \E^n$ of $p \in M$ is the $k$-dimensional subspace
		\[ T_pM = d_oX(\R^k) \subset \R^n \]
	where $X: \R^k \supset V  \to \E^n$ is a parametrisation of $M$ around $p=X(o)$.
\end{definition}

\begin{remark}
	$T_pM$ is independent of the parametrisation $X$.
	
	Let $\tilde X = X_o\mu$ around $p = \tilde X (\tilde o)$ where $\mu: \tilde V \to V $ is a diffeomorphism with $\mu(\tilde o)=o$. Then
		\[ d_o\tilde X = d_{\tilde o}(X \circ \mu) = (d_oX)\circ d_{\tilde o}\mu. \]
	Therefore
		\[ d_{\tilde o}\tilde X(\R^k) = (d_oX)(d_{\tilde o}\mu(\R^k)) = d_oX(\R^k). \]
\end{remark}

\begin{remark}
	If we have $M=F^{-1}(\{0\})$ is defined as the level set of a submersion $F: U \to \R^{n-k}$ then 
		\[ T_pM = \ker d_pF. \]
		
	This follows from the following: 
	$F \circ X=0$ for any parametrisation $X:V \to \E^n$ around $p=X(o)$. Then the chain rule gives us
		\[ 0 = d_o(F \circ X) = (d_oF)\circ d_oX. \]
		\[ 0= (d_pF)\circ (d_oX(\R^k)) = (d_pF)(T_pM). \]
	Therefore $T_pM \subset \ker d_pF$. But since $F$ is a submersion $\dim (\ker d_pF)=k$ and $\dim(T_pM)=k$.
\end{remark}

\begin{example}
	Lets consider a set of orthogonal maps 
		\[ O(3)=\{ A \in \mathrm{GL}(3): AA^T = id \}
		= \{ A \in \mathrm{GL}(3): F(A)=0 \} \]
	where $F: \mathrm{GL}(3) \to \mathrm{Sym}(3), F(A) = AA^T-id$. $ \mathrm{Sym}(3) $ is a $6$-dimensional vector space.
	
		\[ d_AF: \mathrm{End}(\R^3) \to \mathrm{Sym}(3), d_AF(B) = BA^T + AB^T. \]
	This surjects since any element of $\mathrm{Sym}(3)$ can be written as $Y+Y^T$ so let $B=YA$. Hence $F$ is a submersion and $O(3)$ is a $3$-dimensional submanifold of $\mathrm{End}(\R^3) \approx M_{3 \times 3}$.
		\[ d_AF(B) = 0 \quad \Longleftrightarrow \quad
			BA^T + AB^T=0 \quad\Longleftrightarrow\quad
			BA^T \in \square(3), \]
	where $\square(3)$ is the skew-symmetric-endomorphisms. Therefore
		\[ T_AO(3) = \{ B \in \mathrm{GL}(3)= \mathrm{End}(\R^3) | BA^T \in \square(3) \} \]
	with is $3$-dimensional and
		\[ T_ASO(3) = T_AO(3), \quad A \in SO(3). \]
	
	This is an example of a Lie-group
\end{example}

\begin{exercise}
	Think about $\mathrm{GL}(n)$ and SL$(3)$.
\end{exercise}

Fun-fact:
	\[ \mathrm{End}(\R^n) = \mathrm{Sym}(n) \oplus \square(n). \]
	
\subsection{Functions on submanifolds}

Perviosly functions, vector fields etc. were defined on open sets of affine spaces, where notions of differentiability makes sense.
Now we want to consider functions on domains that are submanifolds of $\E^n$. To do this we define derivatives so that the chain rule holds.

\begin{definition}
	A function $\phi: M \to \E$ on a submanifold $M \subset \R^n$ is said to be differentiable at $p \in M$ with derivative
		\[ d_p\phi := d_0(\phi \circ X) \circ (d_oX)^{-1} : T_pM \to \R  \]
	if $\phi_oX : \R^k \supset V \to \E$ is differentiable at $o$ for some local parametrisation $X: V \to M$ of $M$ around $p$ with $X(o)=p$.
\end{definition}

\begin{remark}
	This definition makes sense as it does not depend on our choice of parametrisation $X$.
	
	Let $\tilde X= X \circ \psi$ is a reparametrisation at $p \in M$ for some diffeomorphism $\psi$ then $\phi \circ\tilde X = \phi  \circ  X\circ\psi$ is differentiable as soon as $\phi \circ X$ is differentiable. Moreover, if we assume $\psi(o)=o$ then 
		\[ d_o(\phi \circ \tilde X) \circ (d_o\tilde X)^{-1}
			= d_o(\phi \circ X \circ \psi) \circ (d_o(X \circ \psi))^{-1}
			 \]
		\[ = d_o(\phi \circ X) \circ d_o\psi \circ (d_o\psi)^{-1} \circ (d_oX)^{-1}
		= d_o(\phi \circ X)  \circ (d_oX)^{-1}. \]
	This definition can be easily generalised to $\E^n$-valued maps and thus to maps between submanifolds.
\end{remark}

\begin{remark}
	Suppose that $\Phi: \E^n \to \E$ is differentiable and $M$ is a submanifold of $\E^n$. Thus $\phi:= \Phi \big|_M :M \to \E$ is differentiable with
		\[ d_p\phi = d_p\Phi \big|_{T_pM} : T_pM \to \R, \quad p \in M. \] 
	
	Let $X: V \to M$ be a parametrisation of $M$ around $p=X(o)$, then $\phi \circ X = \Phi \circ X$ is differentiable and for $\xi = d_oX(x)$ then
		\[ d_p\phi(\xi) = d_o(\phi \circ X) \circ (d_oX)^{-1}(\xi)
			= d_0(\Phi \circ X)(x) = (d_{X(o)} \Phi) \circ d_oX(x)
			= d_p\Phi(\xi). \]  
\end{remark}

\begin{definition}
	Let $\phi: M \to \E$ be differentiable. Then the gradient of $\phi$ at $p \in M$ is the unique vector field $\grad \phi(p) \in T_pM$ with
		\[ d_p\phi(\xi) = \skal{\xi, \grad \phi(p)}, \quad \forall \xi \in T_pM. \]
	Since  $d_p\phi : T_pM \to \R $ , $d_p \phi \in (T_pM)^*$, with Riesz-Fischer $\exists!$ vector $v \in T_pM$ such that $d_p\phi= \skal{.,v}$
\end{definition}

\begin{example}
	Suppose $\R^2 \supset V \in (u,v) \mapsto X(u,v) \in \E^3$ is a parametrised surface with $I= Edu^2 + 2Fdudv + Gdv^2$. Let
		\[ X_u^*:= \frac 1{EG-F^2}(GX_u-FX_v), \quad
			X_v^*:= \frac 1{EG-F^2} (-FX_u +EX_v) \in T_{(u,v)}X. \]
	Cause
		\[ \skal{X_u^*,X_u} = \skal{X_v^*,X_v}=1, \quad
			\skal{X_u^*,X_v} = \skal{X_v^*,X_u} =0 \]
	$X_u^*,X_v^*$ is a dual basis. 
	
	Now let $\phi : M= X(V) \to \E$ be a differentiable function and define $\psi := \phi \circ X$. Then $\xi = d_{(u,v)} X(x)  \in T_{X(u,v)}M$ and
		\[ \skal{\grad \phi(X(u,v)), \xi} = d_{X(u,v)}\phi(\xi) = (d_{(u,v)}\psi)(x). \]
	Thus 
		\[\skal{ \grad \phi \circ X, X_u} = \psi_u, \quad \skal{ \grad \phi \circ X, X_v} = \psi_v.\]
		
	Hence 
		\[ \grad \phi \circ X = X_u^*\psi_u + X_v^*\psi_v
			= \frac {G\psi_u - F\psi_v}{EG-F^2}X_u + \frac {E\psi_v - F\psi_u}{EG-F^2}X_v, \]
	where
		\[ \grad \phi \circ X = V \to TM. \]
\end{example}
	

Now we know how to differentiate functions on submanifolds : we can find analogues of notions such as I \& II, shape operator, covariant derivative.

\begin{definition}
	Let $  \xi $ be a tangential verctor field, i.e., $ \xi : M \rightarrow \R^n $ such that $ \forall p \in M(\xi(p) \in T_pM) $. Let $ \eta \in T_pM $. Then 	
		\[ \nabla_\eta \xi \big|_p = (d_p\xi(\eta))^T = (d_o(\xi_oX)(y))^T \]
	where $ X: V \rightarrow \E^n $ is a local parametrisation of $ M $ at $ p $ with $ p = X(o) $ and $ d_oX(y) = \eta $. As usual $ (\dots)^T $ denotes the tangential partm i.i., the orthogonal projection $ \R^n \rightarrow T_pM $. $ \nabla $ is called the \emph{Levi-Civita connection}.
\end{definition}

\begin{remark}
	In case of parametrised surfaces $ X: V \rightarrow \E^3 $, $ M = X(V) $. Then $ Y:= xi : V \rightarrow \E^3 $ tangential vector field in the sense of chaper 2.
		\[ \nabla_\eta \xi = \nabla_yY \quad d_{(u,v)}X(y) = \eta \quad X(u,v) = p \] 
\end{remark}

This yields a notion of second derivatives on surfaces:

\begin{definition}
	
	The \emph{Hessian} of $ \phi: M \rightarrow \E $ at $ p \in M $
		\[ T_pM \times T_pM \ni (\xi,\eta) \mapsto (\mathrm{hess} \phi)\big|_p(\xi,\eta) = \skal{\eta,\nabla_\eta \mathrm{grad}\phi\big|_p } \]
	is a symmetric tensor.
	
\end{definition}

\begin{proof}
	Exercise/technical.
\end{proof}

\begin{remark}
	
	The Hessian is the covariant derivative of $ d\phi: M \times TM \rightarrow \R $, $ \xi,\eta: M \rightarrow TM $
		\[ \mathrm{hess}\phi(\xi,\eta) = d(d\phi(\eta)(\xi)-d\phi(\nabla_\xi\eta) = ''(\nabla_\xi d\phi)(\eta)'' \]	
\end{remark}

The Hessian depends on the covariant derivative, hence on the induced metric,not just on the differentiable structure on M.

\begin{lemma}[Poincaré Lemma]
	A tangential vector field $ \xi: M \rightarrow \R^n $ has a local potential, i.e., locally $ \xi = \mathrm{grad}\phi $, if and only if	
		\[ (\nu,\eta) \mapsto \skal{\eta, \nabla_\nu \xi} \] is symmetric.
\end{lemma}

\begin{proof}
	
	We saw in the Lemma above that symmetrie of $ (\nu,\eta) \mapsto \skal{\eta,\nabla:\nu \xi} $ is a necessary condition for $ \xi = \mathrm{grad}\phi. $
	For sufficiency, let $ X: V \rightarrow M \cap U $ be a local parametrisation. Let $ \xi_1, \dots, \xi_k : M \rightarrow \R^k$ be a tangential v.f. such that $ \xi_i \circ X = \dfrac{\partial X}{\partial x_i} = dX(\dfrac{\partial}{\partial x_i}) $.
	Then given $ \xi: M \rightarrow \R^n $ we seek a function $ \psi: V\rightarrow \R $ such that $ \dfrac{\partial \psi}{\partial x_i} = \skal{\xi_i, \xi} \circ X $ . Then
		\[ \dfrac{\partial}{\partial x_i}(\skal{\xi_i, \xi} \circ X) = \skal{ \dfrac{\partial}{\partial x_i}(\xi:j \circ X), \xi \circ X}+ \skal{\xi_j \circ X, \dfrac{\partial}{\partial x_i}(\xi \circ X)} \] then by the Leibniz rule:
		\[ =\skal{\D{x_i} (\xi_j \circ X), \xi \circ X} + \skal{\xi_j \circ X,\D{x_i} (\xi \circ X)}
		= \left[ \skal{\D{\xi_i}\xi_j , \xi} + \skal{\xi_j, \D{\xi_i}\xi} \right]\circ X \]
	
	LHS is symmetric as soon as $ \skal{ \xi_j, \nabla_{\xi_i} \xi} $ is.
	(N.B. $ \nabla_{\xi_i}\xi_i $ because $ \nabla $ is ''torsion free''.)
	By Poincaré's Lemma $ \exists \psi: V \rightarrow \R $ such that $ \dfrac{\partial \psi}{\partial x_i}= \skal{\xi_i,\xi} \circ X $.
	Now define $ \phi = \psi \circ X^{-1}: X(v) \rightarrow \R $. Check that $ \xi = \mathrm{grad}\phi $.
\end{proof}