\textbf{Motivation:} Some problems occur with our definition of curves and surfaces:
\begin{itemize}
	\item The sphere is not a surface because no regular parametrisation  (harry (Potter? WTF Jojo?) ball theorem)
	\item The hyperbola is no a surface, because it has two components thus it can not be parametrised by a single regular map on an open interval.
\end{itemize}

The notion of a submanifold resolves these problems at the expense of imposing other restrictions.

\subsection{Submanifolds of $\E^n$}
There are several equivalent characterisations of submanifolds in $\E^n$.

\begin{definition}[1. A submanifols can be locally flattened]
	$M \subset \E^n$ is called a $k$-dimensional submanifold of $\E^n$ if for all $p \in M$ there exists a diffeomorphism $\phi: U \to \tilde U$, where $U \subset \E^n$ is an open neighbourhood of $p$ and $\tilde U \subset \R^n$ is an open neighbourhood of $0$ such that
		\[ \phi(M \cap U)= \tilde U \cap (\R^k \times \{0\}), \]
	where $\R^n = \R^k \oplus \R^{n-k}$.
\end{definition}

\begin{definition}[2. A submanifold is locally a level set]
	$M \subset \E^n$ is a $k$-dimensional submanifold of $\E^n$ if for all $p \in M$ there exist open neighbourhood $U \subset \E^n$ of $p$ and a submersion $F: U \to \R^{n-k}$ such that
		\[ M \cap U = F^{-1}\{0\}. \]
	Where $dpF: \R^n \to \R^{n-k}$ surjects for all $p \in U$.
\end{definition}

\begin{remark}
	In the definition above th is sufficient to require that $dpF: \R^n \to \R^{n-k}$ surjects: if $dpF$ surjects then since $p \mapsto dpF$ is continuous, $dpF$ surjects by the inertia principle on some open neighbourhood $\tilde U \subset U$ of $p$.
\end{remark}

\begin{definition}[3. A submanifold can be locally parametrised]
	$M \subset \E^n$ is a $k$-dimensional submanifold of $\E^n$ if for all $p \in M$ there exists an immersion $X: V \to U$ from an open neighbourhood $V \subset \R^k$ of $0$ to an open neighbourhood $U \subset \E^n$ of $p$ such that
		\[ M \cap U = X(V) \]
	and $X:V \to M \cap U$ is a homeomorphism (with respect to the induced topology on $M \cap U$).
	
	A homeomorphism is continuous and bijective.
\end{definition}

\begin{remark} We get the following exlusions:	
	\begin{itemize}
		\item $X$ being an immersion excludes ''kinks'' such as the singularity of the nilparabola.
		\item $X$ being injective excludes self intersections.
		\item Continuity of $X^{-1}$ excludes ''T-junctions''.
	\end{itemize}
\end{remark}

\begin{proof}
	Proof of equivalence of these definitions:
	
	For $\R^n = \R^k \oplus \R^{n-k}$ we define the submersions
		\[ \pi_1: \R^k \oplus \R^{n-k} \to \R^k, (x,y) \mapsto x, \]
		\[ \pi_2: \R^k \oplus \R^{n-k} \to \R^{n-k}, (x,y) \mapsto y. \]
	First we proof $1.$ implies $2.$:
	
	Let $F:= \pi_2 \circ \phi: U \to \R^{n-k}$. $F$ is a submersion.
	
	Secondly we proof $1.$ implies $3.$:
	
	With $V= \pi_1(\tilde U) \subset \R^k$ we can have
		\[ X:= \phi^{-1}\big|_v: V \to U \]
	is the necessary map.
	If you are bored, you can check that this is an homeomorphism.
	
	Now we proof $3.$ implies $1.$:
	
	Let $X: \R^k \supset V \to \E^n$ parametrisation of $M \cap U = X(V)$. Assume that $X(0)= p$. Let $(t_1, \ldots,t_{n-k})$ be an orthonormal basis of $d_0X(\R^k)^\perp \subset \R^n$. Define
		\[ C \times \R^{n-k}: (x,y) \mapsto \psi(x,y)= X(x)+ \sum_{i=1}^{n-k} t_iy_i, \quad y=(y_1,\ldots,y_{n-k}). \]
	Then 
		\[ d_0\psi(v,w) = \underbrace{d_0X(v)}_{\in d_0X(\R^k)} + \underbrace{\sum_{i=1}^{n-k} t_i w_i}_{\in (d_0X(\R^k))^\perp} = 0 \]
	iff $w_i=0$ for all $i$ and $v=0$ or $(v,w)=0$. 
	
	Then we use the inverse mapping theorem, $\psi$ has a smooth local inverse
		\[ \phi=(\psi\big|_{\tilde U})^{-1}: \psi(\tilde U) \to \tilde U \]
	where $\tilde U \subset V \times \R^k$ open neighbourhood of $0$. Without loss of generality, assume that $\psi(\tilde U) \subset U$ (otherwise take the intersection with U).
	Now, $q \in M \cap \psi(\tilde U)$ implies there exists a $x \in V$ such that $q= X(x)= \psi(x,0) \in \psi(\tilde U)$. On the other hand 
		\[ (x,0) \in \tilde U \Rightarrow \psi(x,0) = X(x) \in M \]
	with means that $q=X(x) \in M \cap \psi(\tilde U)$.
	
	After replacing $\psi(\tilde U)$ with $U$, then $\phi(U \cap M)= \tilde U \cap (\R^k \times \{0\})$.
	
	2. $\Rightarrow$ 1. $ F: U \rightarrow \R^{n-k} $ submersion. Let $ t_1, \dots ,t_n $ be an orthonormal basis of $ \R^n $ such that $ t_1,\dots,t_k $ is an orthonormal basis of $\mathrm{ker} d_pF$. Write $  \R^{n} = \skal{e_1,\dots,e_k} \oplus \R^{n-k} $ then $ q \in U \Rightarrow q = p + \sum_{i = 1}^{n}t_iq_i $ and 
		\[ \phi: U \rightarrow \R^n \quad, \quad \phi(q) = \sum_{i = 1}^{k} e_iq_i + F(q) \]
		\[d_p\phi (v) = \sum_{i=1}^{k}e_iv_i + d_pF(v) = 0 \text{ for } v = \sum_{i = 1}^{k}t_iv_i\] is equivalent to
		$d_pF(v) = 0$ and thus $ v \in \mathrm{ker}d_pF $
		and also to
		\[ \sum_{i=1}^{k}e_iv_i = 0 \text{ thus } \forall i v_i = 0 \Leftrightarrow v = 0 \] 
		Thus $ d_p\phi $ is invertable and by the Inverse Mapping Theorem
			\[ \phi : U \rightarrow \phi(U) \text{ is a diffeomorphism (maybe after shrinking $U$)}. \]
		Now, $ q \in M \cap U \Leftrightarrow F(q) = 0 \Leftrightarrow \phi(q) \in \phi(U) \cap ( \R^k \times \{0\} ) $.
		Thus $ \phi(M \cap U) = \phi(U) \cap (R^k \times \{0\}) $
\end{proof}  

\begin{example}
	\begin{enumerate} The following are manifolds:
		\item Plane: $ \pi = \{ p \in \E^3 : \skal{p-O,n} = d \}, $
		$ F : \E^3 \rightarrow \R, F(q) = \skal{p-O,n}-d $ then $ \pi = F^{-1}(\{0\}) $.
		Also, $ d_pF(v) = \skal{v,n} $ and hence $ d_pF \not \equiv 0 $. Thus $ d_pF $ surjects and $F$ is a submersion.
		
		\item Sphere: $ S = \{ p \in \E^3 ~|~ \skal{p-O,p-O} = r^2 \} $ and $ S $ is implicitly given by the function $ F : \E^3 \rightarrow \R, F(p) = \skal{p-O,p-O}-r^2 $. Here $ d_pF(v) = 2 \skal{v,p-O} $ and $ d_pF $ surjects as long as $p \not \equiv 0$ which does not happen in $ S $.
		$ F\big|_{\E^3 \setminus \{ 0 \} } $ is a submersion.
		
		\item  Hyperboloids: $O + e_1x + e_2y + e_3z$ such that
	 $ F_\pm(O  + e_1x + e_2y + e_3z) = 0 $ where  
	 	\[F_\pm(O  + e_1x + e_2y + e_3z) = \left(\dfrac{x}{a}\right)^2 + \left(\dfrac{y}{b}\right)^2 +\left(\dfrac{z}{c}\right)^2 \pm 1 \]
	 and
	 	\[ \nabla F = 2\left(e_1 \dfrac{x}{a^2} + e_2 \dfrac{y}{b^2} + e_3 \dfrac{z}{c^2}\right). \]
	 Then $ \nabla F = 0 \Leftrightarrow (x,y,z) = 0 $. Therefore $ F\big|_{\E^3 \setminus \{0\}} $ is a submersion with $\skal{\nabla F,v}= dF(v)$.
	\end{enumerate}
\end{example}

\begin{example}
	A counterexample is the following \emph{Lemniscate}.
	$ O + e_1x + e_2y $, where $ x^4 - x^2 + y^2 = 0 $. A regular parametrization is given by $ t \mapsto O + e_1 x(t) + e_2 y(t) = O + e_1 \sin(t) + e_2\sin(t)\cos(t). $ 
	The curve has a self-intersection at $ (x(t),y(t))=(0,0) $ which is equivalent to $ \forall k \in \mathbb{Z} t = k\pi $. Hence this is not a $ 1$-Dimensional submanifold.
\end{example}

\begin{definition}
	The tangent space of a $k$-dimensional submanifold $M \subset \E^n$ of $p \in M$ is the $k$-dimensional subspace
		\[ T_pM = d_oX(\R^k) \subset \R^n \]
	where $X: \R^k \supset V  \to \E^n$ is a parametrisation of $M$ around $p=X(o)$.
\end{definition}

\begin{remark}
	$T_pM$ is independent of the parametrisation $X$.
	
	Let $\tilde X = X_o\mu$ around $p = \tilde X (\tilde o)$ where $\mu: \tilde V \to V $ is a diffeomorphism with $\mu(\tilde o)=o$. Then
		\[ d_o\tilde X = d_{\tilde o}(X \circ \mu) = (d_oX)\circ d_{\tilde o}\mu. \]
	Therefore
		\[ d_{\tilde o}\tilde X(\R^k) = (d_oX)(d_{\tilde o}\mu(\R^k)) = d_oX(\R^k). \]
\end{remark}

\begin{remark}
	If we have $M=F^{-1}(\{0\})$ is defined as the level set of a submersion $F: U \to \R^{n-k}$ then 
		\[ T_pM = \ker d_pF. \]
		
	This follows from the following: 
	$F \circ X=0$ for any parametrisation $X:V \to \E^n$ around $p=X(o)$. Then the chain rule gives us
		\[ 0 = d_o(F \circ X) = (d_oF)\circ d_oX. \]
		\[ 0= (d_pF)\circ (d_oX(\R^k)) = (d_pF)(T_pM). \]
	Therefore $T_pM \subset \ker d_pF$. But since $F$ is a submersion $\dim (\ker d_pF)=k$ and $\dim(T_pM)=k$.
\end{remark}

\begin{example}
	Lets consider the set of orthogonal maps 
		\[ O(3)=\{ A \in \mathrm{GL}(3): AA^T = id \}
		= \{ A \in \mathrm{GL}(3): F(A)=0 \} \]
	where $F: \mathrm{GL}(3) \to \mathrm{Sym}(3), F(A) = AA^T-id$. $ \mathrm{Sym}(3) $ is a $6$-dimensional vector space.
	
		\[ d_AF: \mathrm{End}(\R^3) \to \mathrm{Sym}(3), d_AF(B) = BA^T + AB^T. \]
	This surjects since any element of $\mathrm{Sym}(3)$ can be written as $Y+Y^T$ so let $B=YA$. Hence $F$ is a submersion and $O(3)$ is a $3$-dimensional submanifold of $\mathrm{End}(\R^3) \approx M_{3 \times 3}$.
		\[ d_AF(B) = 0 \quad \Longleftrightarrow \quad
			BA^T + AB^T=0 \quad\Longleftrightarrow\quad
			BA^T \in \square(3), \]
	where $\square(3)$ is the skew-symmetric-endomorphisms. Therefore
		\[ T_AO(3) = \{ B \in \mathrm{GL}(3)= \mathrm{End}(\R^3) | BA^T \in \square(3) \} \]
	with is $3$-dimensional and
		\[ T_ASO(3) = T_AO(3), \quad A \in SO(3). \]
	
	This is an example of a Lie-group
\end{example}

\begin{exercise}
	Think about $\mathrm{GL}(n)$ and SL$(3)$.
\end{exercise}

Fun-fact:
	\[ \mathrm{End}(\R^n) = \mathrm{Sym}(n) \oplus \square(n). \]
	
\subsection{Functions on submanifolds}

Perviosly functions, vector fields etc. were defined on open sets of affine spaces, where notions of differentiability makes sense.
Now we want to consider functions on domains that are submanifolds of $\E^n$. To do this we define derivatives so that the chain rule holds.

\begin{definition}
	A function $\phi: M \to \E$ on a submanifold $M \subset \R^n$ is said to be differentiable at $p \in M$ with derivative
		\[ d_p\phi := d_0(\phi \circ X) \circ (d_oX)^{-1} : T_pM \to \R  \]
	if $\phi_oX : \R^k \supset V \to \E$ is differentiable at $o$ for some local parametrisation $X: V \to M$ of $M$ around $p$ with $X(o)=p$.
\end{definition}

\begin{remark}
	This definition makes sense as it does not depend on our choice of parametrisation $X$.
	
	Let $\tilde X= X \circ \psi$ is a reparametrisation at $p \in M$ for some diffeomorphism $\psi$ then $\phi \circ\tilde X = \phi  \circ  X\circ\psi$ is differentiable as soon as $\phi \circ X$ is differentiable. Moreover, if we assume $\psi(o)=o$ then 
		\[ d_o(\phi \circ \tilde X) \circ (d_o\tilde X)^{-1}
			= d_o(\phi \circ X \circ \psi) \circ (d_o(X \circ \psi))^{-1}
			 \]
		\[ = d_o(\phi \circ X) \circ d_o\psi \circ (d_o\psi)^{-1} \circ (d_oX)^{-1}
		= d_o(\phi \circ X)  \circ (d_oX)^{-1}. \]
	This definition can be easily generalised to $\E^n$-valued maps and thus to maps between submanifolds.
\end{remark}

\begin{remark}
	Suppose that $\Phi: \E^n \to \E$ is differentiable and $M$ is a submanifold of $\E^n$. Thus $\phi:= \Phi \big|_M :M \to \E$ is differentiable with
		\[ d_p\phi = d_p\Phi \big|_{T_pM} : T_pM \to \R, \quad p \in M. \] 
	
	Let $X: V \to M$ be a parametrisation of $M$ around $p=X(o)$, then $\phi \circ X = \Phi \circ X$ is differentiable and for $\xi = d_oX(x)$ then
		\[ d_p\phi(\xi) = d_o(\phi \circ X) \circ (d_oX)^{-1}(\xi)
			= d_0(\Phi \circ X)(x) = (d_{X(o)} \Phi) \circ d_oX(x)
			= d_p\Phi(\xi). \]  
\end{remark}

\begin{definition}
	Let $\phi: M \to \E$ be differentiable. Then the gradient of $\phi$ at $p \in M$ is the unique vector field $\grad \phi(p) \in T_pM$ with
		\[ d_p\phi(\xi) = \skal{\xi, \grad \phi(p)}, \quad \forall \xi \in T_pM. \]
	Since  $d_p\phi : T_pM \to \R $ , $d_p \phi \in (T_pM)^*$, with Riesz-Fischer $\exists!$ vector $v \in T_pM$ such that $d_p\phi= \skal{.,v}$
\end{definition}

\begin{example}
	Suppose $\R^2 \supset V \in (u,v) \mapsto X(u,v) \in \E^3$ is a parametrised surface with $I= Edu^2 + 2Fdudv + Gdv^2$. Let
		\[ X_u^*:= \frac 1{EG-F^2}(GX_u-FX_v), \quad
			X_v^*:= \frac 1{EG-F^2} (-FX_u +EX_v) \in T_{(u,v)}X. \]
	Cause
		\[ \skal{X_u^*,X_u} = \skal{X_v^*,X_v}=1, \quad
			\skal{X_u^*,X_v} = \skal{X_v^*,X_u} =0 \]
	$X_u^*,X_v^*$ is a dual basis. 
	
	Now let $\phi : M= X(V) \to \E$ be a differentiable function and define $\psi := \phi \circ X$. Then $\xi = d_{(u,v)} X(x)  \in T_{X(u,v)}M$ and
		\[ \skal{\grad \phi(X(u,v)), \xi} = d_{X(u,v)}\phi(\xi) = (d_{(u,v)}\psi)(x). \]
	Thus 
		\[\skal{ \grad \phi \circ X, X_u} = \psi_u, \quad \skal{ \grad \phi \circ X, X_v} = \psi_v.\]
		
	Hence 
		\[ \grad \phi \circ X = X_u^*\psi_u + X_v^*\psi_v
			= \frac {G\psi_u - F\psi_v}{EG-F^2}X_u + \frac {E\psi_v - F\psi_u}{EG-F^2}X_v, \]
	where
		\[ \grad \phi \circ X = V \to TM. \]
\end{example}
	

Now we know how to differentiate functions on submanifolds : we can find analogues of notions such as I \& II, shape operator, covariant derivative.

\begin{definition}
	Let $  \xi $ be a tangential verctor field, i.e., $ \xi : M \rightarrow \R^n $ such that $ \forall p \in M(\xi(p) \in T_pM) $. Let $ \eta \in T_pM $. Then 	
		\[ \nabla_\eta \xi \big|_p = (d_p\xi(\eta))^T = (d_o(\xi \circ X)(y))^T \]
	where $ X: V \rightarrow \E^n $ is a local parametrisation of $ M $ at $ p $ with $ p = X(o) $ and $ d_oX(y) = \eta $. As usual $ (\dots)^T $ denotes the tangential part, i.e., the orthogonal projection $ \R^n \rightarrow T_pM $. $ \nabla $ is called the \emph{Levi-Civita connection}.
\end{definition}

\begin{remark}
	In case of parametrised surfaces $ X: V \rightarrow \E^3 $, $ M = X(V) $. Then $ Y:= xi : V \rightarrow \E^3 $ tangential vector field in the sense of chaper 2.
		\[ \nabla_\eta \xi = \nabla_yY \quad d_{(u,v)}X(y) = \eta \quad X(u,v) = p \] 
\end{remark}

This yields a notion of second derivatives on surfaces:

\begin{definition}
	
	The \emph{Hessian} of $ \phi: M \rightarrow \E $ at $ p \in M $
		\[ T_pM \times T_pM \ni (\xi,\eta) \mapsto (\mathrm{hess} \phi)\big|_p(\xi,\eta) = \skal{\eta,\nabla_\eta \mathrm{grad}\phi\big|_p } \]
	is a symmetric tensor.
	
\end{definition}

\begin{proof}
	Exercise/technical.
\end{proof}

\begin{remark}
	
	The Hessian is the covariant derivative of $ d\phi: M \times TM \rightarrow \R $, $ \xi,\eta: M \rightarrow TM $
		\[ \mathrm{hess}\phi(\xi,\eta) = d(d\phi(\eta)(\xi)-d\phi(\nabla_\xi\eta) = ''(\nabla_\xi d\phi)(\eta)'' \]	
\end{remark}

The Hessian depends on the covariant derivative, hence on the induced metric,not just on the differentiable structure on M.

\begin{lemma}[Poincaré Lemma]
	A tangential vector field $ \xi: M \rightarrow \R^n $ has a local potential, i.e., locally $ \xi = \mathrm{grad}\phi $, if and only if	
		\[ (\nu,\eta) \mapsto \skal{\eta, \nabla_\nu \xi} \] is symmetric.
\end{lemma}

\begin{proof}
	
	We saw in the Lemma above that symmetry of $ (\nu,\eta) \mapsto \skal{\eta,\nabla:\nu \xi} $ is a necessary condition for $ \xi = \mathrm{grad}\phi. $
	For sufficiency, let $ X: V \rightarrow M \cap U $ be a local parametrisation. Let $ \xi_1, \dots, \xi_k : M \rightarrow \R^k$ be a tangential v.f. such that $ \xi_i \circ X = \dfrac{\partial X}{\partial x_i} = dX(\dfrac{\partial}{\partial x_i}) $.
	Then given $ \xi: M \rightarrow \R^n $ we seek a function $ \psi: V\rightarrow \R $ such that $ \dfrac{\partial \psi}{\partial x_i} = \skal{\xi_i, \xi} \circ X $ . Then
		\[ \dfrac{\partial}{\partial x_i}(\skal{\xi_i, \xi} \circ X) = \skal{ \dfrac{\partial}{\partial x_i}(\xi:j \circ X), \xi \circ X}+ \skal{\xi_j \circ X, \dfrac{\partial}{\partial x_i}(\xi \circ X)} \] then by the Leibniz rule:
		\[ =\skal{\D{x_i} (\xi_j \circ X), \xi \circ X} + \skal{\xi_j \circ X,\D{x_i} (\xi \circ X)}
		= \left[ \skal{\D{\xi_i}\xi_j , \xi} + \skal{\xi_j, \D{\xi_i}\xi} \right]\circ X \]
	
	LHS is symmetric as soon as $ \skal{ \xi_j, \nabla_{\xi_i} \xi} $ is.
	(N.B. $ \nabla_{\xi_i}\xi_i $ because $ \nabla $ is ''torsion free''.)
	By Poincaré's Lemma $ \exists \psi: V \rightarrow \R $ such that $ \dfrac{\partial \psi}{\partial x_i}= \skal{\xi_i,\xi} \circ X $.
	Now define $ \phi = \psi \circ X^{-1}: X(v) \rightarrow \R $. Check that $ \xi = \mathrm{grad}\phi $.
\end{proof}

\begin{lemma}
	Suppose that $X: \R^n \supset V \to \E^n$ is a local parametrisation of $M \subset \E^n$. Then
		\[ \nabla_{\xi_i}\xi_j - \nabla_{\xi_j}\xi_i=0, \]
	where $\xi_i,\xi_j$ are vector fields so that $\xi_i \circ X = \frac{\partial X}{\partial x_i} , \xi_j \circ X = \frac{\partial X}{\partial x_j}=dX\left( \frac \partial{\partial x_j} \right)$.
\end{lemma}

\begin{proof}
	\[ \left( \nabla_{\xi_i}\xi_j \right) \circ X = \left( d\xi_j (\xi_i)\circ X \right)^T
		= \left( d(\xi_j \circ X)dX^{-1} ( \xi_i \circ X) \right)^T\]
	\[ = \left( d\left( \frac{\partial X}{\partial x_j}\right)  \left( \frac \partial {\partial x_i}\right)\right)^T = \left( \frac {\partial^2 X}{\partial x_j \partial x_i}   \right)^T. \]
	similarly 
		\[ \left( \nabla_{\xi_j}\xi_i \right) \circ X = \left( \frac {\partial^2 X}{\partial x_i \partial x_j}   \right)^T . \]
	Since Schwarz the mixed partial derivatives commute, nothing is left to proof.
\end{proof}

\subsection{Vector fields and flows}

\begin{definition}
	Let $\xi : M \to \R^n$ be a (tangential) vector field on a $k$-dimensional submanifold $M \subset \E^n$. A curve $C: I \to M$ on an open interval $I \subset \R$ is called an \emph{integral curve of $\xi$} if
		\[ C' = \xi \circ C. \]
	It is maximal if it cannot be extended as an integral curve.
\end{definition}

\begin{remark}
	We do not require regularity. For example we could have $\xi=0$. Then all integral curves are constant.
\end{remark}

\begin{lemma}
	Through any point $p \in M$ passes a unique maximal integral curve of a vector field $\xi$.
\end{lemma}

\begin{proof}
	Let $X: V \to M$ be a local parametrisation of $M$ around $X(o)=p \in M$ and write $\xi \circ X = dX(y)$, i.e.,
		\[ \xi \circ X(\tilde o) = d_{\tilde o} X(y(\tilde o)) \]
	for all $\tilde o \in V$ and $y : V \to \R^k$.
	
	The ansatz $C= X \circ \gamma$ yields
		\[ C' = \xi \circ C  \]
	iff
		\[ dX \circ \gamma' = dX(y \circ \gamma). \]
	Since $d_{\tilde o}X$ is an isomorphism for all $\tilde o \in V$, this holds for all $\gamma' = y \circ \gamma$. Then by applying Picard-Lindelöf then the initial-value-problem (IVP) 
		\[ \gamma' = y \circ \gamma, \qquad \gamma(0)=o \]
	has a solution $\gamma: J \to V$ on some open interval $J \subset \R$ with $0 \in J$. This is unique up to extension.
\end{proof}

The max integral curves of a vector field $\xi$ can be assembled into a single map.

\begin{theorem, definition}
	Give a smooth tangential vector field $\xi$ on a submanifold $M$. There exists an unique smooth map called its \emph{maximal flow}
		\[ \Phi : W \to M, \quad (t,p) \mapsto \Phi_t(p) \]
	on an open neighbourhood $W$ of $\{0\}\times M \subset \R\times M$ so that 
	\begin{enumerate}
		\item $\Phi_0=id$
		\item $I_p:= \{ t| (t,p) \in W \}$ is an open interval containing $0$ for all $p \in M$
		\item $I_p \ni t \mapsto \Phi_i (p)$ is the maximal integral curve of $\xi$ through $p$.
	\end{enumerate}
\end{theorem, definition}

\begin{proof}
	1),2) and 3) uniquely define $\Phi$.
	
	Check that $W=\bigcup_{p \in M } I_p \times \{p\}$ is an open subset of $\R \times M$ and  that $\Phi$ is smooth. The smoothness dependence on the initial conditions. 
\end{proof}

\begin{remark, definition}
	If $M \subset \E^n$ is compact (closed and bounded) or more generally, $\xi$ is compactly supported on $M$, i.e., there exists a compact $V \subset M$ such that $\xi |_{M \setminus V}=0$, then
		\[ W=\R \times M, \text{ anf} \Phi_{s+t} = \Phi_s \circ \Phi_t. \]
	Moreover $\Phi_t : M \to M$ is a diffeomorphism for any fixed $t \in \R$ and thus $\Phi: \R \times M \to M$ defines a $1$-parameter group of diffeomorphisms. This is a subgroup of the group of diffeomorphisms from $M$ to $M$ and is often called \emph{flow}.
\end{remark, definition}

\begin{example}
	Let $M = \{ p \in \E^3 : \skal{p-O,p-O} =1 \}$ be the unit sphere centred around $O$. Define $\xi(p)= e_3 \times (p-O)$. This resembles the rotation of the earth.
	
	If we write $p=O+ e_1x+e_2y+e_3z$ then 
		\[ \xi(p)=-e_1y + e_2x \]
	and 
		\[ \Phi_t(p)=e_1(x\cos t -y\sin t) + e_2 (x \sin t + y \cos t) + e_3z. \]
\end{example}

\begin{remark}
	(c.f., $C_t(s)$ in Clairaut's theorem) resembles the flow $\Phi_t(C(s))$ on a surface of revolution.
\end{remark}

WARNING: In the case $\xi$ is not compactly supported there may not be an $\varepsilon>0$ such that $W \supset (-\varepsilon,\varepsilon) \times M$. For example, with
	\[ M= \{ O+e_1u + e_2v : \abs{u}<1 \} \subset \E^2 \text{ and } \xi = \pmat{1\\0}. \]
The flow for some fixed $p= O+e_1u +e_2v$ is 
	\[ \Phi_t(p): (-1-u, 1-u) \ni t \mapsto O+e_1(u+t)+e_2 v. \]
Therefore we cannot obtain a $1$-parameter group of diffeomorphisms $M \to M$.

However, we obtain a \emph{local flow}.
	
\begin{remark}
	In general, the maximal flow $\Phi : W \to M$ is a local flow, i.e., There exists an open neighbourhood $(-\varepsilon , \varepsilon) \times U \subset M$ of $(0,p)$ for any point $p \in M$ such that
	\begin{enumerate}
		\item $\Phi_t |_{U}: U \to \Phi_t(U)$ is a diffeomorphism for all $t \in (-\varepsilon,\varepsilon)$
		\item $\Phi_{s+t}(q) = (\Phi_s \circ \Phi_t)(q)$ wherever $q \in U$ and $s,t,a+t \in (-\varepsilon,\varepsilon)$
	\end{enumerate}
\end{remark} 

\begin{proof}
	\begin{enumerate}
		\item Fix $p \in M$. Since $W \subset \R \times M$ there exists an open neighbourhood $U$ of $p$ and an $\varepsilon>0$ such that $(-\varepsilon,\varepsilon) \times U \subset E$ (box topology). Since $\Phi_0 |_{U} = id_U$ is a diffeomorphism, by inertia, $\Phi_t|_U$ is a diffeomorphism for all $t \in (-\varepsilon,\varepsilon)$. Perhaps after shrinking.
		
		\item Suppose $s \mapsto C(s)= \Phi_{s+t}(q)$ for a fixed $t$. Then 
		\[ C'(s) = \xi \circ C(s)\text{ and } C(0)=\Phi_t(q) \in M, \]
		i.e., $C$ is the integral curve of $\xi$ through $\Phi_t(q)$. Hence by uniqueness $C(s)=\Phi_s\circ \Phi_t(q)$.
	\end{enumerate}
\end{proof}

\begin{remark}
	
	$ \xi: M \rightarrow \R^n $ is a tangential vector field and let $ \phi : M \rightarrow \R $ be a differentiable function. Then we define a new function
		\[ \xi\phi: M \rightarrow \R, \quad p \mapsto (\xi\phi)(p) := (d_p\phi)(\xi(p)). \]
	Thus, we can think of a vector field $ \xi $ as a differential operator yielding a directional derivative of $ \phi $ at each point. In particular, $ \phi \mapsto \xi\phi $ is linear and the Leibniz-rule
		\[ \xi(\phi\psi) = (\xi\phi)\psi + \phi(\xi\psi) \]
	holds.
	In abstract definition of manifolds, vector fields are often characterized in this way.
	
\end{remark}

\begin{lemma}
	If ($ (\xi\phi)(p) = 0 $ for all functions $ \phi: M \rightarrow\R $ then $ \xi(p) = 0 $
\end{lemma}
\begin{proof}
	
	Let $ X : V \rightarrow M $ be a local parametrization of $ M $ around $ p = X(o) $.
	Consider the ''coordinate functions'':
		\[ \phi_i = \pi_i \circ X^{-1}: X(v) \rightarrow \R \]
	where $ \pi_i:\R^k \rightarrow \R, \quad (y_1, \dots, y_k)^t \mapsto y_i $.
	We use these as test functions: Let  $ y = (y_1, \dots, y_k)^ \in \R^k $ such that $ \xi(p) = d_oX(y). $ Then $ (\xi\phi_i)(p) = (d_o\phi_i)(\xi(p))= d_o(\phi_i \circ X)(d_oX)^{-1}(d_oX(y)) = d_o\pi_i(y) = y_i $. This holds for all $ i $ thus $ y = 0 \Rightarrow \xi(p) = 0 $.
	
\end{proof}

\begin{lemma, definition}
	
	Let $ \xi, \eta $ be smooth vector fields on $ M $. Then there is exactly one vetor field $ [\xi,\eta] $ on $ M $ such that, for every smooth function $ \varphi:M \rightarrow \R, $
		\[ [\xi,\eta]\varphi = \xi(\eta\varphi)-\eta(\xi\varphi). \]
	$ [\xi,\eta] $ is called the \emph{Lie bracket of $ \xi $ and $ \eta $}.
	
\end{lemma, definition}

\begin{proof}
	
	Let $ X: V \rightarrow M $ be a parametrisation of $ M $ around $ p $. Let $ \xi_i $ be a vector field such that $ \xi_i \circ X = \frac{\partial X}{\partial x_i}. $ Thus 
		\[ (\xi_i \phi) \circ X = (d\phi)(\xi_i) = d(\phi_oX) \circ dX^{-1}(\xi_i \circ X) = d(\phi \circ X)(\frac{\partial}{\partial x_i}) = \frac{\partial}{\partial x_i}(\phi \circ X). \]
	 Then
	 	\[ (\xi_i(\xi_j \phi) - \xi_j(\xi_i \phi)) \circ X = \frac{\partial}{\partial x_i}(\xi_j\phi \circ X) - \frac{\partial}{\partial x_j}(\xi_i\phi \circ X) = \frac{\partial}{\partial x_i}(\frac{\partial(\phi \circ X)}{\partial x_j} - \frac{\partial}{\partial x_j}\frac{\partial(\phi \circ X)}{\partial x_i}) = 0 \]
	since mixed partial derivatives commute.
	
	$ \{ \xi_1(p), \dots, \xi_k(p) \} $ for a basis for $ T_pM, \forall p \in M $ Write $ \xi = \sum_{i =1}^{k} \alpha_i \xi_i $ and $ \eta = \sum_{i=1}^{k}\beta_i \xi_i $ for some differentiable functions $ \alpha_i,\beta_i $. Now, by the Leibniz rule
		\[ \xi(\eta \phi) - \eta(\xi \phi) = \sum_{i =1}^{k} \sum_{j =1}^{k} (\alpha_j ( \xi_j \beta_i) - \beta_j(\xi_j\alpha_i))(\xi_i\phi_i). \]
	Thus with $ [\xi,\eta] := \sum_{j =1}^{k} (\alpha_j ( \xi_j \beta_i) - \beta_j(\xi_j\alpha_i))\xi_i $ we have that 
		\[ [\xi,\eta]\phi = \xi(\eta\phi)-\eta[\xi\phi] \] for all functions $ \phi: M \rightarrow \R $. By the previous lemma, this is the unique vector field with this property.
\end{proof}

\begin{remark}
	Properties of Lie bracket:
	\begin{itemize}
		\item skew symmetric: $ [\xi,\eta] = - [\eta,\xi] $ and
		\item Jacobi identity: $ [\xi, [\eta,\zeta]] + [\zeta,[\xi,\eta]] + [\eta,[\zeta,\xi]] = 0 $.
	\end{itemize}
\end{remark}

\begin{definition}
	Thus with the Lie bracket as multiplication, the vector space of smooth vector fields is a \emph{Lie algebra}. This defines a Lie algebra.
\end{definition}

\begin{remark}
	We saw in the above proof that $[\xi_i,\xi_j]=0$. ww know from a previous lemma that
	\[ \nabla_{\xi_i} \xi_j - \nabla_{\xi_j}\xi_i = 0 \]
	for $\nabla$ the Levi-Civita connection. One can deduce that
	\[ [\xi,\eta] = \nabla_\xi \eta - \nabla_\eta \xi  \] 
	for all vector fields $\xi,\eta$ on $M$. This property of $\nabla$ is known as ''torsion free''.
\end{remark}

\begin{definition}
	
	Two vector fields $ \xi, \eta $ \emph{commute} if $ [\xi,\eta] = 0 $.
	
\end{definition}

\begin{example}
	Not all vector fields commute. $ M = X(V) $, $ X: \R^2 \supseteq V \rightarrow \E^3 $ parametrized surface. Let $ \xi_i:V \rightarrow \R^, \xi_i\circ X = \frac{\partial X}{\partial x_i}. $ Let $ Y = \xi_1 + a\xi_2 $ where $ a $ satisfies $ a\circ X = \phi_1(x) = x_1 $. Since 
		\[ [\xi_1, a\xi_2]\varphi
		= \xi_1(a\xi_2 \varphi)- a\xi_2(\xi_1\varphi)
		= (\xi_1 a)\xi_2\varphi + a\xi_2(\xi_1\varphi) - a\xi_2(\xi_1\varphi)
		= (\xi_1 a)\xi_2\varphi \]
	we get 
		\[ [\xi_1,Y] = [\xi_1,\xi_1] + [\xi_1,a\xi_2] = (\xi_1a)\xi_2 \]
	and 
		\[ (\xi_1 a) \circ X = (da(\xi_1))\circ X = d(a \circ X) \circ (dX^{-1})(\xi_1) = \frac {\partial \pi_1}{\partial x_1} = 1.   \]
	Hence, $ [\xi_1,Y] = \xi_2 \neq 0 $.
\end{example}

If I flow along $ \xi $ and then $ \eta $, is this the same as flowing along $ \eta $ and then $ \xi $?

\begin{example}
	
	We can look at the vector field that is generated by the ''tangential vectors of the polar coordinates'' -- those are the tangential vectors going in the radial direction and those going in the angular direction. In this case the vector fields commute. On the other hand, by choosing horizontal vectors instead of the tangential vectors of the angular direction forfeits this property.  
	
\end{example}

\begin{definition}
	
	Two (local) flows $ \Phi,\Psi $ \emph{commute} if (whenever all terms are well-defined) 
		\[ \Phi_t \circ \Psi_s = \Psi_s \circ \Phi_t \]
	
\end{definition}

\begin{theorem}
	
	Two vector fields commute if and only if their maximal flows commute, i.e. $ [\xi,\eta] = 0 $	
	
\end{theorem}

\begin{proof}
exercise for the reader.
\end{proof}

\begin{theorem, definition}
	
	Let $ \xi_i \quad i = 1, \dots, R $ be pairwise commuting vector fields, i.e. $ [\xi_i,\xi_j]= 0 $ \text{whenever} $ i \neq j $ on a $ k- $ dimensional submanifold $ M $ that are linearly independent at every point $ p \in M $. Then there exists a local parametrisation $ X : \R^k \supseteq V \rightarrow M $ around each point $ p \in M $ such that 
		\[ \xi_i \circ X = \dfrac{\partial X}{\partial x_i}\quad \text{ for } i = 1, \dots, k \]
	$ \xi_q, \dots, \xi_k $ is called the \emph{Gaussian basis field of $ X $}.
	
\end{theorem, definition}

\begin{remark}
	
	We have already seen that the vector fields of a Gaussian basis field commute.
	
\end{remark}

\begin{proof}
	
	Let $ \Phi^i $ denote the maximal flow of $ \xi_i $. Fix $ p \in M $ and define 
		\[ X(x_1,\dots, x_k)= \Phi^1_{x_1}\circ \Phi^2_{x_2}\circ \dots \circ \Phi^k_{x_k}(p) \]
	on an open neighborhood of $ 0 \in \R^k $ where the expression is defined.
	Note that $ X(0)=p. $
		\[ \dfrac{\partial X}{\partial x_i}(x_1,\dots,x_k) = \dfrac{\partial}{\partial x_i} (\Phi^1_{x_1}\circ \dots \circ \Phi^k_{x_k}(p)) = \dfrac{\partial}{\partial x_i} (\Phi^i_{x_i} \circ \Phi^1_{x_1}\circ \dots \circ \Phi^k_{x_k}(p))  \]	
		\[ = \xi_i \circ \Phi^i_{x_i} \circ \Phi^1_{x_1} \circ \dots \circ \Phi^k_{x_k} = \xi_i \circ \Phi^1_{x_1} \circ \dots \circ \Phi^k_{x_k} = \xi_i \circ X(x_1, \dots, x_k).  \]
	Since $ \xi_i $ are linearly independent, $ X $ is an immersion and thus defines a local parametrisation of $ M $ around $ p $.
		
\end{proof}

\subsection{Surfaces revisited}

We shall now return to surfaces in $ \E^3 $ and re-investigate their geometry from the perspective of manifolds.
Suppose $ M \subseteq \E^3 $ is a $ 2- $ dimensional submanifold. Suppose that $ M $ is \emph{orientable}, i.e., $ \exists \nu: M \rightarrow S^2 $ submanifold, such that $ \forall p \in M ~ \nu(p) \perp T_pM $. Then $ \nu $ is called \emph{a Gauss map } of $ M $ . Note that a surface admits two Gauss maps and a choice of one of them -- this equips $ M $ with an \emph{orientation} and makes $ M $ an \emph{orientable submanifold}.

\begin{definition}
	
	The \emph{shape operator} of $ (M,\nu) $ is given by	
		\[ \s := -d\nu \quad \s(p):T_pM \rightarrow T_pM  \]
	
	
\end{definition}

\begin{remark}
	If $ X:V \rightarrow M $ is a local parametrisation of $ M $ around $ X(u,v)=p \in M $ and $ N:= \nu \circ X $ then 
		\[ \s_p = -d_p\nu = - d_{(u,v)}(\nu \circ X)(d_{(u,v)X})^{-1} = -d_{(u,v)}N \circ (d_{(u,v)}X)^{-1}. \]
	Thus it yields our previous definition of the shape operator. Hence, the principal, mean and Gauss curvatures of $ (M,\nu) $ are the eigenvalues, $ \frac{1}{2} \mathrm{trace} $ and the determinant of $ \s $ respectively.
\end{remark}

\begin{definition}
	
	Two vector fields $ \xi,\eta $ on $ M $ are called a \emph{local basis field} around $ p \in M $ if $ \forall q \in U ~ \mathrm{span}\{ \eta(q), \xi(q) \} = T_qM$, where $ U $ is an open neighborhood around $ p \in M $.
	$ (\xi,\eta) $ is called:
	\begin{itemize}
		\item \emph{orthnormal} if $ \skal{\xi,\eta} = 0, |\xi|^2 = |\eta|^2 = 1 $ on $ U $.
		\item \emph{principal} if $ \s\xi = \kappa^+\xi, \s\eta = \kappa^-\eta $ on $ U $.
	\end{itemize} 
	
\end{definition}

\begin{remark}
	
	A submanifold $ M $ does not necessarily carry a non-vanishing (smooth) vector field: e.g. every vector field on $ S^2 $ must have at least one zero (hairy ball theorem).
	
\end{remark}

\begin{lemma}
	If $ [\xi,\eta]=a\xi - b\eta $ for some functions $ a,b:M \rightarrow \R $ for an orthonormal basis field $ (\xi,\eta) $, then
		\[ \nabla_\xi \xi = - b \eta, \quad \nabla_\xi \eta = a\xi, \quad \nabla_\eta \xi =  b\eta, \quad \nabla_\eta \eta = - a\xi \]
\end{lemma}

\begin{proof}
	
	$ \nabla $ is a ''metric connection'' so that 
		\[ d_\xi\skal{\eta_1,\eta_2} = \skal{\nabla_\xi \eta_1,\eta_2} +\skal{\eta_1,\nabla_\xi \eta_2}. \]
	Now 
		\[\skal{\nabla_\xi \xi,\xi} = \dfrac{1}{2} d_\xi\skal{\xi,\xi} = 0 \]
		\[\skal{\nabla_\xi \eta,\xi} + \skal{\eta,\nabla_\xi \xi} = d_\xi\skal{\eta,\xi} = 0 \]
	Similarly, $ \skal{\nabla_\eta \xi,\eta} + \skal{\xi,\nabla_\eta \eta} = 0  $ and $ \skal{\nabla_\eta \eta,\eta} = 0 $.
	Thus 
		\[ \nabla_\xi \xi = \alpha \eta, \quad \nabla_\xi \eta = -\alpha \xi + \beta \eta, \quad \nabla_\eta \xi =  \gamma\xi - \delta \eta, \quad \nabla_\eta \eta = \delta \xi. \]
	But we also know that $ \nabla $ is torsion free, i.e.,
		\[ \alpha \xi - \beta \eta  = [\xi,\eta] = \nabla_\xi \eta - \nabla_\eta \xi = (\gamma - \alpha)\xi + (\beta + \delta)\eta = - \alpha \xi + \delta \eta. \]
	Thus $ -a = +  \alpha, \delta = -b $
		 	
\end{proof}

\begin{lemma}
	
	If $ (\xi,\eta) $ is a principal orthonormal local basis field, then
		\[ [\xi,\eta] = - \dfrac{1}{\kappa^+ - \kappa^-}((\eta\kappa^+)\xi + (\xi\kappa^-)\eta). \]
	
\end{lemma}

\begin{proof}
	
	Write $ [\xi,\eta] = a\xi - b\eta $ as above. By the Codazzi equations
		\[ 0 = (\nabla_\xi \s)\eta - (\nabla_\eta \s) \xi = \nabla_\xi(\s\eta) - \s\nabla_xi \eta - \nabla_\eta(\s\xi) - \s\nabla_eta \xi  \]
		\[ = \nabla_\xi(\kappa- \eta) - a \s \xi - \nabla_\eta(\kappa^+\xi) + b\s\eta \]
		\[ = (\xi\kappa^-) \eta + \kappa^-a\xi -a\kappa^+\xi - (\eta\kappa^+)\xi - \kappa^+b\eta + b \kappa^-\eta. \]
	By collecting the terms
		\[ 0 = [ (\kappa^- \kappa^+)a - (\eta\kappa^+) ]\xi + [ \xi \kappa^- - b(\kappa^+ - \kappa^-) ] \eta \]
		\[ \Rightarrow a = \dfrac{\eta \kappa^+}{\kappa^+ - \kappa^-}, \quad b= \dfrac{xi \kappa^-}{\kappa^+ - \kappa^-}. \]
	We would like to show that there exists curvature line coordinates on a surface away from umbilic points. This requires finding functions $ \lambda,\mu: M \rightarrow \R $ such that $ [ \lambda\xi,\mu\eta ] = 0 $ and  then applying the last theorem of the last section to $ \xi_1 = \lambda\xi, \xi_2 = \mu \eta $. This is not a trivial problem so we just consider some special cases.
	
			
\end{proof}