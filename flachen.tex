
\subsection{Parametrisierung \& Metrik}
\begin{definition}
	

Eine Abbildung \[ X:\R^2 \supseteq M \rightarrow \E \] heißt \emph{parametrisierte Fläche}, falls $M$ offen und zusammenhängend ist und $X$ regulär ist, d.h., $  \forall(u,v) \in M~:~d_{(u,v)}X: \R^2 \rightarrow V $ ist injektiv.
Wir sagen auch: $  X  $ ist eine \emph{Parametrisierung} der \emph{Fläche} $ X(M) \subseteq \E $.
Wobei $ d_{(u,v)} $ definiert ist über:
	\[d_{(u,v)}X(\pmat{x \\y})= X_u(u,v)\cdot x + X_v(u,v) \cdot y. \]

\end{definition}

\begin{remark}
	
 Äquivalent zur letzten Forderung ist die Forderung, dass die Jacobi-Matrix maximalen Rang hat. Diese braucht aber eine Festgelegte Basis, was oft zu Schwierigkeiten bei Berechnungen führt und wird daher von Prof. Jeromin nicht empfohlen.

\end{remark}

\begin{remark}
	Einmal mehr sind alle geforderten Abbildungen so oft differenzierbar, wie wir das wünschen.
\end{remark}

\begin{remark}
	
	$ d_{(u,v)} : \R^2 \rightarrow V$ ist die Ableitung am Punkt $ (u,v) \in M $, \[X(u+x,v+y) \approx X(u,v) + d_{(u,v)}  X(\begin{pmatrix}
	x\\
	y
	\end{pmatrix}) = X(u,v) + X_u(u,v)\cdot x + X_v(u,v)\cdot y,  \]
	wir können also identifizieren: \[ d_{(u,v)}X \cong (X_u,X_v)(u,v), \]
	bzw., nach Wahl einer Basis von V, mit der Jacobi-Matrix am Punkt $ (u,v) $.
	
\end{remark}

\begin{example}
	


Ein \emph{Helicoid} $ X: \R^2 \rightarrow \E^3 $ ist die \emph{(Regel-)Fläche}  
\[ \R^2 \ni (r,v) \mapsto O + e_1r \cos(v) + e_2 r \sin(v) + e_3 v \in \E^3. \]

Wir zeigen, dass $(X_r,X_v)(r,v)$ linear unabhängig für alle $ (r,v) \in \R^2 $ sind:

\[ X_r(r,v) = e_1\cos(v) + e_2 \sin(v) \neq 0 \]
\[ X_v(r,v) = -e_1r \sin(v) + e_2 r \cos(v) + e_3 \neq 0 \]

und da $ X_v(r,v) $ von $e_3 \neq 0$  abhängt sind die beiden linear unabhängig.

\end{example}

\begin{example}
	
	Eine übliche Parametrisierung von $ \mathbb{S}^2 \subseteq \E^3 $ (mit Mittelpunkt $O \in \E^3$) ist  
	\[ (u,v) \mapsto O + e_1\cos(u)\cos(v) + e_2\cos(u)\sin(v)+e_3\sin(u) \]
	liefert keine parametrisierte Fläche, da die Sphäre an den Polen nicht regulär ist.
	Dies ist also nur eine Parametrisierung auf $ M = (-\frac{\pi}{2},\frac{\pi}{2}) \times \R $
	
	Insbesondere kann man sogar zeigen, dass es keine (reguläre) Parametrisierung der (ganzen) Sphäre gibt (''Hairy Ball Theorem'' bzw. ''Satz vom Igel'').
	
	$ \mathbb{S}^2 $ ist also \textbf{keine} Fläche im Sinne der Definition. Dieses ''Problem'' wird später gelöst.
\end{example}

\begin{lemma, definition}
	
	Die \emph{induzierte Metrik} oder \emph{erste Fundamentalform} einer parametrisierten Fläche $ X: M \rightarrow \E$ ist definiert durch
	\[ \mathrm{I} := \langle dX,dX \rangle. \]
	Für jeden Punkt $ (u,v) \in M $ liefert 
	\[ \R^2 \times \R^2 \ni \left( \pmat{x_1\\y_1}, \pmat{x_2\\y_2} \right)  \mapsto \mathrm{I} \big|_{(u,v)}(\begin{pmatrix}
	x_1 \\
	y_1
	\end{pmatrix},\begin{pmatrix}
	x_2 \\
	y_2
	\end{pmatrix}) := \left\langle d_{(u,v)}X(\begin{pmatrix}
	x_1 \\
	y_1
	\end{pmatrix}),  d_{(u,v)}X(\begin{pmatrix}
	x_2 \\
	y_2
	\end{pmatrix})  \right\rangle \]
	eine positiv definite, symmetrische Bilinearform.
\end{lemma, definition}

\begin{proof}
	%Das darf ich wohl machen.. \#getrektluka
	Zu zeigen ist, dass $I\big| _{(u,v)}$ für jeden Punkt $(u,v)\in M$ eine positiv definite symmetrische Bilinearform ist. 
	
	Weil $ I\big| _{(u,v)} $ eine Komposition aus linearen Funktionen und einer Bilinearform ist, ist auch $I\big| _{(u,v)}$ linear. Die Symmetrie ist ebenfalls leicht ersichtlich. Fehlt noch die positive Definitheit.
	
	Sei $\pmat{x\\y} \neq 0$ beliebig, so gilt
		\[ I\big| _{(u,v)}(\pmat{x\\y},\pmat{x\\y})
			= \skal{d_{(u,v)}X(\pmat{x\\y}),d_{(u,v)}X(\pmat{x\\y})}> 0. \]
	Die letzte Ungleichung gilt, weil $d_{(u,v)}X$ injektiv und linear ist, daher bildet nur $0$ auf $0$ ab. Also ist $d_{(u,v)}X(\pmat{x\\y})\neq 0$. Der Rest folgt, weil $\skal{.,.}$ positiv definit ist.
\end{proof}

\begin{remark}
	
	$I$ wird oft mit Hilfe der \emph{Gramschen Matrix} notiert: \[ \mathrm{I} = \begin{pmatrix}
	E & F\\
	F & G
	\end{pmatrix} = E du^2 + 2Fdudv + Gdv^2 \] mit
	\[ E=|X_u|^2,\quad F = \langle X_u,X_v \rangle,\quad G = |X_v|^2.  \]
	


Dann gilt für $ (u,v) \in M$ : \[ I \big|_{(u,v)}(\begin{pmatrix}
x_1 \\
y_1
\end{pmatrix},\begin{pmatrix}
x_2 \\
y_2
\end{pmatrix}) := \skal{ d_{(u,v)}X(\begin{pmatrix}
x_1 \\
y_1
\end{pmatrix}),  d_{(u,v)}X(\begin{pmatrix}
x_2 \\
y_2
\end{pmatrix}) } \] \[= \skal{ X_u(u,v)x_1 + X_v(u,v)y_1, X_u(u,v)x_2 + X_v(u,v)y_2 } \]
\[ = E(u,v)x_1x_2 + F(u,v)(x_1y_2 + x_2y_1) + G(u,v)y_1y_2 \]
\[ = \begin{pmatrix}
x_1 \\
y_1
\end{pmatrix}^t \begin{pmatrix}
E & F \\ 
F & G
\end{pmatrix}\Bigg|_{(u,v)} \begin{pmatrix}
x_2\\
y_2
\end{pmatrix} \]

\end{remark}

\begin{example}
	
	\begin{enumerate}
		
		\item Ein \emph{Zylinder}
			\[ (u,v) \mapsto X(u,v) := O + e_1x(u) + e_2y(u) +e_3v \]
		hat induzierte Metrik \[ I = (x'^2 + y'^2)du^2 + dv^2. \]
		Insbesondere: Ist $ u \mapsto O+e_1x(u) + e_2y(u) $ bogenlängenparametrisiert, so ist $ X $ \emph{isometrisch},
		\[I = du^2 + dv^2\]
		
		\item Das \emph{Helicioid} \[ (r,v) \mapsto O + e_1r\cos(v)+e_2r\sin(v) + e_3v \]
		hat Metrik \[ I\big|_{(r,v)}=dr^2 + (1+r^2)dv^2. \]
		Mit einer Umparametrisierung $r = r(u) = \sinh(u)$ erhält man \[ I \big|_{(u,v)}= \cosh^2(u)(du^2+dv^2), \] d.h., $X$ wird \emph{konform} (winkeltreu).
		
	\end{enumerate}
	
\end{example}

\begin{definition}
	
	Eine Parametrisierte Fläche $ X : M \rightarrow \E $ heißt 
	\begin{enumerate}
		\item \emph{konform}, falls $ E = G, F = 0 $
		\item \emph{isometrisch}, falls $ E = G = 1, F= 0 $.
	\end{enumerate}
	
\end{definition}


\begin{remark}
	
	Für Kurven: Jede Kurve kann isometrisch/nach Bogenlänge umparametrisiert werden(\ref*{umpar}).  Für Flächen: Im Allgemeinen gibt es keine isometrische (Um-)Parametrisierung.
	
\end{remark}

\begin{theorem}
	
	Jede Fläche kann lokal konform (um-)parametrisiert werden. 
	
\end{theorem}

\begin{proof}
	
	Ist echt cool laut Jeromin (braucht bissi so Fana und so \ldots).
	Falls der Leser Zeit hat, wird ihm nahegelegt den Beweis nachzuschauen.
	Hier ein Link zu einem Beweis dieser Tatsache:

 	\href{https://thibaultlefeuvre.files.wordpress.com/2017/02/coord_isotherm.pdf}{Link zu PDF hier klicken.}
 	
	\end{proof}

\begin{remark}
	
	Dieser Satz ist die Grundlage, um (reelle) Flächen als komplexe Kurve zu intepretieren. Eine weitreichende Betrachtungsweise \ldots
\end{remark}

\begin{remark}
	
	Um den Satz zu verstehen:\\
	''lokal'' heißt, dass -- für jeden Punkt $ (u,v) \in M $ -- der Definitionsbereich $M$ so eingeschränkt werden kann -- auf eine Umgebung des Punktes$ (u,v) $ -- dass die Behauptung wahr ist;
	''Umparametrisierung'' wie für Kurven definiert:
\end{remark}

\begin{definition}
	
	Eine \emph{Umparametrisierung} einer parametrisierten Fläche $ X: M \rightarrow \E $ ist eine neue parametrisierte Fläche 
		\[ \widetilde{X}=X\circ(u,v), \quad \widetilde{M} \rightarrow \E, \]
	mit einem \emph{Diffeomorphismus}: 
		\[ (u,v): \widetilde{M} \rightarrow M, \] 
	d.h., eine glatte $ (C^\infty) $ Bijektion mit glatter Inverser $ (u,v)^{-1}:M \rightarrow \widetilde{M}. $
		
\end{definition}

\begin{remark}
	
	Für \[(x,y)\mapsto \widetilde{X}(x,y) = X(u(x,y),v(x,y)) \in \E^3  \] gilt (Kettenregel) \[ \widetilde{X}_x = (X_u\circ (u,v))\cdot u_x + (X_v \circ (u,v)) \cdot v_x \]
	\[ \widetilde{X}_y = (X_u \circ (u,v))\cdot u_y + (Y_v \circ (u,v)) \cdot v_y \]
	und somit 
		\[ \widetilde{X}_x \times \widetilde{X}_y = ((X_u\times X_v)\circ (u,v))\cdot (u_xv_y - u_yv_x), \]
	d.h., $ \widetilde{X} $ ist regulär.
	
\end{remark}

\subsection{Gaußabbildung und Weingartentensor}

\begin{definition}
	
	Eine Fläche $ X : M \rightarrow \E^3 $ hat an jedem Punkt $ X(u,v) $ eine \emph{Tangentialebene} und eine \emph{Normalgerade}:
		\[ \mathcal{T}(u,v) := X(u,v) + [ \{ X_u(u,v), X_v(u,v) \} ], \]
		\[ \mathcal{N}(u,v) := X(u,v) + [ \{ (X_u \times X_v)(u,v) \} ]; \]
	dies entspricht einer orthogonalen Zerlegung
		\[ V= [ \{ X_u(u,v),X_v(u,v) \} ] \oplus_\perp [ \{ (X_u \times X_v)(u,v) \} ] \]
	von $ V $ in einen \emph{Tangentialraum} und einen \emph{Normalraum} von $X$ am Punkt $X(u,v).$
	
\end{definition}
	
\begin{definition}
	
	Das \emph{Tangential-} bzw. \emph{Normalenbündel} einer Fläche $ X: M \rightarrow \E $ ist gegeben durch die Abbildung 
		\[ (u,v) \mapsto T_{(u,v)}X := [ \{X_u(u,v),X_v(u,v)\} ], \]
		\[ (u,v) \mapsto N_{(u,v)}X := \{ X_u(u,v),X_v(u,v) \}^{\perp}. \]
	Eine Abbildung $ Y : M \rightarrow V $ heißt 
	
	\begin{itemize}
		
		\item \emph{Tangentialfeld} entlang $ X $, falls 
			\[ \forall (u,v) \in M: ~ Y(u,v) \in T_{(u,v)}X \],
			
		\item \emph{Normalenfeld} entlang $ X $, falls
			\[ \forall(u,v) \in M: ~ Y(u,v) \in N_{(u,v)}X. \]
		
	\end{itemize}

	Die \emph{Gaußabbildung} einer Fläche $ X: M \rightarrow \E^3 $ ist das Einheitsnormalenfeld (ENF):
		\[ N:= \dfrac{X_u \times X_v}{\abs{X_u \times X_v}}: M \rightarrow V. \]
	
\end{definition}
	
\begin{example}
	\textbf{Roationsfläche:}
	Für jede Rotationsfläche
		\[ (u,v) \mapsto X(u,v) := O + e_1r(u)\cos(v) + e_2 r(u) \sin(v) + e_3h(u) \]
	ist jede Profilkurve $ v \equiv \mathrm{const} $ der orthogonale Schnitt der Fläche mit der Ebene $ x\sin(v) = y \cos(v) $ der Meridiankurve; die Gaußabbildung erhält man also durch $ \dfrac{\pi}{2} $ Drehung des ETFs (''Einheitstangentialfeldes'') in der Ebene der Kurve 
		\[ N(u,v) = \{ -(e_1\cos(v) + e_2\sin(v))h'(u)+e_3r'(u) \} \dfrac{1}{\sqrt{(r'^2 + h'^2)}(u)}. \]
	Überprüfung des Vorzeichens:
	\[ \det \pmat{r'\cos & -r\sin & -h'\cos \\ r'\sin & r\cos & -h'\sin \\ h' & 0 & r'} = h'^2r + r'^2r = r(r'^2 + h'^2) > 0. \]
	
\end{example}
	
\begin{remark}
	
	Die Gaußabbildung einer Fläche ist ein geometrisches Objekt, d.h.,
	nach einer Euklid. Bewegung $ \widetilde{X} = \widetilde{O} + A(X-O), $ liefert
		\[ \widetilde{N}=\dfrac{\widetilde{X}_u \times \widetilde{X}_v}{|\widetilde{X}_u \times \widetilde{X}_v|} = \dfrac{AX_u \times A X_v}{\abs{AX_u \times AX_v}} = \dfrac{A(X_u \times X_v)}{\abs{A(X_u \times X_v)}} = A \dfrac{X_u \times  X_v}{\abs{X_u \times X_v}} = AN. \]
	Das vorletzte Gleichheitszeichen gilt, weil für $A \in \mathrm{SO}(3)$ gilt $ \abs{A}=1 $. 
	Eine Spiegelung liefert $ \widetilde{N}~=~-AN, $ d.h., wechselt das Vorzeichen -- was auch eine (ordnungsumkehrende) Umparametrisierung tut, z.B.: $ (u,v) \mapsto(v,u) $.
	Demnach ist $ N $ ''geometrisch'' bis auf Vorzeichen.
	
\end{remark}

\begin{remark}
	
	Ordnungsprobleme tauchen in unserem Setting mit parametrisierten (!) Flächen nicht auf: Die Gaußabbildung einer parametrisierten Fläche ist wohldefiniert; eine nicht-orientierbare Fläche (e.g. Möbiusband) kann durch eine doppelt überlagerte Parametrisierung beschrieben werden.
	
\end{remark}


\textbf{Erinnerung:} Die Normalkrümmung $ \kappa_n $ eines Streifens $(X,N) $ ist definiert durch 
	\[ 0=N'^T + T\abs{X'} \kappa_n = N'^T + X'\kappa_n, \]
wobei $ t \mapsto N'^T(t) \in T_tX $ den Tangentialanteil von $ N' $ bezeichnet, i.e.
	\[ N'^T = N' - \nabla^\perp N = T\skal{T,N'}. \]

\begin{lemma, definition}
	
	Die Ableitung der Gaußabbildung ist tangentialwertig,
	\[ \forall (u,v) \in M : ~ d_{(u,v)}N : \R^2 \rightarrow T_{(u,v)}X. \]
	Damit können wir den \emph{Formoperator} von $ X $ am Punkt $ (u,v) \in M $ definieren:
	\[ \s\big|_{(u,v)}:= - d_{(u,v)}N \circ (d_{(u,v)}X)^{-1} \in \mathrm{End(T_{(u,v)}X)}. \]
	
\end{lemma, definition}

\begin{proof}
	
	\begin{enumerate}
		
		\item Die Ableitung von N ist tangentialwertig. Nämlich:
			\[ 1 \equiv \abs{N}^2 \Rightarrow 0 = 2 \skal{N,dN} \Rightarrow \forall (u,v) \in M \forall \pmat{x \\y} \in \R^2 : d_{(u,v)}N(\pmat{x \\y}) \in T_{(u,v)X}. \]
		
		\item  $\s$ ist wohldefiniert:
		Da für $ (u,v) \in M, d_{(u,v)}X: \R^2 \rightarrow V $ injektiv ist, liefert dies einen Isomorphismus
			\[ d_{(u,v)}X: \R^2 \rightarrow T_{(u,v)}X \subseteq V, \]
		der invertiert werden kann, um eine lineare Abbildung
		
			\[ (d_{(u,v)}X)^{-1}: T_{(u,v)}X \rightarrow \R^2 \]
		
		zu erhalten.
		
		\item $ \s\big|_{(u,v)} $ ist Endomorphismus:
		Als Verkettung linearer Abbildungen
			\[ T_{(u,v)}X \xrightarrow{(d_{(u,v)}X)^{-1}} \R^2 \xrightarrow{-d_{(u,v)}N} T_{(u,v)}X \]
		
	\end{enumerate}
	
\end{proof}

\begin{remark}
	Die Abbildung $\mapsto \s \big|_{(u,v)} \in End(T_{(u,v)}X)$ liefert ein Endomorphismenfeld $\s$, welches man auch \emph{Weingartentensorfeld} nennt.
\end{remark}

\begin{remark}
	Da $\left( X_u(u,v) , X_v(u,v) \right)$ eine Basis von $T_{(u,v)}X$ ist, kann $\s \big|_{(u,v)}$ durch die Werte auf der Basis bestimmt werden:
	\[ \s X_u = -dN \circ (dX)^{-1}(X_u)= 
	-dN \circ \pmat{1\\0} = -N_u \]
	und 
	\[ \s X_v = -N_v. \]
\end{remark}

\begin{lemma}
	$\s \big|_{(u,v)} \in End(T_{(u,v)}X)$ ist symmetrisch für jedes $(u,v) \in M$.
\end{lemma}

\begin{proof}
	Wir verifizieren Symmetrie auf der Basis $(X_u(u,v),X_v(u,v))$ von $T_{(u,v)}X$:
	
	Da $N \perp X_u,X_v$ erhalten wir 
	\[ 0=\skal{X_u,N}_v = \skal{X_{uv}, N} + \skal{X_u,N_v}
	= \skal{X_{vu},N}- \skal{X_u,\s X_v}. \]
	Ebenfalls
	\[ 0=\skal{X_v,N}_u 
	= \skal{X_{vu},N}- \skal{X_v, \s X_u}. \]
	Also 
	\[ \skal{X_u,\s X_v} = \skal{\s X_u,X_v}. \]
\end{proof}


\begin{remark}
	Wie bei Streifen kann der Formoperator analog zu $\kappa_n$ durch die Gleichung 
	\[ 0=dN + \s \circ dX = dN^T + S\circ dX \] beschrieben werden.
	Der Formoperator ''kodiert'' also die Krümmung einer Fläche.
\end{remark}

\begin{definition}
	Sei $\s$ der Formoperator der Fläche $X : M \to \E^3$, dann heißt
	\begin{itemize}
		\item $H=\frac 12 \mathrm{tr} \s $ die \emph{mittlere Krümmung} von $X$,
		\item $K=\det S$ die \emph{Gauß Krümmung} von $X$ und 
		\item die Eigenwerte $\kappa^\pm = H \pm \sqrt{H^2-K}$ und die \emph{Eigenrichtungen} von $\s$ sind die \emph{Hauptkrümmungen} bzw. \emph{Hauptrichtungen} von $X$. 
	\end{itemize}
\end{definition}

\begin{remark} Es gilt
	$H=\frac 12 (\kappa^+ +\kappa^-)$ -- daher auch der Name \emph{mittlere Krümmung}.
\end{remark}

\begin{example}
	Eine Rotationsfläche parametrisiert nach Bogenlänge ist
	\[ X(u,v)= O+e_1r(u)\cos v +e_2 r(u) \sin v + e_3h(v), \]
	mit $r'^2+h'^2=1$. Damit folgt $r'r''+h'h''=0$. Mit der Gaußabbildung
	\[ N(u,v)= -e_1h'(u)\cos v - e_2h'(u) \sin v + e_3v'(u) \]
	bekommt man
	\[ N_v + X_v\frac {h'}r = (e_1\sin v - e_2\cos v)\left( h' - r\frac {h'}r \right)=0, \]
	\[ N_u + X_u\left( r'h''-r''h' \right)\]
	\[= (e_1 \cos v +e_2\sin v ) \left( h''+r'(r'h''-r''h') \right) + e_3 (r''+h'(r'h''-r''h'))
	= 0. \]
	Also liefern $X_u$ und $X_v$ Krümmungsrichtungen zu Hauptrichtungen
	\[ \kappa^+ = r'h''-r''h' \quad \text{ und }\quad
	\kappa^- = \frac {h'}r. \]		
\end{example}


\begin{remark}
	Formoperatoren und Krümmungen sind geometrische Objekte:
	\begin{itemize}
		\item Ist $\tilde X = X \circ (u,v)$ eine Umparametrisierung und $\tilde N = N \circ (u,v)$ so ist 
		\[ \tilde{\s} = -d\tilde N \circ (d\tilde X)^{-1} = -\left( d_{(u,v)}N \circ d(u,v) \right) \circ \left( d_{(u,v)X} \circ d(u,v)  \right)^{-1} \]
		\[ = -d_{(u,v)}N \circ d(u,v) \circ d(u,v)^{-1} \circ \left( d_{(u,v)}X \right)^{-1}
		= \s\big |_{(u,v)}, \]
		also insbesondere
		\[ \tilde H = H \circ (u,v), \quad \text{ und }\quad
		\tilde K = K \circ (u,v), \quad \text{ etc.} \]
		
		\item Ist $\tilde X=\tilde O+A(X-O)$ mit $A \in SO(3)$ so bekommt man 
		\[ \tilde{\s}= -d\tilde N \circ \left( d\tilde X \right)^{-1}
		= -A\circ dN \circ (dX)^{-1} \circ A^{-1}
		= A\circ \s \circ A^{-1}, \]
		insbesondere also
		\[ \tilde H = H, \quad \text{ und }\quad
		\tilde K = K, \quad \text{ etc.} \]
		Die Krümmungsrichtungen werden mit der Fläche gedreht:
		\[ \ker (\mathrm{id} \kappa^\pm -\tilde{\s}) = A\ker (\mathrm{id} \kappa^\pm-\s). \]
	\end{itemize}
\end{remark}	
	
\begin{definition}
	
	Ein Punkt $X(u,v)$ einer Fläche heißt 
	\begin{itemize}
		
		\item \emph{Nabelpunkt} (umbilic), falls $\kappa^+(u,v) = \kappa^-(u,v) \quad (\Longleftrightarrow (H^2- K)(u,v)=0), $
		\item \emph{Flachpunkt} (flatpoint), falls $\kappa^+(u,v) = \kappa^-(u,v)=0.$
		
	\end{itemize}

	
\end{definition}

\begin{remark}
	
	Ein Punkt $X(u,v)$ ist Nabelpunkt bzw. Flachpunkt, falls
	\[ \mathcal{S}\big|_{(u,v)} = H(u,v)\mathrm{id}_{T_{(u,v)}X} \text{ bzw. } \mathcal{S}\big|{(u,v)}=0. \]
	
\end{remark}

\begin{example}
	
	Falls $  (u,v) \mapsto X(u,v) $ Werte in einer (festen) Ebene
		\[ \pi = \{ X \in \E^3 ~|~ \skal{X-0,n} = d \} \] 
	annimmt, d.h., $ \skal{dX,n}=0 $. Also ist $ N \equiv \pm n $, weshalb $\s \equiv 0$ und jeder Punkt der Fläche ist ein Flachpunkt.\\
	Umgekehrt: Ist jeder Punkt einer Fläche $ X $ ein Flachpunkt, $ \s \equiv 0 $, so folgt $ N \equiv \mathrm{const.} $, also nimmt $ X $ Werte in einer (festen) Ebene an: Mit $ O \in \E^3 $ beliebig \[ 0 = \skal{dX,N} = d\skal{X-O,N} - \skal{X-O,dN} \Rightarrow \mathrm{const.} \equiv \skal{X-O,N}, \] wobei $ \skal{dX,N} $ eine Abbildung $ ((u,v), \pmat{x \\ y}) \mapsto \skal{d_{(u,v)}X\pmat{x\\y},N(u,v)} $ ist.

	\textbf{Matrixdarstellung}: Mit $(X_u,X_v)$ als tangentiales Basisfeld kann der Weingartentensor als Matrix geschrieben werden:
	$ (\s X_u, \s X_v) = (-N_u, -N_v) = (X_u,X_v) \pmat{s_{11} & s_{12} \\ s_{21} & s_{22}} $
	
	also, vermittels der Skalarprodukte der Basisvektoren $ X_u,X_v $, 
	\[\pmat{-\skal{X_u,N_u} & - \skal{X_u,N_v} \\ - \skal{X_v,N_u} & - \skal{X_v,N_v} } = \pmat{E &F \\ F &G} \pmat{s_{11} & s_{12} \\ s_{21} & s_{22}}, \] d.h., als die Gram Matrix der zweiten Fundamentalform:
	
	
\end{example}

\begin{lemma, definition}
	 Die \emph{zweite Fundamentalform} einer Fläche $ X: M \rightarrow \E^3 $ ist definiert als \[ \mathrm{II} := \skal{dX,dN}, \] wobei $ \mathrm{II}: ((u,v), \pmat{x \\ y},\pmat{\widetilde{x} \\ \widetilde{y}}) \mapsto \skal{d_{(u,v)}X \pmat{x\\y},d_{(u,v)}N \pmat{\widetilde{x}\\\widetilde{y}}}; $
	 an jedem Punkt $ (u,v) $ erhält man so eine symmetrische Bilinearform 
	 
	 \[ \mathrm{II}\big|_{(u,v)}: \R^2 \times \R^2 \rightarrow \R, \quad (\pmat{x_1 \\ y_1},\pmat{x_2 \\ y_2}) \mapsto - \skal{d_{(u,v)}X\pmat{x_1 \\ y_1},d_{(u,v)}N\pmat{x_2 \\ y_2}}. \]
	 
\end{lemma, definition}

\begin{proof}
	Dass $ \mathrm{II}\big|_{(u,v)} $ Bilinearform ist, ist klar. Weiters gilt mit der Leibniz-Regel: 
		\[ {\color{ForestGreen} \mathrm{II}(\pmat{1 \\ 0},\pmat{0 \\ 1}) = }-\skal{X_u,N_v}= \skal{X_{uv},N}= \skal{X_{vu},N} {=} -\skal{X_v,N_u} {\color{ForestGreen} = \mathrm{II}(\pmat{0 \\ 1},\pmat{1 \\ 0}) }, \] 
	also ist $ \mathrm{II} $ symmetrisch an jedem Punkt. 
\end{proof}

\begin{remark}
	
	Ist die erste Fundamentalform gegeben, so kann die zweite Fundamentalform aus dem Weingartentensor berechnet werden - und umgekehrt.
	
	\textbf{Warnung:} Obwohl der Weingartentensor symmetrisch ist, ist seine Darstellungsmatrix (bzgl. $ (X_u,X_v) $) normalerweise \textbf{nicht} symmetrisch:
	
	\[ \pmat{s_{11} & s_{12} \\ s_{21} & s_{22}} = \frac{1}{EG-F^2} \pmat{G & -F \\ -F & E} \pmat{e & f \\ f & g} = \frac{1}{EG-F^2} \pmat{Ge-Ff & Gf-Fg \\ Ef-Fe & Eg-Ff} \]
	
\end{remark}

\begin{remark}
	Die Gauß-Krümmung ist gegeben durch: \[ K = \frac{eg-f^2}{EG-F^2} \]
\end{remark}

\subsection{Kovariante Ableitung und Krümmungstensor}

\begin{definition}
	
	Die \emph{kovariante Ableitung} eines Tangentialfeldes $ Y:M\rightarrow V $ entlang $ X:M\rightarrow \E $ ist der Tangentialteil seiner Ableitung 
		\[ \nabla Y := (dY)^T = dY - \skal{dY,N}N, \] wobei $ \nabla $ den \emph{Levi-Civita Zusammenhang} entlang $ X $ bezeichnet.
	
\end{definition}

\begin{remark}
	
	Wegen $ \skal{X_{uu},N}= -\skal{X_u,N_u}= e $ usw. bekommt man 
		\[ \nabla_{\dfrac{\partial}{\partial u}}X_u = X_{uu}-Ne \text{ und } \nabla_{\dfrac{\partial}{\partial v}} = X_{uv}-Nf \]
		\[ \nabla_{\dfrac{\partial}{\partial v}}X_v = X_{vv}-Ng \text{ und } \nabla_{\dfrac{\partial}{\partial u}} = X_{uv}-Nf \]
	
	Wobei $$ \nabla_{\dfrac{\partial}{\partial u}}Y := Y_u - \skal{Y_u,N}N $$
	
\end{remark}

\begin{lemma}
	
	Der L-C Zusammenhang erfüllt die Leibniz-Regel,
		\[ \nabla(Yx) = (\nabla Y)x + Ydx \text{ für } x \in C^\infty(M) \] 
	und ist \emph{metrisch}, d.h., 
		\[ d\skal{Y,Z} = \skal{\nabla Y,Z} + \skal{Y,\nabla Z} \]
	
\end{lemma}

\begin{proof}
	
	Die Leibniz-Regel gilt, da für $x \in C^\infty(M)$:
		\[ \nabla(Yx) = dYx + Ydx- \skal{dYx,N}N - \skal{Ydx,N}N = (\nabla Y)x + Ydx, \]
	weil $ Y $ ein Tangentialfeld ist, ist $ \skal{Ydx,N} = 0. $ 
	
	Er ist metrisch, da:
		\[  \skal{\nabla Y,Z} + \skal{Y,\nabla Z} = \skal{dY,Z} -\skal{dY,N}\skal{N,Z} + \skal{Y,dZ} - \skal{Y,N}\skal{dZ,N} \]
		\[ = \skal{dY,Z} + \skal{Y,dZ} = d\skal{Y,Z}. \]
	
\end{proof}

\textbf{Matrixdarstellung:} Da die kovariante Ableitung eines TVFs $ Y $ tangential ist, erhält man für $ Y = X_v :$

\begin{equation} \label{Eq:Gamma}
\begin{split}
		 \left(\nabla_{\dfrac{\partial}{\partial u}}X_u,\nabla_{\dfrac{\partial}{\partial u}}X_v \right) &= (X_u,X_v)\Gamma_1  \\
		 \left(\nabla_{\dfrac{\partial}{\partial v}}X_u,\nabla_{\dfrac{\partial}{\partial v}}X_v \right) &= (X_u,X_v)\Gamma_2 
\end{split}
\end{equation} 
	mit 
		\[ \Gamma_i = \pmat{\Gamma_{i1}^1 & \Gamma_{i2}^1 \\ \Gamma_{i1}^2 & \Gamma_{i2}^2}. \]
		
	Also gilt für eine allgemeine TVF $ Y=X_ux+X_vy=dX\pmat{x \\ y} $, 
		\[ \nabla_{\dfrac{\partial}{\partial u}}Y= \nabla_{\dfrac{\partial}{\partial u}}\left ( \left( X_u,X_v \right) \pmat{x \\y}    \right ) = \left( \nabla_{\dfrac{\partial}{\partial u}}X_u,\nabla_{\dfrac{\partial}{\partial u}}X_v \right) \pmat{x \\y} + X_u,X_v \pmat{x_u \\ y_u} \]

		\[ = \left( X_u, X_v \right) \Gamma_1\pmat{x \\ y} + \left( X_u, X_v \right) \pmat{x_u \\ y_u} \] 
		
		\begin{equation} \label{Eq:Gammaallg}
		\begin{split}
			&= \left( X_u, X_v \right) \left( \dfrac{\partial}{\partial u} +\Gamma_1 \right) \pmat{x \\ y}  \quad \text{ und ebenso }
		 \\ 
		 \nabla_{\dfrac{\partial}{\partial v}}Y &= \left( X_u, X_v \right) \left( \dfrac{\partial}{\partial v} +\Gamma_2 \right) \pmat{x \\ y}, 
		 \end{split}
		\end{equation} 
	oder, anders gesagt:
		\[ \nabla_{\dfrac{\partial}{\partial u}} \circ dX = dX \circ \left( \dfrac{\partial}{\partial u} + \Gamma_1 \right)  \]
	und genauso
		\[  \nabla_{\dfrac{\partial}{\partial v}} \circ dX = dX \circ \left( \dfrac{\partial}{\partial v} + \Gamma_2 \right).   \]	
	
	Man bemerke: mit $ \pmat{x \\ y} = \pmat{1 \\ 0} \cong \dfrac{\partial}{\partial u} $ erhält man
		\[ \nabla_{\dfrac{\partial}{\partial u}}\left( \left( X_u,X_v \right) \pmat{1 \\ 0} \right) = \nabla_{\dfrac{\partial}{\partial u}}X_u = X_u \Gamma_{11}^1 + X_v\Gamma_{11}^2 = \left( X_u,X_v \right) \Gamma_1\pmat{1 \\ 0}   \] 
	und
		\[ \nabla_{\dfrac{\partial}{\partial v}} \left( \left( X_u,X_v \right) \pmat{1 \\ 0} \right) = \nabla_{\dfrac{\partial}{\partial v}}X_u = \left( X_u,X_v \right) \Gamma_2\pmat{1 \\ 0} \]
	und ebenso für  $ \pmat{x \\ y} = \pmat{0 \\ 1} \cong \dfrac{\partial}{\partial v} $, also ist \ref*{Eq:Gamma} ein Spezialfall von \ref*{Eq:Gammaallg}
	
	\begin{remark}
		\textbf{WARUNUNG:} $ \nabla_{\dfrac{\partial}{\partial u}} $ und $ \nabla_{\dfrac{\partial}{\partial v}} $ sind \emph{Differentialoperatoren} (keine Endomorphismen) - trotzdem benutzen wir Matrizen, um sie zu beschreiben.
	\end{remark}

\begin{lemma, definition}
	
	$\Gamma_{ij}^k$ heißen die \emph{Christoffel-Symbole} von $ X $; sie sind symmetrisch: 
		\[ \Gamma_{ij}^k = \Gamma_{ji}^k \]
	
\end{lemma, definition}

\begin{proof}
	
		\[ X_u\Gamma_{12}^1 + X_v \Gamma_{12}^2 = \nabla_{\dfrac{\partial}{\partial u}}X_v = \left( X_{vu} \right)^T = \left( X_{uv} \right)^T = \nabla_{\dfrac{\partial}{\partial v}}X_u = X_u\Gamma_{21}^1 + X_v\Gamma_{21}^2   \]
	also $ \Gamma_{12}^1 = \Gamma_{21}^1 $ und $ \Gamma_{12}^2 = \Gamma_{21}^2 $.
\end{proof}

\begin{theorem}[Koszul's Formeln]
	
	Mit Matrizen 
		\[ I = \pmat{E &f \\ F &G} \text{ und } J = \pmat{0 &-1 \\ 1 & 0 } \]
	gilt: 
		\[ \frac12 I_u - \frac{E_v - F_u}{2}J = I\Gamma_1, \]
		\[ \frac12 I_v + \frac{G_u - F_v}{2}J = I\Gamma_2. \]
		
\end{theorem}

\begin{proof}
	
	Multipliziere $ \nabla_{\dfrac{\partial}{\partial u}}X_u = X_u\Gamma_{11}^1 + X_v\Gamma_{11}^2 $ mit $ Y_u $ und $ X_v $, um zu erhalten:
		\[ E \Gamma_{11}^1 + F \Gamma_{11}^2 = \skal{X_u, \nabla_{\dfrac{\partial}{\partial u}}X_u} = \frac12 \skal{X_u,X_u}_u = \frac12 E \]
	
		\[ F\Gamma_{11}^1 + G \Gamma_{11}^2 = \skal{X_v, \nabla_{\dfrac{\partial}{\partial u}}X_u} = \skal{X_v,X_u}_u - \skal{\nabla_{\dfrac{\partial}{\partial u}}X_v,X_u} = \skal{X_v,X_u}_u - \skal{\nabla_{\dfrac{\partial}{\partial v}}X_u,X_u} \]
		\[ = F_u - \frac12 E_v. \]
	
	Insgesamt also
		\[ \pmat{E &F \\F &G} \pmat{\Gamma_{11}^1 & \Gamma_{12}^1 \\ \Gamma_{11}^2 & \Gamma_{12}^2} = \frac12 \pmat{E_u \\ F_u} - \dfrac{E_v - F_u}{2} \pmat{0 \\ 1} \]
		\[ = \left( \frac12 \pmat{E_u & F_u \\ F_u & G_u} - \dfrac{E_v - F_u}{2} \pmat{0 &-1 \\ 1 & 0 } \right).  \]
	
	Die anderen Teile der Gleichung  folgen ebenso.
	
\end{proof}

\begin{corollary}
	Die Christoffel - Symbole $ \Gamma_{ij}^k $ hängen nur von der induzierten Metrik ab.
\end{corollary}

\begin{example}
	
	Für eine isometrische Parametrisierung, $  E = G = 1 $ und $  F = 0 $, erhält man mit den Konszul-Formeln $ \Gamma_{ij}^k = 0 $.
	
\end{example}


\begin{definition}
	
	Für ein TVf $ Y : M \rightarrow V $ entlang $ X: M \rightarrow \E $ definieren wir den \emph{(Riemannschen) Krümmungstensor} $ R $ von $X$ durch 
		\[ RY := \nabla_{\dfrac{\partial}{\partial u}}\nabla_{\dfrac{\partial}{\partial v}}Y - \nabla_{\dfrac{\partial}{\partial v}}\nabla_{\dfrac{\partial}{\partial u}}Y. \]
	
\end{definition}

\begin{remark}
	
	Dies ist eine vereinfachte Form, für Flächen, des ''wahren'' Krümmungstensors.
	
\end{remark}

\begin{lemma}
	
	$ R $ ist ein schiefsymmetrischer Tensor des Tangentialbündels $ TX $, d.h., 
	$R\big|_{(u,v)} \in \mathrm{End}(T_{(u,v)}X) $ ist schiefsymmetrisch für jedes $ (u,v) \in M $, und 
		\[ R(Yx) = (RY)x \text{ für } x \in C^\infty(M). \]
	
\end{lemma}

\begin{proof}
	
	$R\big|_{(u,v)} \in \mathrm{End(T_{(u,v)}X)}$ ist klar.
	Schiefsymmetrie:
		\[\frac12 (\abs{Y}^2)_{uv} = \skal{Y \nabla_{\dfrac{\partial}{\partial u}}Y}_v = \skal{Y, \nabla_{\dfrac{\partial}{\partial v}}\nabla_{\dfrac{\partial}{\partial u}}Y} + \skal{\nabla_{\dfrac{\partial}{\partial v}}Y,\nabla_{\dfrac{\partial}{\partial u}}Y} \]
		\[ = \skal{Y, \nabla_{\dfrac{\partial}{\partial u}}\nabla_{\dfrac{\partial}{\partial v}}Y} + \skal{\nabla_{\dfrac{\partial}{\partial u}}Y, \nabla_{\dfrac{\partial}{\partial v}}Y} = \frac12 \left( \abs{Y}^2\right)  \]
		\[ \Rightarrow \skal{Y,RY} = \frac12 \left(\abs{Y}^2\right)_{vu}- \frac12\left(\abs{Y}^2\right)_{uv}=0 \]
	Tensoreigenschaft:
		\[ \nabla_{\dfrac{\partial}{\partial u}}\nabla_{\dfrac{\partial}{\partial v}}(Yx)= \nabla_{\dfrac{\partial}{\partial u}}\left(\left( \nabla_{\dfrac{\partial}{\partial v}}Y \right)x + Yx_v\right) \]
		\[ = \left( \nabla_{\dfrac{\partial}{\partial u}} \nabla_{\dfrac{\partial}{\partial v}} Y \right)x + \underbrace{\left( \nabla_{\dfrac{\partial}{\partial v}}Y \right)x_u + \left( \nabla_{\dfrac{\partial}{\partial u}}Y \right)x_v + Yx_{uv}}_{\text{symmetrisch in } \dfrac{\partial}{\partial u} \text{ und } \dfrac{\partial}{\partial v}} \]
	also
		\[ R(Yx)=(RY)x. \]
	
\end{proof}
