
\subsection{Parametrisierung \& Metrik}
\begin{definition}
	

Eine Abbildung \[ X:\R^2 \supseteq M \rightarrow \E \] heißt \emph{parametrisierte Fläche}, falls $M$ ein offen und zusammenhängend ist und $X$ regulär ist, d.h., $  \forall(u,v) \in M~:~d_{(u,v)}X: \R^2 \rightarrow V $ ist injektiv.
Wir sagen auch: $  X  $ ist \emph{Parametrisierung} der \emph{Fläche} $ X(M) \subseteq \E $

\end{definition}

\begin{remark}
	
 Äquivalent zur letzten Forderung ist die Forderung, dass die Jacobi-Matrix maximalen Rang hat. Diese braucht aber eine Festgelegte Basis, was oft zu Schwierigkeiten bei Berechnungen führt und wird daher von Prof. Jeromin nicht empfohlen.

\end{remark}

\begin{remark}
	Einmal mehr sind alle geforderten Abbildungen so oft differenzierbar, wie wir das wünschen
\end{remark}

\begin{remark}
	
	$ d_{(u,v)} : \R^2 \rightarrow V$ ist die Ableitung am Punkt $ (u,v) \in M $, \[X(u+x,v+y) \approx X(u,v) + d_{(u,v)}  X(\begin{pmatrix}
	x\\
	y
	\end{pmatrix}) = X(u,v) + X_u(u,v)\cdot x + X_v(u,v)\cdot y,  \]
	wir können also identifizieren: \[ d_{(u,v)}X \cong (X_u,X_v)(u,v), \]
	bzw., nach Wahl einer Basis von V, mit der Jacobi-Matrix am Punkt $ (u,v) $.
	
\end{remark}

\begin{example}
	


Ein \emph{Helicoid} $ X: \R^2 \rightarrow \E^3 $ ist die \emph{(Regel-)Fläche}  
\[ \R^2 \supseteq (r,v) \mapsto O + e_1r \cos(v) + e_2 r \sin(v) + e_3 v \in \E^3. \]

Wir zeigen, dass $(X_r,X_v)(r,v))$ linear unabhängig für alle $ (r,v) \in \R^2 $ sind:

\[ X_r(r,v) = e_1\cos(v) + e_2 \sin(v) \neq 0 \]
\[ X_v(r,v) = -e_1r \sin(v) + e_2 r \cos(v) + e_3 \neq 0 \]

und da $ X_r(r,v) $ von $e_3 \neq 0$  abhängt sind die beiden linear unabhängig.

\end{example}

\begin{example}
	
	Eine übliche Parametrisierung von $ \mathbb{S}^2 \subseteq \E^3 $ (mit Mittelpunkt $O \in \E^3$) ist  
	\[ (u,v) \mapsto O + e_1\cos(u)\cos(v) + e_2\cos(u)\sin(v)+e_3\sin(u) \]
	liefert keine parametrisierte Fläche, da die Sphäre an den Polen nicht regulär ist.
	Dies ist also nur eine Parametrisierung auf $ M = (-\frac{\pi}{2},\frac{\pi}{2}) \times \R $
	
	Insbesondere kann man sogar zeigen, dass es keine (reguläre) Parametrisierung der (ganzen) Sphäre gibt (''Hairy Ball Theorem'' bzw. ''Satz vom Igel'').
	
	$ \mathbb{S}^2 $ ist also \textbf{keine} Fläche im Sinne der Def. Dieses ''Problem'' wird später gelöst.
\end{example}

\begin{lemma, definition}
	
	Die \emph{induzierte Metrik} oder \emph{erste Fundamentalform} einer parametrisierten Fläche $ X: M \rightarrow \E$ ist definiert durch
	\[ I := \langle dX,dX \rangle \]
	für jeden Punkt $ (u,v) \in M $ liefert 
	\[ \R^2 \times \R^2 \ni \left( \pmat{x_1\\y_1}, \pmat{x_2\\y_2} \right)  \mapsto I \big|_{(u,v)}(\begin{pmatrix}
	x_1 \\
	y_1
	\end{pmatrix},\begin{pmatrix}
	x_2 \\
	y_2
	\end{pmatrix}) := \left\langle d_{(u,v)}X(\begin{pmatrix}
	x_1 \\
	y_1
	\end{pmatrix}),  d_{(u,v)}X(\begin{pmatrix}
	x_2 \\
	y_2
	\end{pmatrix})  \right\rangle \]
	eine positiv definite symmetrische Bilinearform.
\end{lemma, definition}

\begin{proof}
	%Das darf ich wohl machen.. \#getrektluka
	Zu zeigen ist, dass $I\big| _{(u,v)}$ für jeden Punkt $(u,v)\in M$ eine positiv definite symmetrische Bilinearform ist. 
	
	Weil $ I\big| _{(u,v)} $ eine Komposition aus linearen Funktionen und einer Bilinearform ist, ist auch $I\big| _{(u,v)}$ linear. Die Symmetrie ist ebenfalls leicht ersichtlich. Fehlt noch die positive Definitheit.
	
	Sei $\pmat{x\\y} \neq 0$ beliebig, so gilt
		\[ I\big| _{(u,v)}(\pmat{x\\y},\pmat{x\\y})
			= \skal{d_{(u,v)}X(\pmat{x\\y}),d_{(u,v)}X(\pmat{x\\y})}> 0. \]
	Die letzte Ungleichung gilt, weil $d_{(u,v)}X$ injektiv und linear ist, daher bildet nur $0$ auf $0$ ab. Also ist $d_{(u,v)}X(\pmat{x\\y})\neq 0$. Der Rest folgt, weil $\skal{.,.}$ positiv definit ist.
\end{proof}

\begin{remark}
	
	$I$ wird oft notiert mit Hilfe der \emph{Gramschen Matrix}, \[ I = \begin{pmatrix}
	E & F\\
	F & G
	\end{pmatrix} = E du^2 + 2Fdudv + Gdv^2 \] mit
	\[ E=|X_u|^2, F = \langle X_u,X_v \rangle, G = |X_v|^2  \]
	


Dann gilt für $ (u,v) \in M$ : \[ I \big|_{(u,v)}(\begin{pmatrix}
x_1 \\
y_1
\end{pmatrix},\begin{pmatrix}
x_2 \\
y_2
\end{pmatrix}) := \langle d_{(u,v)}X(\begin{pmatrix}
x_1 \\
y_1
\end{pmatrix}),  d_{(u,v)}X(\begin{pmatrix}
x_2 \\
y_2
\end{pmatrix})  \rangle \] \[= \langle X_u(u,v)x_1 + X_v(u,v)y_1, X_u(u,v)x_2 + X_v(u,v)y_2 \rangle \]
\[ = E(u,v)x_1x_2 + F(u,v)(x_1y_2 + x_2y_1) + G(u,v)y_1y_2 \]
\[ = \begin{pmatrix}
x_1 \\
y_1
\end{pmatrix}^t \begin{pmatrix}
E & F \\ 
F & G
\end{pmatrix}\Bigg|_{(u,v)} \begin{pmatrix}
x_2\\
y_2
\end{pmatrix} \]

\end{remark}

\begin{example}
	
	\begin{enumerate}
		
		\item Ein \emph{Zylinder} $$ (u,v) \mapsto X(u,v) := O + e_1x(u) + e_2y(u) +e_3 $$
		hat induzierte Metrik \[ I = (x'^2 + y'^2)du^2 + dv^2. \]
		Insbesondere: Ist $ u \mapsto O+e_1x(u) + e_2y(u) $ bogenlängenparametrisiert, so ist $ X $ \emph{isometrisch},
		\[I = du^2 + dv^2\]
		
		\item Das \emph{Helicioid} \[ (r,v) \mapsto O + e_1r\cos(v)+e_2r\sin(v) + e_3v \]
		hat Metrik \[ I\big|_{(r,v)}=dr^2 + (1+r^2)dv^2. \]
		Mit einer Umparametrisierung $r = r(u) = \sinh(u)$ erhält man \[ I \big|_{(u,v)}= \cosh^2(u)(du^2+dv^2), \] d.h., $X$ wird \emph{konform} (winkeltreu).
		
	\end{enumerate}
	
\end{example}

\begin{definition}
	
	Eine Parametrisierte Fläche $ X : M \rightarrow \E $ heißt 
	\begin{enumerate}
		\item \emph{konform}, falls $ E = G, F = 0 $
		\item \emph{isometrisch}, falls $ E = G = 1, F= 0 $.
	\end{enumerate}
	
\end{definition}


\begin{remark}
	
	Für Kurven: Jede Kurve kann isometrisch/nach Bogenlänge umparametrisiert werden(\ref{umpar}).  Für Flächen: Im Allgemeinen gibt es keine isometrische (Um-)Parametrisierung.
	
\end{remark}

\begin{theorem}
	
	Jede Fläche kann lokal konform (um-)parametrisiert werden. 
	
\end{theorem}

\begin{proof}
	
	Ist echt cool laut Jeromin (braucht bissi so Fana und so..)
	Falls der Leser Zeit hat, wird ihm nahegelegt den Beweis nachzuschauen.
	
\end{proof}

\begin{remark}
	
	Dieser Satz ist die Grundlage, um (reelle) Flächen als komplexe Kurve zu intepretieren. Eine weitreichende Betrachtungsweise..
\end{remark}

\begin{remark}
	
	Um den Satz zu verstehen:\\
	''lokal'' heißt, dass -- für jeden Punkt $ (u,v) \in M $ -- der Definitionsbereich $M$ so eingeschränkt werden kann -- auf eine Umgebung des Punktes$ (u,v) $ -- dass die Behauptung wahr ist;
	''Umparametrisierung'' wie für Kurven definiert:
\end{remark}

\begin{definition}
	
	Eine \emph{Umparametrisierung} einer parametrisierten Fläche $ X: M \rightarrow \E $ ist eine neue parametrisierte Fläche \[ \widetilde{X}=X\circ(u,v)~;~ \widetilde{M} \rightarrow \E \]
	mit einem \emph{Diffeomorphismus}: \[ (u,v): \widetilde{M} \rightarrow M, \] d.h., eine glatte $ (C^\infty) $ Bijektion mit glatter Inverser $ (u,v)^{-1}:M \rightarrow \widetilde{M} $
		
\end{definition}

\begin{remark}
	
	Für \[(x,y)\mapsto \widetilde{X}(x,y) = X(u(x,y),v(x,y)) \in \E^3  \] gilt (Kettenregel) \[ \widetilde{X}_x = (X_u\circ (u,v))\cdot u_x + (X_v \circ (u,v)) \cdot v_x \]
	\[ \widetilde{X}_y = (X_u \circ (u,v))\cdot u_y + (Y_v \circ (u,v)) \cdot v_y \]
	und somit 
	\[ \widetilde{X}_x \times \widetilde{X}_y = ((X_u\times X_v)\circ (u,v))\cdot (u_xv_y - u_yv_x) \] d.h., $ \widetilde{X} $ ist regulär.
	
\end{remark}



\textbf{HIER FEHLT EINIGES  GANZE VO.. Mach ich heute Abend oder morgen -- FUUUUUUU \dots}

\begin{definition}
	
	Ein Punkt $X(u,v)$ einer Fläche heißt 
	\begin{itemize}
		
		\item \emph{Nabelpunkt} (umbilic), falls $\kappa^+(u,v) = \kappa^-(u,v) (\leftrightarrow (H^2- K)(u,v)=0) $
		\item \emph{Flachpunkt} (flatpoint), falls $\kappa^+(u,v) = \kappa^-(u,v)=0$
		
	\end{itemize}

	
\end{definition}

\begin{remark}
	
	Ein Punkt $X(u,v)$ ist Nabelpunkt oder Flachpunkt, falls
	\[ \mathcal{S}\big|_{(u,v)} = H(u,v)id_{T_{(u,v)}X} \text{ bzw. } \mathcal{S}\big|{(u,v)}=0 \]
	
\end{remark}

\begin{example}
	
	Falls $  (u,v) \mapsto X(u,v) $ Werte in einer (festen) Ebene annimmt,
	\[ \pi = \{ X \in \E^3 ~|~ \skal{X-0,n} = d \} \] d.h., $ \skal{dX,n}=0 $, so ist $ N \equiv \pm n $, demnach also $\S \equiv 0$ und jeder Punkt der Fläche ist Flachpunkt.\\
	Umgekehrt; Ist jeder Punkt einer Fläche $ X $ ein Flachpunkt, $ \s \equiv 0 $, so folgt $ N \equiv \mathrm{const.} $, also nimmt $ X $ Werte in einer (festen) Ebene an: Mit $ O \in \E^3 $ beliebig \[ 0 = \skal{dX,N} = d\skal{X-O,N} - \skal{X-O,dN} \Rightarrow \mathrm{const.} \equiv \skal{X-O,N}, \] wobei $ \skal{dX,N} $ eine Abbildung $ ((u,v), \pmat{x \\ y}) \mapsto \skal{d_{(u,v)}X\pmat{x\\y},N(u,v)} $ ist.

	\textbf{Matrixdarstellung}: Mit $(X_u,X_v)$ als tangentiales Basisfeld kann der Weingartentensor als Matrix geschrieben werden:
	$ (\s X_u, \s X_v) = (-N_u, -N_v) = (X_u,X_v) \pmat{\s_{11} & \s_{12} \\ \s_{21} & \s_{22}} $
	
	also, vermittels der Skalarprodukte der Basisvektoren $ X_u,X_v $, 
	\[\pmat{-\skal{X_u,N_u} & - \skal{X_u,N_v} \\ - \skal{X_v,N_u} & - \skal{X_v,N_v} } = \pmat{E &F \\ F &G} \pmat{s_{11} & s_{12} \\ s_{21} & s_{22}}, \] d.h., die Gram Matrix der zweiten Fundamentalform:
	
\end{example}

\begin{lemma, definition}
	 Die \emph{zweite Fundamentalform} einer Fläche $ X: M \rightarrow \E^3 $ ist definiert als \[ \mathrm{II} := \skal{dX,dN}, \] wobei $ \mathrm{II}: ((u,v), \pmat{x \\ y},\pmat{\widetilde{x} \\ \widetilde{y}}) \mapsto \skal{d_{(u,v)}X \pmat{x\\y},d_{(u,v)}N \pmat{\widetilde{x}\\\widetilde{y}}}; $
	 an jedem Punkt $ (u,v) $ erhält man so eine symmetrische Bilinearform 
	 
	 \[ \mathrm{II}\big|_{(u,v)}: \R^2 \times \R^2 \rightarrow \R, (\pmat{x_1 \\ y_1},\pmat{x_2 \\ y_2}) \mapsto - \skal{d_{(u,v)}X\pmat{x_1 \\ y_1},d_{(u,v)}N\pmat{x_2 \\ y_2}}. \]
	 
\end{lemma, definition}

\begin{proof}
	Dass $ \mathrm{II}\big|_{(u,v)} $ Bilinearform ist, ist klar. Weiters: 
		\[ {\color{Maroon} \mathrm{II}(\pmat{1 \\ 0},\pmat{0 \\ 1}) = }-\skal{X_u,N_v}= \skal{X_{uv},N}= \skal{X_{vu},N} = -\skal{X_v,N_u} {\color{Maroon} = \mathrm{II}(\pmat{0 \\ 1},\pmat{1 \\ 0}) }, \] 
	also ist $ \mathrm{II} $ symmetrisch an jedem Punkt. 
\end{proof}

\begin{remark}
	
	Ist die erste Fundamentalform gegeben, so kann die zweite Fundamentalform aus dem Weingartentensor berechnet werden - und umgekehrt.
	
	\textbf{Warnung:} Obwohl der Weingartentensor symmetrisch ist, ist seine Darstellungsmatrix (bzgl. $ (X_u,X_v) $) normalerweise \textbf{nicht} symmetrisch:
	
	\[ \pmat{s_{11} & s_{12} \\ s_{21} & s_{22}} = \frac{1}{EG-F^2} \pmat{G & -F \\ -F & E} \pmat{e & f \\ f & g} = \frac{1}{EG-F^2} \pmat{Ge-Ff & Gf-Fg \\ Ef-Fe & Eg-Ff} \]
	
\end{remark}

\begin{remark}
	Die Gauß-Krümmung ist gegeben durch: \[ K = \frac{eg-f^2}{EG-F^2} \]
\end{remark}

\subsection{Kovariante Ableitung und Krümmungstensor}

\begin{definition}
	
	Die \emph{kovariante Ableitung} eines Tangentialfeldes $ Y:M\rightarrow V $ entlang $ X:M\rightarrow \E $ ist der Tangentialteil seiner Ableitung 
		\[ \nabla Y := (dY)^T = dY - \skal{dY,N}N, \] wobei $ \nabla $ den \emph{Levi-Civita Zusammenhang} entlang $ X $ bezeichnet.
	
\end{definition}

\begin{remark}
	
	Wegen $ \skal{X_{uu},N}= -\skal{X_u,N_u}= e $ usw. bekommt man 
		\[ \nabla_{\dfrac{\partial}{\partial u}}Xu = X_{uu}-Ne \text{ und } \nabla_{\dfrac{\partial}{\partial v}} = X_{uv}-Nf \]
		\[ \nabla_{\dfrac{\partial}{\partial v}}Xv = X_{vv}-Ng \text{ und } \nabla_{\dfrac{\partial}{\partial u}} = X_{uv}-Nf \]
	
\end{remark}

\begin{lemma}
	
	Der L-C Zusammenhang erfüllt die Leibniz-Regel,
		\[ \nabla(Yx) = (\nabla Y)x + Ydx \text{ für } x \in C^\infty(M) \] 
	und ist \emph{metrisch}, d.h., 
		\[ d\skal{Y,Z} = \skal{\nabla Y,Z} + \skal{Y,\nabla Z} \]
	
\end{lemma}

\begin{proof}
	
	gemma Jojo
	
\end{proof}

\textbf{Matrixdarstellung:} Da die kovariante Ableitung eines TVfs $ Y $ tangential ist, erhält man für $ Y = X_v :$

		\[ \left(\nabla_{\dfrac{\partial}{\partial u}}X_u,\nabla_{\dfrac{\partial}{\partial u}}X_v \right) = (X_u,X_v)\Gamma_1 \] 
		\[ \left(\nabla_{\dfrac{\partial}{\partial u}}X_u,\nabla_{\dfrac{\partial}{\partial u}}X_v \right) = (X_u,X_v)\Gamma_2 \] 
	mit 
		\[ \Gamma_i = \pmat{\Gamma_{i1}^1 & \Gamma_{i2}^1 \\ \Gamma_{i1}^2 & \Gamma_{i2}^2}. \]
		
	Also, für eine allgemeine TVf $ Y=X_ux+X_vy=dX\pmat{x \\ y} $, 
		\[ \nabla_{\dfrac{\partial}{\partial u}}Y= \nabla_{\dfrac{\partial}{\partial u}}\left ( \left( X_u,X_v \right) \pmat{x \\y}    \right ) = \left( \nabla_{\dfrac{\partial}{\partial u}}X_u,\nabla_{\dfrac{\partial}{\partial u}}X_v \right) \pmat{x \\y} + X_u,X_v \pmat{x_u \\ y_u} \]

		\[ = \left( X_u, X_v \right) \Gamma_1\pmat{x \\ y} + \left( X_u, X_v \right) \pmat{x_u \\ y_u} = \left( X_u, X_v \right) \left( \dfrac{\partial}{\partial u} +\Gamma_1 \right) \pmat{x \\ y}    \]
		
	und ebenso 
		\[ \nabla_{\dfrac{\partial}{\partial v}} = \left( X_u, X_v \right) \left( \dfrac{\partial}{\partial v} +\Gamma_2 \right) \pmat{x \\ y},  \] 
	oder, anders gesagt:
		\[ \nabla_{\dfrac{\partial}{\partial u}} \circ dX = dX \circ \left( \dfrac{\partial}{\partial u} + \Gamma_1 \right)  \]
	und genauso
		\[  \nabla_{\dfrac{\partial}{\partial v}} \circ dX = dX \circ \left( \dfrac{\partial}{\partial v} + \Gamma_2 \right).   \]	
	
	Man bemerke: mit $ \pmat{x \\ y} = \pmat{1 \\ 0} \cong \dfrac{\partial}{\partial u} $ erhält man
		\[ \nabla_{\dfrac{\partial}{\partial u}}\left( \left( X_u,X_v \right) \pmat{1 \\ 0} \right) = \nabla_{\dfrac{\partial}{\partial u}}X_u = X_u \Gamma_{11}^1 + X_v\Gamma_{11}^2 = \left( X_u,X_v \right) \Gamma_1\pmat{1 \\ 0}   \] 
	und
		\[ \nabla_{\dfrac{\partial}{\partial v}} \left( \left( X_u,X_v \right) \pmat{1 \\ 0} \right) = \nabla_{\dfrac{\partial}{\partial v}}X_u = \left( X_u,X_v \right) \Gamma_2\pmat{1 \\ 0} \]
	und ebenso für  $ \pmat{x \\ y} = \pmat{0 \\ 1} \cong \dfrac{\partial}{\partial v} $, also ist {\color{Cyan} STERN EIN SPEZIALFALL VON STERNSTERN}
	
	\begin{remark}
		\textbf{WARUNUNG:} $ \nabla_{\dfrac{\partial}{\partial u}} $ und $ \nabla_{\dfrac{\partial}{\partial v}} $ sind \emph{Differentialoperatoren} (keine Endomorphismen) - trotzdem benutzen wir Matrizen, um sie zu beschreiben.
	\end{remark}

\begin{lemma, definition}
	
	$\Gamma_{ij}^k$ heißen die \emph{Christoffel-Symbole} von $ X $; sie sind symmetrisch: 
		\[ \Gamma_{ij}^k = \Gamma_{ji}^k \]
	
\end{lemma, definition}

\begin{proof}
	
		\[ X_u\Gamma_{12}^1 + X_v \Gamma_{12}^2 = \nabla_{\dfrac{\partial}{\partial u}}X_v = \left( X_{vu} \right)^T = \left( X_{uv} \right)^T = \nabla_{\dfrac{\partial}{\partial v}}X_u = X_u\Gamma_{21}^1 + X_v\Gamma_{21}^2   \]
	also $ \Gamma_{12}^1 = \Gamma_{21}^1 $ und $ \Gamma_{12}^2 = \Gamma_{21}^2 $.
\end{proof}

\begin{theorem}[Koszul's Formeln]
	
	Mit Matrizen 
		\[ I = \pmat{E &f \\ F &G} \text{ und } J = \pmat{0 &-1 \\ 1 & 0 } \]
	gilt: 
		\[ \frac12 I_u - \frac{E_v - F_u}{2}J = I\Gamma_1, \]
		\[ \frac12 I_v + \frac{G_u - F_v}{2}J = I\Gamma_2. \]
		
\end{theorem}

\begin{proof}
	
	Multipliziere $ \nabla_{\dfrac{\partial}{\partial u}}X_u = X_u\Gamma_{11}^1 + X_v\Gamma_{11}^2 $ mit $ Y_u $ und $ X_v $, um zu erhalten:
		\[ E \Gamma_{11}^1 + F \Gamma_{11}^2 = \skal{X_u, \nabla_{\dfrac{\partial}{\partial u}}X_u} = \frac12 \skal{X_u,X_u}_u = \frac12 E \]
	
		\[ F\Gamma_{11}^1 + G \Gamma_{11}^2 = \skal{X_v, \nabla_{\dfrac{\partial}{\partial u}}X_u} = \skal{X_v,X_u}_u - \skal{\nabla_{\dfrac{\partial}{\partial u}}X_v,X_u} = \skal{X_v,X_u}_u - \skal{\nabla_{\dfrac{\partial}{\partial v}}X_u,X_u} \]
		\[ = F_u - \frac12 E_v. \]
	
	Insgesamt also
		\[ \pmat{E &F \\F &G} \pmat{\Gamma_{11}^1 & \Gamma_{12}^1 \\ \Gamma_{11}^2 & \Gamma_{12}^2} = \frac12 \pmat{E_u \\ F_u} - \dfrac{E_v - F_u}{2} \pmat{0 \\ 1} \]
		\[ = \left( \frac12 \pmat{E_u & F_u \\ F_u & G_u} - \dfrac{E_v - F_u}{2} \pmat{0 &-1 \\ 1 & 0 } \right).  \]
	
	Die anderen Teile der Gleichung  folgen ebenso.
	
\end{proof}

\begin{corollary}
	Die Christoffel - Symbole $ \Gamma_{ij}^k $ hängen nur von der induzierten Metrik ab.
\end{corollary}

\begin{example}
	
	Für eine isometrische Parametrisierung, $  E = G = 1 $ und $  F = 0 $, erhält man mit den Konszul-Formeln $ \Gamma_{ij}^k = 0 $.
	
\end{example}


\begin{definition}
	
	Für ein TVf $ Y : M \rightarrow V $ entlang $ X: M \rightarrow \E $ definieren wir den \emph{(Riemannschen) Krümmungstensor} $ R $ von $X$ durch 
		\[ RY := \nabla_{\dfrac{\partial}{\partial u}}\nabla_{\dfrac{\partial}{\partial v}}Y - \nabla_{\dfrac{\partial}{\partial v}}\nabla_{\dfrac{\partial}{\partial u}}Y. \]
	
\end{definition}

\begin{remark}
	
	Dies ist eine vereinfachte Form, für Flächen, des ''wahren'' Krümmungstensors.
	
\end{remark}

\begin{lemma}
	
	$ R $ ist ein schiefsymmetrischer Tensor des Tangentialbündels $ TX $, d.h., 
	$R\big|_{(u,v)} \in \mathrm{End}(T_{(u,v)}X) $ ist schiefsymmetrisch für jedes $ (u,v) \in M $, und 
		\[ R(Yx) = (RY)x \text{ für } x \in C^\infty(M). \]
	
\end{lemma}

\begin{proof}
	
	$R\big|_{(u,v)} \in \mathrm{End(T_{(u,v)}X)}$ ist klar.
	Schiefsymmetrie:
		\[\frac12 (\abs{Y}^2)_{uv} = \skal{Y \nabla_{\dfrac{\partial}{\partial u}}Y}_v = \skal{Y, \nabla_{\dfrac{\partial}{\partial v}}\nabla_{\dfrac{\partial}{\partial u}}Y} + \skal{\nabla_{\dfrac{\partial}{\partial v}}Y,\nabla_{\dfrac{\partial}{\partial u}}Y} \]
		\[ = \skal{Y, \nabla_{\dfrac{\partial}{\partial u}}\nabla_{\dfrac{\partial}{\partial v}}Y} + \skal{\nabla_{\dfrac{\partial}{\partial u}}Y, \nabla_{\dfrac{\partial}{\partial v}}Y} = \frac12 \left( \abs{Y}^2\right)  \]
		\[ \Rightarrow \skal{Y,RY} = \frac12 \left(\abs{Y}^2\right)_{vu}- \frac12\left(\abs{Y}^2\right)_{uv}=0 \]
	Tensoreigenschaft:
		\[ \nabla_{\dfrac{\partial}{\partial u}}\nabla_{\dfrac{\partial}{\partial v}}(Yx)= \nabla_{\dfrac{\partial}{\partial u}}\left(\left( \nabla_{\dfrac{\partial}{\partial v}}Y \right)x + Yx_v\right) \]
		\[ = \left( \nabla_{\dfrac{\partial}{\partial u}} \nabla_{\dfrac{\partial}{\partial v}} Y \right)x + \underbrace{\left( \nabla_{\dfrac{\partial}{\partial v}}Y \right)x_u + \left( \nabla_{\dfrac{\partial}{\partial u}}Y \right)x_v + Yx_{uv}}_{\text{symmetrisch in } \dfrac{\partial}{\partial u} \text{ und } \dfrac{\partial}{\partial v}} \]
	also
		\[ R(Yx)=(RY)x. \]
	
\end{proof}
